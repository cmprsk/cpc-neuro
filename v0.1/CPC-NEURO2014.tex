\documentclass[11pt,a4paper,oldfontcommands]{memoir}
\usepackage[utf8]{inputenc}
\usepackage[T1]{fontenc}
\usepackage{microtype}
\usepackage[dvips]{graphicx}
\usepackage{xcolor}
\usepackage{times}
\usepackage[brazilian]{babel}
\usepackage{indentfirst}
\usepackage{multirow}
\usepackage{caption}
\usepackage{longtable}

\usepackage[
breaklinks=true,colorlinks=true,
%linkcolor=blue,urlcolor=blue,citecolor=blue,% PDF VIEW
linkcolor=black,urlcolor=black,citecolor=black,% PRINT
bookmarks=true,bookmarksopenlevel=2]{hyperref}

\usepackage{geometry}
% PDF VIEW
% \geometry{total={210mm,297mm},
% left=25mm,right=25mm,%
% bindingoffset=0mm, top=25mm,bottom=25mm}
% PRINT
\geometry{total={210mm,297mm},
left=20mm,right=20mm,
bindingoffset=10mm, top=25mm,bottom=25mm}

\OnehalfSpacing
%\linespread{1.3}

%%% CHAPTER'S STYLE
\chapterstyle{bianchi}
%\chapterstyle{ger}
%\chapterstyle{madsen}
%\chapterstyle{ell}
%%% STYLE OF SECTIONS, SUBSECTIONS, AND SUBSUBSECTIONS
\setsecheadstyle{\Large\bfseries\sffamily\raggedright}
\setsubsecheadstyle{\large\bfseries\sffamily\raggedright}
\setsubsubsecheadstyle{\bfseries\sffamily\raggedright}


%%% STYLE OF PAGES NUMBERING
%\pagestyle{companion}\nouppercaseheads 
%\pagestyle{headings}
%\pagestyle{Ruled}
\pagestyle{plain}
\makepagestyle{plain}
\makeevenfoot{plain}{\thepage}{}{}
\makeoddfoot{plain}{}{}{\thepage}
\makeevenhead{plain}{}{}{}
\makeoddhead{plain}{}{}{}


\maxsecnumdepth{subsection} % chapters, sections, and subsections are numbered
\maxtocdepth{subsection} % chapters, sections, and subsections are in the Table of Contents


%%%---%%%---%%%---%%%---%%%---%%%---%%%---%%%---%%%---%%%---%%%---%%%---%%%

\begin{document}

%%%---%%%---%%%---%%%---%%%---%%%---%%%---%%%---%%%---%%%---%%%---%%%---%%%
%   TITLEPAGE
%
%   due to variety of titlepage schemes it is probably better to make titlepage manually
%
%%%---%%%---%%%---%%%---%%%---%%%---%%%---%%%---%%%---%%%---%%%---%%%---%%%
\thispagestyle{empty}

{%%%
\sffamily
\centering
\Large

{\large
Hospital Infantil Albert Sabin - Secretaria de Saúde do Estado do Ceará
}

{\LARGE
Centro Pediátrico do Câncer
}

~\vspace{\fill}

{\huge 
Protocolos de Quimioterapia
}

{\Large
Neuro-oncologia
}

\vspace{2.5cm}

{\LARGE
Francisco Hélder Cavalcante Félix\\
Juvenia Bezerra Fontenele
}

\vspace{3.5cm}

Manual de tratamento clínico de pacientes pediátricos com tumores cerebrais\\[1em]

Versão 0.1\\
\small Código: http://bit.ly/fhcflx-2ifVArn

\vspace{3.5cm}

%Revisão: Dra. Nádia M. Trompieri

\vspace{\fill}

Janeiro 2017

%%%
}%%%

\cleardoublepage
%%%---%%%---%%%---%%%---%%%---%%%---%%%---%%%---%%%---%%%---%%%---%%%---%%%
%%%---%%%---%%%---%%%---%%%---%%%---%%%---%%%---%%%---%%%---%%%---%%%---%%%

\tableofcontents*

\clearpage

%%%---%%%---%%%---%%%---%%%---%%%---%%%---%%%---%%%---%%%---%%%---%%%---%%%
%%%---%%%---%%%---%%%---%%%---%%%---%%%---%%%---%%%---%%%---%%%---%%%---%%%

\chapter{Introdução}

\begin{center}
\begin{table}
\renewcommand{\arraystretch}{1.5}
 \caption{\tiny incidência relativa de grupos histológicos de tumores cerebrais em criança e adolescentes relatada no Brasil. A classificação está de acordo com a ICCC. A última coluna mostra dados não publicados de nosso serviço. P = Tinho et al, 2011; R = Rosemberg et al, 2007; H = HIAS, não publicado; NI = não informado.}
\begin{tabular}{c|c|ccc|c|c}
	\hline
	\multicolumn{4}{c|}{Tipos tumorais (histologia)}&{P}&{R}&{H}	\\
	\hline
	\multicolumn{1}{c|}{\multirow{15}{*}{Neuroepiteliais}}&{\multirow{7}{*}{Gliomas}}&{\multirow{3}{*}{Astrocitomas}}&\multicolumn{1}{|c|}{Pilocítico}&{\multirow{3}{*}{37\%}}&{18,2\%}&{10,6\%}
	\\
	\cline{4-4}\cline{6-7}
    \multicolumn{1}{c|}{}&&&\multicolumn{1}{|c|}{Difuso}&&{6,2\%}&{7\%}\\
	\cline{4-4}\cline{6-7}
    \multicolumn{1}{c|}{}&&&\multicolumn{1}{|c|}{Anaplásico}&&{4,4\%}&{2,2\%}\\
	\cline{3-7}
    \multicolumn{1}{c|}{}&{}&\multicolumn{2}{c|}{Oligodendrogliomas}&{NI}&{0,9\%}&{0,9\%}\\
	\cline{3-7}    
    \multicolumn{1}{c|}{}&&{\multirow{2}{*}{Ependimomas}}&\multicolumn{1}{|c|}{Clássico}&{\multirow{2}{*}{6,8\%}}&{\multirow{2}{*}{7,4\%}}&{7,9\%}\\
    \cline{4-4}\cline{7-7}
    \multicolumn{1}{c|}{}&&&\multicolumn{1}{|c|}{Anaplásico}&&&{4\%}\\
	\cline{3-7}    
    \multicolumn{1}{c|}{}&&\multicolumn{2}{c|}{Glioblastoma}&{NI}&{3,7\%}&{3,5\%}\\
    \cline{2-7}
    \multicolumn{1}{c|}{}&{\multirow{6}{*}{Embrionários}}&{\multirow{3}{*}{Meduloblastomas}}&\multicolumn{1}{|c|}{Clássico}&{\multirow{3}{*}{13,6\%}}&{\multirow{3}{*}{11,2\%}}&{21\%}\\
	\cline{4-4}\cline{7-7}
    \multicolumn{1}{c|}{}&&&\multicolumn{1}{|c|}{Desmoplásico}&&&{3,5\%}\\
	\cline{4-4}\cline{7-7}
    \multicolumn{1}{c|}{}&&&\multicolumn{1}{|c|}{Anaplásico}&&&{0,4\%}\\
	\cline{3-7}
    \multicolumn{1}{c|}{}&{}&\multicolumn{2}{c|}{Pineoblastomas}&{\multirow{3}{*}{3,9\%}}&{NI}&{0,4\%}\\
	\cline{3-4}\cline{6-7} 
    \multicolumn{1}{c|}{}&&{\multirow{2}{*}{PNET}}&\multicolumn{1}{|c|}{Supratentorial}&&{\multirow{2}{*}{2,7\%}}&{1,3\%}\\
    \cline{4-4}\cline{7-7}
    \multicolumn{1}{c|}{}&&&\multicolumn{1}{|c|}{Outros}&&&{0,9\%}\\
    \cline{2-7}
    \multicolumn{1}{c|}{}&{\multirow{2}{*}{Neural}}&\multicolumn{2}{c|}{Ganglioglioma}&{\multirow{2}{*}{NI}}&{4,6\%}&{0,8\%}\\
    \cline{3-4}\cline{6-7}
    \multicolumn{1}{c|}{}&&\multicolumn{2}{c|}{Neurocitoma, DNET, ganglioneuroma}&&{3\%}&{0\%}\\
    \hline
	\multicolumn{1}{c|}{Meninges}&\multicolumn{3}{c|}{Meningioma}&{NI}&{3\%}&{NI}\\
	\hline
	\multicolumn{1}{c|}{Endócrino}&\multicolumn{3}{c|}{Craniofaringioma}&{10,5\%}&{11\%}&{NI}\\
	\hline
	\multicolumn{1}{c|}{TCG}&\multicolumn{3}{c|}{Tumores de células germinativas}&{6,1}&{3,6\%}&{NI}\\
    \hline
\end{tabular}
\end{table}
\end{center}

\section{Tumores cerebrais na infância e adolescência}
\begin{center}
\begin{figure}[htpb]
\includegraphics[scale=0.9,trim = 30mm 10mm 8mm 15mm,clip]{../fig/fig1.pdf}
\caption{Tratamento de crianças com tumores cerebrais, com histologia}
%\label{Rotulo}
\end{figure}
\begin{figure}[!htb]
\includegraphics[trim = 20mm 50mm 10mm 13mm,clip]{../fig/fig2.pdf}
\caption{Tratamento de crianças com tumores cerebrais, sem histologia}
%\label{Rotulo}
\end{figure}
\end{center}

Os tumores cerebrais são um grupo heterogêneo de doenças neoplásicas de comportamento variável, com as características comuns de relativa raridade, elevada morbidade e elevada mortalidade. Dentre as neoplasias infantis, no entanto, constituem (como um grupo) o primeiro tumor sólido e a segunda neoplasia maligna mais frequente nas crianças, atrás apenas das leucemias, perfazendo em torno de \(20\%\) das neoplasias pediátricas. A sua incidência varia de acordo com a região do mundo. Nos EUA, a incidência anual ajustada para a idade de tumores cerebrais malignos primários na população de \(0-15\) anos foi de \(3,4\) por \(10^5\) pessoas-ano entre \(2004-2008\) \cite{Ostrom01102014}. Já na Europa, entre \(1988-1997\), a incidência reportada foi de \(2,99\) por \(10^5\) \cite{Peris-Bonet}. Esta incidência é mais alta do que a usualmente reportada na Ásia, onde relatos indicam entre \(1,8-2,2\) casos por \(10^5\) \cite{CNCR21430}. No Brasil, o primeiro relato do Registro de Câncer de Base Populacional indicou uma incidência de \(0,9\) a \(3,2\) por \(10^5\), semelhante à estatística do mundo desenvolvido ocidental \cite{IJC24799}. Fortaleza teve uma das menores incidências relatadas, \(1,3\) casos por \(10^5\), o que pode indicar subdiagnóstico. Hoje em dia, no entanto, já não é apropriado falar em “tumores cerebrais” infantis, sem separar as diversas entidades patológicas entre si, as quais têm incidência, tratamento e prognóstico muito díspares.

Os tumores cerebrais mais frequentes em crianças são os astrocitomas pilocíticos, tumores de comportamento incerto, ora classificados como benignos, ora como malignos. Eles representam em torno de 1\(8\%\) dos tumores cerebrais infantis. Em seguida, vem os tumores embrionários, a maior parte dos quais meduloblastomas, os tumores malignos mais comuns da infància, que representam em torno de \(15\%\) dos diagnósticos de tumor cerebral em crianças \cite{Ostrom01102014}. Astrocitomas pilocíticos são tumores indolentes, de crescimento lento, tratados principalmente pela ressecção cirúrgica, a qual é curativa na maioria dos casos, com pouca probabilidade de disseminação e virtualmente ausência de transformação maligna \cite{gan}. Já os meduloblastomas são tumores indiferenciados, com elevado índice mitótico, com acentuada propensão à disseminação e recidiva, necessitando de terapia adjuvante com radioquimioterapia após ressecção cirúrgica \cite{partap}. Estes dois tipos tumorais, que juntos correspondem a mais de \(30\%\) dos casos de tumores cerebrais em crianças, têm hoje um excelente prognóstico quando comparado ao passado. Outros tipos tumorais menos frequentes, todavia, têm resultados menos brilhantes com o tratamento atualmente disponível. Tumores de tronco cerebral, normalmente não biopsiados na sua maioria, constituem cerca de \(10\%\) dos tumores cerebrais infantis, e têm um prognóstico extremamente reservado, com apenas um subgrupo pequeno de pacientes com tumores neste sítio alcançando sobrevida prolongada.

O tratamento de tumores cerebrais em crianças e adolescentes evoluiu significantemente nas últimas décadas. Dos anos 80 até hoje, o conhecimento sobre o papel das várias modalidades de terapia (cirurgia, radioterapia e quimioterapia) ficou mais claro e programas terapêuticos específicos para cada tipo de doença puderam ser desenvolvidos. Hoje em dia, a maioria das crianças com um diagnóstico de tumor cerebral conseguirá ser adequadamente tratada e alcançará sobrevida prolongada. O manejo dos efeitos colaterais a longo prazo da terapia e das sequelas da doença são as principais preocupações na neuro-oncologia pediátrica moderna \cite{merchant}. No Brasil, estudos de sobrevida de pacientes pediátricos com tumores cerebrais são raros. Nosso grupo publicou recentemente uma análise de sobrevida de \(103\) pacientes pediátricos diagnosticados com tumores cerebrais entre 2000 e 2006 num único centro hospitalar, mostrando resultados que se assemelham aqueles dos registros populacionais dos EUA e Europa para as principais patologias \cite{araujo}.

Os protocolos listados foram avaliados e qualificados segundo a nova classificação de níveis de evidência da OCEBM \cite{ocebm}. A partir desta classificação, foram selecionados os tratamentos com maior qualidade de evidência, os quais podem ser recomendados rotineiramente. Lacunas no conhecimento atual foram listadas (não exaustivamente). De acordo com a classificação 2011 da OCEBM, os ensaios controlados e randomizados são considerados evidência de nível 2, enquanto os estudos não controlados e séries de casos (equivalentes) são considerados evidência de nível 4 (tratamento). Nenhum trabalho com nível 3 de evidência (controlados, porém não randomizados) foi encontrado. Alguns ensaios foram desenhados para obter informações sobre história natural da doença. Grandes coortes para estudo de prognóstico (\textit{inception cohort}) são consideradas nível 2 de evidência, enquanto coortes de qualidade menor ou grupos controle de ensaios randomizados são nível 3.

\section{Por quê criar um compêndio de protocolos de quimioterapia?}
O Hospital Infantil Albert Sabin (HIAS) é uma instituição hospitalar da administração direta da saúde da Secretaria de Saúde do Estado do Ceará, habilitado como unidade de assistência de alta complexidade em neurologia/neurocirurgia, UNACON exclusiva de oncologia pediátrica, UTI pediátrica nível II e hospital de ensino, nível de atenção de alta complexidade, atendendo pelo SUS \cite{cnes}. O Centro Pediátrico do Câncer é o anexo do HIAS onde o tratamento oncológico clínico é realizado, contando ainda com equipe multiprofissional de atenção às crianças com câncer. Tem \(22\) leitos de internação em enfermaria (\(2\) isolamentos), \(06\) leitos de UTIP, e \(05\) consultórios para atendimento ambulatorial. O ambulatório e a enfermaria contam com material para atendimento às urgências e emergências, incluindo carrinho de emergência completo com drogas e equipamento para reanimação. O CPC conta com plantão médico 24h por dia.

O HIAS-CPC é referência estadual para o tratamento de crianças com tumores cerebrais, servindo uma população de 8,8 milhões de habitantes (um e meio milhão de crianças e jovens até 18 anos) \cite{estat}. A incidência ajustada para a idade de tumores cerebrais pediátricos no Ceará foi estimada em \(1,3\) casos por \(10^5\), entre 1998 e 2002 \cite{inca}. Seu papel é fundamental para o diagnóstico, tratamento e acompanhamento de centenas de crianças com câncer, incluindo tumores cerebrais. O HIAS-CPC recebeu cerca de \(35\) novos pacientes com tumores cerebrais ao ano entre 2007 e 2013 (um total de \(250\)). Isto indica que a esmagadora maioria das crianças com esta doença no estado do Ceará são tratadas no HIAS-CPC. Dessa forma, torna-se imprescindível que a qualidade da atenção à saúde dispensada a estes pequenos pacientes em nosso serviço hospitalar seja continuamente revisada, avaliada e padronizada.



\chapter{O tratamento de tumores cerebrais em crianças}

\section{Gliomas de baixo grau}

\subsection{Avaliação da literatura}

\begin{figure}[!htb]
\includegraphics[trim = 18mm 30mm 15mm 12mm,clip]{../fig/fig3.pdf}
\caption{Tratamento de crianças com gliomas de baixo grau.}
%\label{Rotulo}
\end{figure}

Esse grupo inclui os astrocitomas, oligodendrogliomas, gangliogliomas e tumores gliais mistos ou variantes (grupos IIIa, b e d da ICCC3) \cite{CNCR20910}. A denominação de baixo grau refere-se à classificação da OMS para tumores do sistema nervoso central, a qual divide as neoplasias em 4 grupos, baseada em critérios histológicos. A classificação da OMS para tumores do sistema nervoso central constitui uma “escala de malignidade”, mais do que um esquema de estadiamento convencional \cite{louis}. Os tumores classificados como Grau I ou II são coletivamente denominados tumores de baixo grau de malignidade, enquanto aqueles classificados como Grau III ou IV são designados tumores de alto grau de malignidade. Os tumores de baixo grau são comumente tratados apenas cirurgicamente, com elevados índices de cura e sobrevida prolongada. Tumores astrocíticos e oligodendrogliais de baixo grau têm bom prognóstico associado à ressecção cirúrgica como única terapia. Todavia, a possibilidade de ressecção cirúrgica completa varia muito de acordo com o sítio tumoral \cite{wisof}.
Gliomas cerebelares são passíveis de ressecção completa em \(60-70\%\) dos casos, a maioria são astrocitomas pilocíticos (grau I) e seu comportamento é praticamente benigno \cite{gan}. A recidiva após ressecção e a progressão para tumores de maior grau de malignidade são muito raras. Mesmo tumores incompletamente ressecados mostram uma baixa propensão a progredir. A sobrevida livre de progressão em 5 anos após a cirurgia é de \(84-91\%\). A sobrevida livre de progressão em 5 anos em pacientes com doença residual é de \(54-63\%\) \cite{wisof}. Gliomas da via óptica e hipotálamo (e demais tumores diencefálicos ou da linha média) são lesões difusas, infiltrativas, em sua maioria astrocitomas de baixo grau (pilocítico ou difuso), com maior chance de disseminação e metástase no neuro-eixo, com maior incidência em pacientes com neurofibromatose tipo 1. Devido a sua natureza infiltrativa e ao risco de sequelas visuais e neuro-endócrinas, a ressecção cirúrgica não é realizada na maioria dos casos e a biópsia somente está indicada nos casos de imagem atípica. A maioria dos pacientes é tratada com base apenas em imagens sugestivas. Apesar de sua histologia, têm um prognóstico mais reservado do que os pacientes com lesões cerebelares \cite{gan}. A sobrevida livre de progressão em 5 anos é de \(47\%\) \cite{wisof}. Oligodendrogliomas, tumores mistos e variantes de tumores astrocíticos são raros em crianças (\(1\%\) ou menos de todos os tumores cerebrais). São tumores da substância branca supratentorial, infiltrativos, e o controle cirúrgico é curativo na maioria. Terapia adjuvante não está bem definida para estes tumores \cite{gan}. A sobrevida livre de progressão em 5 anos é de \(67\%\) \cite{wisof}.

O papel da cirurgia no controle dos gliomas de baixo grau está bem estabelecido. O estudo prospectivo multi-institucional do Children’s Oncology Group (COG) CCG9891 avaliou uma coorte de \(518\) pacientes diagnosticados com tumores de origem glial, tratados inicialmente com ressecção cirúrgica. Ocorreu revisão central da histologia de todos os casos incluídos. Do total, \(64\%\) dos pacientes não tinha evidência de doença residual após a cirurgia, \(20\%\) tinha doença residual limitada (\(< 1,5 cm^3\)) e \(16\%\) tinha doença residual significante (\(> 1,5 cm^3\)). A maioria dos pacientes (\(76\%\)) tinha astrocitoma pilocítico, \(6\%\) astrocitoma difuso, \(8\%\) ganglioglioma e \(10\%\) oligodendroglioma, tumores mistos ou variantes. A maioria dos pacientes (\(73\%\)) tinha 5 anos ou mais. A maioria (\(57\%\)) tinha tumores cerebelares, \(24\%\) de hemisférios cerebrais, \(14\%\) da linha média e \(4\%\) das vias ópticas ou hipotálamo. A sobrevida livre de progressão em 5 anos de toda a coorte foi de \(80\%\), sendo \(84 \:a\: 91\%\) para tumores cerebelares, \(78\%\) para hemisférios cerebrais, \(65\%\) para a linha média e \(47\%\) para vias ópticas ou hipotálamo (\(p<0,001\)) (nível 2). A sobrevida livre de progressão em 5 anos foi de \(83\%\) para astrocitomas pilocíticos, \(88\%\) para gangliogliomas, \(66\%\) para astrocitomas difusos e \(67\%\) para outros tumores (\(p=0,64\)) (nível 2). Finalmente, a ressecção cirúrgica completa foi o fator isolado de maior impacto na progressão nesta coorte, \(94\%\) dos pacientes com ressecção completa estavam livres de progressão após 5 anos, enquanto \(59\%\) dos pacientes com doença residual limitada e \(53\%\) dos pacientes com doença residual significante alcançaram sobrevida livre de progressão prolongada (\(p<0,01\)) (nível 2). A conclusão é de que a ressecção completa deve ser tentada sempre que possível (ou seja, desde que não acarrete comprometimento funcional) para os pacientes pediátricos com gliomas de baixo grau (nível 4). Além disso, o fato de que mais de \(50\%\) dos pacientes com doença residual não progrediram em 5 anos indica que as intervenções terapêuticas adjuvantes devem ser postergadas até que ocorra progressão objetiva da doença. No entanto, apesar de sua histologia aparentemente benigna, \(44\%\) dos pacientes progrediram mesmo com doença residual muito limitada, o que indica a necessidade de monitorização dos pacientes com ressecção incompleta, independente da quantidade de tumor residual (nível 2) \cite{wisof}.

Fica evidente que um número significativo de pacientes pediátricos com gliomas de baixo grau sofre recidiva após controle cirúrgico ou não pode ter seu tumor ressecado. Nestes casos, indica-se terapia adjuvante com a intenção de evitar a progressão da doença. Vários estudos exploraram a contribuição da radioterapia e quimioterapia no tratamento de gliomas de baixo grau progressivos. Um ensaio fase II não controlado estudou 78 crianças com gliomas de baixo grau tratadas com radioterapia conformacional. Os pacientes tinham astrocitoma pilocítico (\(n=50\)), tumores de via óptica ou hipotálamo sem biópsia (\(n=13\)), astrocitoma difuso (\(n=4\)), ganglioglioma (\(n=3\)) e oligodendroglioma, tumores mistos ou variantes (\(n=8\)). A maioria dos tumores localizava-se no diencéfalo (\(47\)), \(17\) no cerebelo e \(3\) nos hemisférios cerebrais. Treze pacientes tinham NF-1, \(25\) receberam QT previamente e \(65\) sofreram cirurgia (biópsia ou ressecção incompleta). O tratamento foi indicado nos pacientes sintomáticos na avaliação inicial ou com evidência radiológica de progressão ou, ainda, com uma lesão residual numa área de risco para progressão. Dentre os pacientes cujo tratamento primário foi radioterapia, mais da metade iniciou o tratamento em menos de 90 dias após o diagnóstico. A sobrevida livre de progressão em 5 anos do grupo foi de \(87\%\). Treze pacientes apresentaram progressão com uma mediana de tempo de \(83\) meses. Quatro pacientes apresentaram falha terapêutica, desenvolvendo doença metastática. Não ocorreu diferença digna de nota entre os tipos histológicos (nível 2). Nenhum dos pacientes com NF-1 teve progressão ou malignização. Um paciente da série desenvolveu um glioma de alto grau na região do campo de irradiação, \(78\) meses após o tratamento. A incidência cumulativa de vasculopatia na série foi de cerca de \(5\%\) em 7 anos e o principal fator de risco para esta complicação foi a idade menor que 5 anos (nível 2) \cite{Merchant01082009}. Em relação aos efeitos cognitivos, um declínio de \(10\) pontos de QI foi estimado para crianças com 5 anos de idade ao tratamento, 5 anos após a radioterapia. O risco cumulativo de desenvolver insuficiência tireoidiana foi de \(64\%\) e de deficiência de GH foi de \(49\%\), em 10 anos. A incidência cumulativa de déficit auditivo foi de cerca de \(6\%\) em 10 anos. A presença de NF-1 foi um fator de risco para vasculopatia e déficit cognitivo (nível 2) \cite{Merchant01082009.2}. Em conclusão, esta série mostrou inequivocamente que a radioterapia pode controlar adequadamente os gliomas de baixo grau pediátricos não controlados cirurgicamente, com uma elevada proporção de pacientes tendo sobrevida prolongada sem progressão (nível 4). No entanto, isso ocorre às custas de frequentes efeitos colaterais, provavelmente permanentes. A radioterapia para gliomas de baixo grau deve ser evitada em pacientes com menos de 5 anos, devido ao risco de vasculopatia (nível 2). Apesar do risco cumulativo de déficit auditivo ser baixo e do fato do declínio cognitivo ser menor com o avançar da idade, adiar a radioterapia o quanto for possível parece razoável.

Com o intuito de atrasar o início da radioterapia, vários estudos foram realizados com diferentes esquemas de quimioterapia em crianças com gliomas de baixo grau recorrentes ou progressivos. As combinações mais utilizadas foram: carboplatina e vincristina \cite{packer,gnekow}; procarbazina, tioguanina, lomustina e vincristina (TPCV) \cite{prados}; cisplatina e etoposido \cite{mass}. Packer et al trataram \(78\) pacientes até 15 anos com gliomas de baixo grau confirmados por histologia ou imagem típica, progressivos, reportando \(56\%\) de resposta radiológica objetiva e \(68\%\) de sobrevida livre de progressão em 3 anos (nível 4). A maioria dos pacientes (\(n=32\)) tinha astrocitoma fibrilar (difuso), \(17\) tinham astrocitoma pilocítico e \(26\) não tinham histologia. A maioria dos pacientes tinha tumores diencefálicos (\(n=52\)), \(12\) tinham no tronco e 6 em outros locais. Somente pacientes que sofreram ressecção de \(50\%\) ou menos das lesões foram admitidos. Não ocorreu revisão central de histologia ou imagens. Este ensaio clínico não avaliou se o esquema conseguia adiar o início da radioterapia, principal motivo do tratamento, devido ao curto tempo de seguimento \cite{packer}. A carboplatina fora testada pelo Pediatric Oncology Group (POG), em comparação com a iproplatina, num ensaio fase II, randomizado. Um grupo de pacientes pediátricos com tumores cerebrais histologicamente verificados, recorrentes ou progressivos, foi avaliado. O subgrupo de pacientes com astrocitoma de baixo grau (\(12\) pacientes, agregando pacientes de um ensaio não randomizado prévio do POG) não mostrou resposta radiológica objetiva, mas a maioria dos pacientes apresentou estabilização prolongada da doença com a carboplatina, o que motivou os pesquisadores a testá-la num grupo maior \cite{fried}. O ensaio não randomizado HIT-LGG 1996, do grupo de pediatria oncológica dos países de língua alemã (GPOH) utilizou um esquema de carboplatina e vincristina diferente daquele do COG. Um relato do subgrupo com gliomas hipotalâmico-quiasmáticos que recebeu quimioterapia (\(n=123\)) mostrou sobrevida livre de progressão de \(61\%\) em 5 anos \cite{gnekow} (nível 4). Os resultados completos do ensaio ainda não foram publicados, mas foram apresentados na reunião anual da Société Internationale d’Oncologie Pédiatrique (SIOP) de 2005. A sobrevida livre de eventos em 5 anos relatada foi de \(43\%\) para o grupo tratado com quimioterapia \cite{gnekow2}. Com o intuito de tentar melhorar estes resultados, a SIOP e o GPOH iniciaram conjuntamente o ensaio SIOP-LGG 2004, o qual terminou de cadastrar pacientes em 2012. Este ensaio randomizado compara carboplatina e vincristina com carboplatina, vincristina e etoposide \cite{gnekow3}. Em 2002, um grupo italiano relatou um grupo de 34 crianças com gliomas de baixo grau não ressecáveis, a maior parte hipotalâmico-quiasmáticos (\(n=29\)), tratadas com cisplatina e etoposide. Eles mostraram uma sobrevida livre de progressão de \(78\%\) em 3 anos, com \(11\) pacientes obtendo remissão parcial e 1 completa (nível 4). No entanto, uma quantidade significativa de pacientes apresentou toxicidade auditiva, um efeito colateral conhecido da cisplatina \cite{mass}.

Prados \textit{et al}  trataram \(42\) crianças até 18 anos com gliomas de baixo grau histologicamente confirmados (exceto tumores de diencéfalo em pacientes com NF-1 ou de vias ópticas), com doença progressiva. A sobrevida livre de progressão foi de \(45\%\) em 3 anos, com mediana de \(2,5\) anos para progressão (nível 4). A maioria dos pacientes tinha astrocitoma pilocítico (\(n=23\)), \(11\) tinham astrocitoma (sem outra especificação), \(6\) não tinham histologia e \(2\) tinham oligodendroglioma ou ganglioglioma. A maioria dos pacientes tinham tumores hipotalâmicos ou quiasmáticos (\(n=33\)), \(4\) talâmicos e \(5\) em outras localizações. A maioria dos pacientes sofreu ressecção parcial ou subtotal. Este esquema foi derivado de experimentos pré-clínicos que mostraram que a combinação das drogas utilizadas tinha efeitos sinérgicos nas células neoplásicas \cite{prados}. Apesar da aparente superioridade da combinação carboplatina e vincristina, os ensaios tinham grandes diferenças entre si quanto aos diagnósticos histológicos e topográficos dos pacientes, além de diferenças na terapia prévia. Para definir qual o melhor dentre os dois esquemas, um ensaio fase III randomizado foi levado a cabo pelo COG e seus resultados publicados recentemente \cite{Ater20072012}. O estudo avaliou \(274\) pacientes com 10 anos ou menos, com gliomas de baixo grau e com doença residual (mais de \(5\%\) da lesão inicial ou \(1,5 cm^2\)) ou progressiva. Ocorreu revisão central das imagens e da patologia. Pacientes com tumores hipotalâmico-quiasmáticos foram incluídos com base nas imagens. A sobrevida livre de progressão em 5 anos foi de \(45\%\) para todo o grupo, sendo de \(39\%\) para o esquema carboplatina-vincristina e \(52\%\) para o esquema TPCV. Esta diferença não foi significante num teste de log-rank, mas mostrou-se significante num modelo de sobrevida com fração de cura (cure rate model), onde parte desta diferença deveu-se a pacientes com sobrevida prolongada (\(p<0,05\)) (nível 2). O ensaio encontrou dois preditores independentes da sobrevida livre de eventos: idade (menor risco entre 1 e 5 anos e doença residual (menor risco se \(<3 cm^2\)) (nível 2).

O resultado deste ensaio deve ser encarado com senso crítico. Apesar de ambos os regimes terapêuticos aparentemente terem conseguido adiar a progressão nos pacientes estudados, a comparação com o subgrupo com doença residual significante do CCG9891 indica que deve-se ter cautela na indicação de tratamentos adjuvantes. Pacientes com tumor residual maior que \(1,5 cm^2\) e menor que \(3,0 cm^2\) serão melhor seguidos com conduta expectante? Para pacientes com mais de 5 anos e mais de \(3,0 cm^2\) de tumor residual, deve-se indicar radioterapia precocemente? A quimioterapia tem papel restrito aos pacientes menores de 5 anos com progressão documentada e naqueles com gliomas hipotalâmico-quiasmáticos? A ausência de ensaios comparativos entre as abordagens terapêuticas e de ensaios com controles não tratados impede a resposta destas questões com certeza. A conclusão sobre a terapia dos gliomas de baixo grau é de que, hoje em dia, temos evidência de relativa boa qualidade documentando a história natural deste grupo de tumores, mas a ausência de adequados estudos controlados e randomizados ainda suscita dúvidas quanto à melhor conduta em cada situação. A partir dos dados que temos até o momento, um esquema racional de tratamento para gliomas de baixo grau pediátricos inclui a melhor ressecção cirúrgica possível (mantendo ao máximo a função), seguimento de perto de todos os pacientes com doença residual (independente da quantidade), aguardar a progressão para indicar terapia adjuvante (mesmo quando doença residual), evitar radioterapia em menores de 5 anos e portadores de NF-1 através do uso de quimioterapia (TPCV um pouco superior a carboplatina-vincristina) e tratar com radioterapia lesões progressivas após cirurgia e/ou quimioterapia. Infelizmente, mesmo com essa abordagem baseada em evidência, uma quantidade significativa de pacientes terá doença progressiva apesar da melhor terapia, mostrando que a idéia geral de que os gliomas de baixo grau pediátricos são “benignos” deve ser revista e que ainda é necessário definir subgrupos de risco.

%\subsection{O panorama molecular dos gliomas de baixo grau}

%\section{Tumores embrionários}


\bibliographystyle{unsrt}
\bibliography{bib}

\appendix

\chapter{Esquemas de quimioterapia}

No Centro Pediátrico do Câncer do Hospital Infantil Albert Sabin, utilizamos um total de 4 (quatro) protocolos principais de tratamento quimioterápico para tumores cerebrais em crianças e adolescentes, baseados na literatura que foi citada neste manual. Estes protocolos foram adaptados a partir dos racionais dos ensaios clínicos descritos, com modificações pertinentes à realidade e disponibilidade de recursos em nosso serviço hospitalar. Além disso, quaisquer conclusões oriundas dos resultados destes ensaios clínicos, bem como informações de outros trabalhos e de outras fontes, foram usadas para adaptar os esquemas de tratamento à luz da melhor evidência disponível no momento em que este manual foi escrito. Alguns dos ensaios clínicos utilizados como modelo para parametrizar nossos protocolos ainda estão em andamento. Neste caso, apenas a parte não randomizada, não experimental dos esquemas foi adaptada e utilizada, mas não os braços de tratamento experimental ou não comprovado por evidências científicas.

Pacientes com condições patológicas que não têm nenhum tratamento amplamente aceito, ou sobre as quais recaem controvérsias importantes quanto à terapêutica, não poderão ser tratados com os protocolos principais descritos, mas poderão entrar, mediante consentimento informado, em esquemas de tratamento \textit{off-label} (não padronizado). Estes esquemas incluem aqueles sobre os quais a evidência científica presente é inconclusiva ou preliminar. Tratamentos baseados em trabalhos observacionais, ensaios clínicos fase I ou II com número limitado de pacientes estão nesta condição. Os resultados favoráveis de ensaios fase II que tenham recrutado maior número de pacientes, mesmo que não randomizado, poderão servir de base para o tratamento de pacientes entre os protocolos principais, em vista da falta de evidências científicas de qualidade maior, como ficou exposto no texto deste manual.

Apesar de tratados conforme os racionais de ensaios clínicos conhecidos, os pacientes não estão sendo recrutados para pesquisa, e isso é deixado claro antes do início do tratamento. Quaisquer esquemas alternativos de tratamento aceitáveis do ponto de vista de chances de sucesso e risco de efeitos adversos são informados aos responsáveis pelos pacientes. Estes podem escolher livremente entre os protocolos principais ou tratamentos alternativos aceitáveis.

Nas páginas que se seguem, apresentamos as folhas de acompanhamento ambulatorial dos pacientes que estão em tratamento quimioterápico em nosso serviço hospitalar. Estas folhas são anexadas a cada prontuário do paciente e são preenchidas de acordo com o andamento do tratamento, anotando doses administradas, principais complicações, atrasos, modificações de doses, atualização de informações, entre outros dados. As versões aqui mostradas são as mais atuais quando da publicação deste manual.
\cleardoublepage
\chapter{Protocolos principais}
\cleardoublepage

\section{GLIOMA DE BAIXO GRAU -- Adaptado do ensaio COG-A9952}
\let\thefootnote\relax\footnotetext{Versão Janeiro/2017}
\textbf{Racional:} no estudo fase III do COG, a QT possibilitou adiar a RT em pacientes com gliomas de baixo grau recorrentes ou progressivos.\footnote{Ater \textit{et al}, 2012} Dois esquemas foram comparados: o TPCV, mais antigo, e carboplatina-vincristina. Embora o esquema TPCV tenha mostrado resultados algo superiores (não estatisticamente significantes), o segundo esquema ainda é o preferido pelo menor risco de efeitos a longo prazo.

\textbf{Elegível:} pacientes com menos de 12 anos ou portadores de NF-1, com astrocitoma de baixo grau (pilocítico, difuso, outros), oligodendroglioma, ganglioglioma, tumores mistos (oligoastrocitomas, outros), tumores de vias ópticas/hipotálamo (imagem típica, mesmo sem biópsia). Incluir tumores focais de tronco, excluir DIPG, ou tumores difusos de linha média H3K27M+. NÃO INICIAR ESTE PROTOCOLO EM CRIANÇAS GRAVEMENTE ENFERMAS.


\textbf{Alternativa:} a conduta expectante é uma opção, uma vez que, via de regra, o crescimento destes tumores é lento e sua progressão demora anos, ou mesmo décadas. Pacientes de maior risco, como aqueles com lesões de vias ópticas ou hipotálamo, síndrome diencefálica ou com lesões de crescimento rápido devem ser tratados sem grande demora. Se possível, uma nova ressecção cirúrgica deve ser avaliada. O tratamento com vimblastina semanal também pode ser usado como primeira linha. A principal alternativa adjuvante para pacientes com mais de 5 anos e sem NF-1 é a RT local. Pacientes com astrocitomas difusos têm maior risco de transformação maligna após RT. 

\def\entrywithlabel[#1]#2{\parbox{#1}{{\small #2:} \hrulefill}}
\def\entrywithlabelunder[#1]#2{\parbox{#1}{\hrulefill\\[-.75ex]\centerline {#2}}}
\def\entrywithlabelraised[#1]#2{\parbox{#1}{\smash{\raise-1ex\hbox{{\tiny #2}}}\hrulefill}}
\def\boxentry[#1]#2{{\setlength{\fboxsep}{-\fboxrule}\fbox{\parbox{#1}{\smash{\raise-6.5pt\hbox{~{\tiny #2}}}\vspace{2ex}\mbox{}}}}}
\def\boxpar[#1]#2#3{{\setlength{\fboxsep}{-\fboxrule}\fbox{\parbox[][#2][t]{#1}{\mbox{}\\[-.125\baselineskip]\mbox{}~#3}}}}

\vspace{5mm}
\entrywithlabel[.96\hsize]{\textbf{Nome}}\hfill \\

\entrywithlabel[.45\hsize]{\textbf{Peso}}\hfill  \entrywithlabel[.45\hsize]{\textbf{Estatura}}

\subsection{Indução: 10 semanas}

\begin{center}
\begin{longtable}{p{1cm}p{5cm}|p{5cm}|p{3cm}}
    \hline
    \multicolumn{1}{c|}{Exames:}&{Neut(\(>1,5\times10^3\)):}&{Plaq(\(>10^5\)):}&{TGO:}\\
    \hline
    \multicolumn{1}{c|}{\multirow{2}{*}{\textbf{D1}}}&{Carboplatina \(175\)mg/m\(^2\)}&{Administrado: (  ) Sim (  ) Não}&{Rubrica}\\
    \multicolumn{1}{c|}{}&{Vincristina \(1,5\) mg/m\(^2\)}&{Data:}&\\
    \hline\\
    \hline
    \multicolumn{1}{c|}{\multirow{2}{*}{\textbf{D8}}}&{Carboplatina \(175\)mg/m\(^2\)}&{Administrado: (  ) Sim (  ) Não}&{Rubrica}\\
    \multicolumn{1}{c|}{}&{Vincristina \(1,5\) mg/m\(^2\)}&{Data:}&\\
    \hline
    \\
    \hline
    \multicolumn{1}{c|}{\multirow{2}{*}{\textbf{D15}}}&{Carboplatina \(175\)mg/m\(^2\)}&{Administrado: (  ) Sim (  ) Não}&{Rubrica}\\
    \multicolumn{1}{c|}{}&{Vincristina \(1,5\) mg/m\(^2\)}&{Data:}&\\
    \hline
    \\
    \hline
    \multicolumn{1}{c|}{\multirow{2}{*}{\textbf{D22}}}&{Carboplatina \(175\)mg/m\(^2\)}&{Administrado: (  ) Sim (  ) Não}&{Rubrica}\\
    \multicolumn{1}{c|}{}&{Vincristina \(1,5\) mg/m\(^2\)}&{Data:}&\\
    \hline
    \multicolumn{1}{c|}{Exames:}&{Neut(\(>1,5\times10^3\)):}&{Plaq(\(>10^5\)):}&{TGO:}\\
    \hline
    \\
    \hline
    \multicolumn{1}{c|}{\multirow{2}{*}{\textbf{D29}}}&{Vincristina \(1,5\) mg/m\(^2\)}&{Administrado: (  ) Sim (  ) Não}&{Rubrica}\\
    \multicolumn{1}{c|}{}&&{Data:}&\\
    \hline
    \\
    \hline
    \multicolumn{1}{c|}{\multirow{2}{*}{\textbf{D36}}}&{Vincristina \(1,5\) mg/m\(^2\)}&{Administrado: (  ) Sim (  ) Não}&{Rubrica}\\
    \multicolumn{1}{c|}{}&&{Data:}&\\
    \hline
    \\
    \hline
    \multicolumn{1}{c|}{\multirow{2}{*}{\textbf{D43}}}&{Carboplatina \(175\)mg/m\(^2\)}&{Administrado: (  ) Sim (  ) Não}&{Rubrica}\\
    \multicolumn{1}{c|}{}&{Vincristina \(1,5\) mg/m\(^2\)}&{Data:}&\\
    \hline
    \multicolumn{1}{c|}{Exames:}&{Neut(\(>1,5\times10^3\)):}&{Plaq(\(>10^5\)):}&{TGO:}
    \\
    \hline
    \\
	\hline    
    \multicolumn{1}{c|}{\multirow{2}{*}{\textbf{D50}}}&{Carboplatina \(175\)mg/m\(^2\)}&{Administrado: (  ) Sim (  ) Não}&{Rubrica}\\
    \multicolumn{1}{c|}{}&{Vincristina \(1,5\) mg/m\(^2\)}&{Data:}&\\
    \hline
    \\
    \hline
    \multicolumn{1}{c|}{\multirow{2}{*}{\textbf{D57}}}&{Carboplatina \(175\)mg/m\(^2\)}&{Administrado: (  ) Sim (  ) Não}&{Rubrica}\\
    \multicolumn{1}{c|}{}&{Vincristina \(1,5\) mg/m\(^2\)}&{Data:}&\\
    \hline
    \\
    \hline
    \multicolumn{1}{c|}{\multirow{2}{*}{\textbf{D64}}}&{Carboplatina \(175\)mg/m\(^2\)}&{Administrado: (  ) Sim (  ) Não}&{Rubrica}\\
    \multicolumn{1}{c|}{}&{Vincristina \(1,5\) mg/m\(^2\)}&{Data:}&\\
    \hline
\end{longtable}
\textbf{Intervalo de 21 dias.}
\end{center}

\subsection{Manutenção: 48 semanas (08 blocos de 6 semanas)}

\begin{center}
\begin{longtable}{p{1cm}p{5cm}|p{5cm}|p{3cm}}
    \hline
    \multicolumn{1}{c|}{\multirow{2}{*}{\textbf{D85}}}&{Carboplatina \(175\)mg/m\(^2\)}&{Administrado: (  ) Sim (  ) Não}&{Rubrica}\\
    \multicolumn{1}{c|}{}&{Vincristina \(1,5\) mg/m\(^2\)}&{Data:}&\\
    \hline
    \\
    \hline
    \multicolumn{1}{c|}{\multirow{2}{*}{\textbf{D92}}}&{Carboplatina \(175\)mg/m\(^2\)}&{Administrado: (  ) Sim (  ) Não}&{Rubrica}\\
    \multicolumn{1}{c|}{}&{Vincristina \(1,5\) mg/m\(^2\)}&{Data:}&\\
    \hline
    \multicolumn{1}{c|}{Exames:}&{Neut(\(>1,5\times10^3\)):}&{Plaq(\(>10^5\)):}&{TGO:}
    \\
    \hline
    \\
    \hline
    \multicolumn{1}{c|}{\multirow{2}{*}{\textbf{D99}}}&{Carboplatina \(175\)mg/m\(^2\)}&{Administrado: (  ) Sim (  ) Não}&{Rubrica}\\
    \multicolumn{1}{c|}{}&{Vincristina \(1,5\) mg/m\(^2\)}&{Data:}&\\
    \hline
    \\
    \hline
    \multicolumn{1}{c|}{\multirow{2}{*}{\textbf{D106}}}&{Carboplatina \(175\)mg/m\(^2\)}&{Administrado: (  ) Sim (  ) Não}&{Rubrica}\\
	\multicolumn{1}{c|}{}&&{Data:}&\\
    \hline
\end{longtable}
\textbf{Intervalo de 21 dias.}
\end{center}

\pagebreak

\vspace{5mm}
\entrywithlabel[.96\hsize]{\textbf{Nome}}\hfill \\

\entrywithlabel[.45\hsize]{\textbf{Peso}}\hfill  \entrywithlabel[.45\hsize]{\textbf{Estatura}}

\begin{center}
\begin{longtable}{p{1cm}p{5cm}|p{5cm}|p{3cm}}
    \hline
    \multicolumn{1}{c|}{\multirow{2}{*}{\textbf{D127}}}&{Carboplatina \(175\)mg/m\(^2\)}&{Administrado: (  ) Sim (  ) Não}&{Rubrica}\\
    \multicolumn{1}{c|}{}&{Vincristina \(1,5\) mg/m\(^2\)}&{Data:}&\\
    \hline
    \\
    \hline
    \multicolumn{1}{c|}{\multirow{2}{*}{\textbf{D134}}}&{Carboplatina \(175\)mg/m\(^2\)}&{Administrado: (  ) Sim (  ) Não}&{Rubrica}\\
    \multicolumn{1}{c|}{}&{Vincristina \(1,5\) mg/m\(^2\)}&{Data:}&\\
    \hline
    \multicolumn{1}{c|}{Exames:}&{Neut(\(>1,5\times10^3\)):}&{Plaq(\(>10^5\)):}&{TGO:}
    \\
    \hline
    \\
    \hline
    \multicolumn{1}{c|}{\multirow{2}{*}{\textbf{D141}}}&{Carboplatina \(175\)mg/m\(^2\)}&{Administrado: (  ) Sim (  ) Não}&{Rubrica}\\
    \multicolumn{1}{c|}{}&{Vincristina \(1,5\) mg/m\(^2\)}&{Data:}&\\
    \hline
    \\
    \hline
    \multicolumn{1}{c|}{\multirow{2}{*}{\textbf{D148}}}&{Carboplatina \(175\)mg/m\(^2\)}&{Administrado: (  ) Sim (  ) Não}&{Rubrica}\\
	\multicolumn{1}{c|}{}&&{Data:}&\\
    \hline
\end{longtable}
\textbf{Intervalo de 21 dias.}
\end{center}

\begin{center}
\begin{longtable}{p{1cm}p{5cm}|p{5cm}|p{3cm}}
    \hline
    \multicolumn{1}{c|}{\multirow{2}{*}{\textbf{D169}}}&{Carboplatina \(175\)mg/m\(^2\)}&{Administrado: (  ) Sim (  ) Não}&{Rubrica}\\
    \multicolumn{1}{c|}{}&{Vincristina \(1,5\) mg/m\(^2\)}&{Data:}&\\
    \hline
    \\
    \hline
    \multicolumn{1}{c|}{\multirow{2}{*}{\textbf{D176}}}&{Carboplatina \(175\)mg/m\(^2\)}&{Administrado: (  ) Sim (  ) Não}&{Rubrica}\\
    \multicolumn{1}{c|}{}&{Vincristina \(1,5\) mg/m\(^2\)}&{Data:}&\\
    \hline
    \multicolumn{1}{c|}{Exames:}&{Neut(\(>1,5\times10^3\)):}&{Plaq(\(>10^5\)):}&{TGO:}
    \\
    \hline
    \\
    \hline
    \multicolumn{1}{c|}{\multirow{2}{*}{\textbf{D183}}}&{Carboplatina \(175\)mg/m\(^2\)}&{Administrado: (  ) Sim (  ) Não}&{Rubrica}\\
    \multicolumn{1}{c|}{}&{Vincristina \(1,5\) mg/m\(^2\)}&{Data:}&\\
    \hline
    \\
    \hline
    \multicolumn{1}{c|}{\multirow{2}{*}{\textbf{D190}}}&{Carboplatina \(175\)mg/m\(^2\)}&{Administrado: (  ) Sim (  ) Não}&{Rubrica}\\
	\multicolumn{1}{c|}{}&&{Data:}&\\
    \hline
\end{longtable}
\textbf{Intervalo de 21 dias.}
\end{center}

\begin{center}
\begin{longtable}{p{1cm}p{5cm}|p{5cm}|p{3cm}}
    \hline
    \multicolumn{1}{c|}{\multirow{2}{*}{\textbf{D211}}}&{Carboplatina \(175\)mg/m\(^2\)}&{Administrado: (  ) Sim (  ) Não}&{Rubrica}\\
    \multicolumn{1}{c|}{}&{Vincristina \(1,5\) mg/m\(^2\)}&{Data:}&\\
    \hline
    \\
    \hline
    \multicolumn{1}{c|}{\multirow{2}{*}{\textbf{D218}}}&{Carboplatina \(175\)mg/m\(^2\)}&{Administrado: (  ) Sim (  ) Não}&{Rubrica}\\
    \multicolumn{1}{c|}{}&{Vincristina \(1,5\) mg/m\(^2\)}&{Data:}&\\
    \hline
    \multicolumn{1}{c|}{Exames:}&{Neut(\(>1,5\times10^3\)):}&{Plaq(\(>10^5\)):}&{TGO:}
    \\
    \hline
    \\
    \hline
    \multicolumn{1}{c|}{\multirow{2}{*}{\textbf{D225}}}&{Carboplatina \(175\)mg/m\(^2\)}&{Administrado: (  ) Sim (  ) Não}&{Rubrica}\\
    \multicolumn{1}{c|}{}&{Vincristina \(1,5\) mg/m\(^2\)}&{Data:}&\\
    \hline
    \\
    \hline
    \multicolumn{1}{c|}{\multirow{2}{*}{\textbf{D232}}}&{Carboplatina \(175\)mg/m\(^2\)}&{Administrado: (  ) Sim (  ) Não}&{Rubrica}\\
	\multicolumn{1}{c|}{}&&{Data:}&\\
    \hline
\end{longtable}
\textbf{Intervalo de 21 dias.}
\end{center}

\begin{center}
\begin{longtable}{p{1cm}p{5cm}|p{5cm}|p{3cm}}
    \hline
    \multicolumn{1}{c|}{\multirow{2}{*}{\textbf{D253}}}&{Carboplatina \(175\)mg/m\(^2\)}&{Administrado: (  ) Sim (  ) Não}&{Rubrica}\\
    \multicolumn{1}{c|}{}&{Vincristina \(1,5\) mg/m\(^2\)}&{Data:}&\\
    \hline
    \\
    \hline
    \multicolumn{1}{c|}{\multirow{2}{*}{\textbf{D260}}}&{Carboplatina \(175\)mg/m\(^2\)}&{Administrado: (  ) Sim (  ) Não}&{Rubrica}\\
    \multicolumn{1}{c|}{}&{Vincristina \(1,5\) mg/m\(^2\)}&{Data:}&\\
    \hline
    {Exames:}&{Neut(\(>1,5\times10^3\)):}&{Plaq(\(>10^5\)):}&{TGO:}
    \\
    \hline
    \\
    \hline
    \multicolumn{1}{c|}{\multirow{2}{*}{\textbf{D267}}}&{Carboplatina \(175\)mg/m\(^2\)}&{Administrado: (  ) Sim (  ) Não}&{Rubrica}\\
    \multicolumn{1}{c|}{}&{Vincristina \(1,5\) mg/m\(^2\)}&{Data:}&\\
    \hline
    \\
    \hline
    \multicolumn{1}{c|}{\multirow{2}{*}{\textbf{D274}}}&{Carboplatina \(175\)mg/m\(^2\)}&{Administrado: (  ) Sim (  ) Não}&{Rubrica}\\
	\multicolumn{1}{c|}{}&&{Data:}&\\
    \hline
\end{longtable}
\textbf{Intervalo de 21 dias.}
\end{center}

\begin{center}
\begin{longtable}{p{1cm}p{5cm}|p{5cm}|p{3cm}}
    \hline
    \multicolumn{1}{c|}{\multirow{2}{*}{\textbf{D295}}}&{Carboplatina \(175\)mg/m\(^2\)}&{Administrado: (  ) Sim (  ) Não}&{Rubrica}\\
    \multicolumn{1}{c|}{}&{Vincristina \(1,5\) mg/m\(^2\)}&{Data:}&\\
    \hline
    \\
    \hline
    \multicolumn{1}{c|}{\multirow{2}{*}{\textbf{D302}}}&{Carboplatina \(175\)mg/m\(^2\)}&{Administrado: (  ) Sim (  ) Não}&{Rubrica}\\
    \multicolumn{1}{c|}{}&{Vincristina \(1,5\) mg/m\(^2\)}&{Data:}&\\
    \hline
    {Exames:}&{Neut(\(>1,5\times10^3\)):}&{Plaq(\(>10^5\)):}&{TGO:}
    \\
    \hline
    \\
    \hline
    \multicolumn{1}{c|}{\multirow{2}{*}{\textbf{D309}}}&{Carboplatina \(175\)mg/m\(^2\)}&{Administrado: (  ) Sim (  ) Não}&{Rubrica}\\
    \multicolumn{1}{c|}{}&{Vincristina \(1,5\) mg/m\(^2\)}&{Data:}&\\
    \hline
    \\
    \hline
    \multicolumn{1}{c|}{\multirow{2}{*}{\textbf{D316}}}&{Carboplatina \(175\)mg/m\(^2\)}&{Administrado: (  ) Sim (  ) Não}&{Rubrica}\\
	\multicolumn{1}{c|}{}&&{Data:}&\\
    \hline
\end{longtable}
\textbf{Intervalo de 21 dias.}
\end{center}

\pagebreak

\vspace{5mm}
\entrywithlabel[.96\hsize]{\textbf{Nome}}\hfill \\

\entrywithlabel[.45\hsize]{\textbf{Peso}}\hfill  \entrywithlabel[.45\hsize]{\textbf{Estatura}}

\begin{center}
\begin{longtable}{p{1cm}p{5cm}|p{5cm}|p{3cm}}
    \hline
    \multicolumn{1}{c|}{\multirow{2}{*}{\textbf{D337}}}&{Carboplatina \(175\)mg/m\(^2\)}&{Administrado: (  ) Sim (  ) Não}&{Rubrica}\\
    \multicolumn{1}{c|}{}&{Vincristina \(1,5\) mg/m\(^2\)}&{Data:}&\\
    \hline
    \\
    \hline
    \multicolumn{1}{c|}{\multirow{2}{*}{\textbf{D344}}}&{Carboplatina \(175\)mg/m\(^2\)}&{Administrado: (  ) Sim (  ) Não}&{Rubrica}\\
    \multicolumn{1}{c|}{}&{Vincristina \(1,5\) mg/m\(^2\)}&{Data:}&\\
    \hline
    {Exames:}&{Neut(\(>1,5\times10^3\)):}&{Plaq(\(>10^5\)):}&{TGO:}
    \\
    \hline
    \\
    \hline
    \multicolumn{1}{c|}{\multirow{2}{*}{\textbf{D351}}}&{Carboplatina \(175\)mg/m\(^2\)}&{Administrado: (  ) Sim (  ) Não}&{Rubrica}\\
    \multicolumn{1}{c|}{}&{Vincristina \(1,5\) mg/m\(^2\)}&{Data:}&\\
    \hline
    \\
    \hline
    \multicolumn{1}{c|}{\multirow{2}{*}{\textbf{D358}}}&{Carboplatina \(175\)mg/m\(^2\)}&{Administrado: (  ) Sim (  ) Não}&{Rubrica}\\
	\multicolumn{1}{c|}{}&&{Data:}&\\
    \hline
\end{longtable}
\textbf{Intervalo de 21 dias.}
\end{center}

\begin{center}
\begin{longtable}{p{1cm}p{5cm}|p{5cm}|p{3cm}}
    \hline
    \multicolumn{1}{c|}{\multirow{2}{*}{\textbf{D379}}}&{Carboplatina \(175\)mg/m\(^2\)}&{Administrado: (  ) Sim (  ) Não}&{Rubrica}\\
    \multicolumn{1}{c|}{}&{Vincristina \(1,5\) mg/m\(^2\)}&{Data:}&\\
    \hline
    \\
    \hline
    \multicolumn{1}{c|}{\multirow{2}{*}{\textbf{D386}}}&{Carboplatina \(175\)mg/m\(^2\)}&{Administrado: (  ) Sim (  ) Não}&{Rubrica}\\
    \multicolumn{1}{c|}{}&{Vincristina \(1,5\) mg/m\(^2\)}&{Data:}&\\
    \hline
    {Exames:}&{Neut(\(>1,5\times10^3\)):}&{Plaq(\(>10^5\)):}&{TGO:}
    \\
    \hline
    \\
    \hline
    \multicolumn{1}{c|}{\multirow{2}{*}{\textbf{D393}}}&{Carboplatina \(175\)mg/m\(^2\)}&{Administrado: (  ) Sim (  ) Não}&{Rubrica}\\
    \multicolumn{1}{c|}{}&{Vincristina \(1,5\) mg/m\(^2\)}&{Data:}&\\
    \hline
    \\
    \hline
    \multicolumn{1}{c|}{\multirow{2}{*}{\textbf{D400}}}&{Carboplatina \(175\)mg/m\(^2\)}&{Administrado: (  ) Sim (  ) Não}&{Rubrica}\\
	\multicolumn{1}{c|}{}&&{Data:}&\\
    \hline
\end{longtable}

\textbf{\textit{Final de Protocolo}}

\textbf{Solicitar imagem (RNM)}

\end{center}

\subsection{Modificações de dose:} 
Iniciar a manutenção se Neut > 1000/mm\(^3\) e Plaq > 100 mil/mm\(^3\). Adiar ciclo 1 semana se Neut < 500mm\(^3\) ou Plaq > 50 mil/mm\(^3\). Qualquer paciente com febre ou neutropenia e/ ou infecção localizada terão tratamento in- terrompido até que estas complicações sejam resolvidas. Para os pacientes com mais do que um atraso de 2 semanas de tratamento, associados com sepse, neutropenia ou uma contagem de plaquetas inferior a 20000, a próxima dose de carboplatina será diminuída em 50\%. Para aqueles pacientes que desenvolverem neurotoxicidade significativa relacionada à vincristina  (queda do pé, íleo), a administração de vincristina será suspensa até que haja evidência de melhora neurológica e a próxima dose de vincristina será reduzida em 50\%.

\textbf{ATENÇÃO:} o objetivo deste protocolo é ADIAR O USO DA RT até a criança atingir uma idade onde os efeitos adversos da radiação sejam reduzidos. A principal resposta deste protocolo é ESTABILIZAÇÃO DA DOENÇA. Logo, é inadequado iniciar este esquema de QT em crianças em regime de internação prolongada, dependentes de cuidados hospitalares, visando "melhorar" sua condição clínica. Igualmente, é inadequado iniciar este protocolo em crianças com risco de complicações graves, como naquelas que têm sequelas importantes e muito limitantes.

\cleardoublepage

\section{GLIOMA DE BAIXO GRAU: TRATAMENTO ALTERNATIVO DE PRIMEIRA LINHA, SEGUNDA LINHA NA CONTRA-INDICAÇÃO AO USO DA CARBOPLATINA OU RECORRÊNCIA APÓS TRATAMENTO}
{\let\thefootnote\relax\footnotetext{Versão Janeiro/2017}}
\textbf{Racional:} num estudo piloto de 2003, o grupo de Eric Bouffet mostrou a viabilidade e boa resposta do uso de vimblastina semanal em pacientes com reação à carboplatina. No estudo fase II do The Hospital for Sick Children, a vimblastina foi eficaz em induzir remissão parcial ou completa em 36\% de 51 pacientes com gliomas de baixo grau recorrentes ou progressivos, após esquemas prévios de quimioterapia e/ou radioterapia \footnote{Lafay-Cousin \textit{et al}, 2003; Bouffet \textit{et al}, 2012}. Um estudo fase II que incluiu 54 pacientes com gliomas de baixo grau progressivos usou a vimblastina como primeira linha, mostrando resultados aparentemente semelhantes a outros estudos (como o COG-A9952) \footnote{Lassaletta \textit{et al}, 2016}.

\textbf{Elegível:} astrocitoma de baixo grau (pilocítico, difuso, outros), oligodendroglioma, ganglioglioma, tumores mistos (oligoastrocitomas, outros), tumores de vias ópticas/hipotálamo (imagem típica, mesmo sem biópsia). Incluir tumores focais de tronco, excluir DIPG, ou tumores difusos de linha média H3K27M+. Pacientes com reação ou contraindicação ao uso de carboplatina; doença recorrente após prévio tratamento com quimioterapia e/ou radioterapia. Alternativa como primeira linha de tratamento. NÃO INICIAR ESTE PROTOCOLO EM CRIANÇAS GRAVEMENTE ENFERMAS.

\textbf{Alternativa:} a conduta expectante é uma opção, uma vez que, via de regra, o crescimento destes tumores é lento e sua progressão demora anos, ou mesmo décadas. Pacientes de maior risco, como aqueles com lesões de vias ópticas ou hipotálamo, síndrome diencefálica ou com lesões de crescimento rápido devem ser tratados sem grande demora. Se possível, uma nova ressecção cirúrgica deve ser avaliada. O protocolo baseado no estudo COG-A9952 (carboplatina e vincristina) pode ser usado como primeira linha. A principal alternativa adjuvante para pacientes com mais de 5 anos e sem NF-1 é a RT local. Pacientes com astrocitomas difusos têm maior risco de transformação maligna após RT. 

\vspace{5mm}
\entrywithlabel[.96\hsize]{\textbf{Nome}}\hfill \\

\entrywithlabel[.45\hsize]{\textbf{Peso}}\hfill  \entrywithlabel[.45\hsize]{\textbf{Estatura}}

\subsection{Quimioterapia adjuvante: 52 semanas ou 1 ano}
\begin{center}
\begin{longtable}{p{1cm}p{5cm}|p{5cm}|p{3cm}}
    \hline
    \multicolumn{1}{c|}{\multirow{2}{*}{\textbf{D1}}}&{Vimblastina \(6,0\)mg/m\(^2\)}&{Administrado: (  ) Sim (  ) Não}&{Rubrica}\\
    \multicolumn{1}{c|}{}&{EV em bolo (max 10mg)}&{Data:}&\\
    \hline
    {Exames:}&{Neut (\(>7,5\times10^2\)):}&{Plaq(\(>10^5\)):}&{TGO:}
    \\
    \hline
    \\
    \hline
    \multicolumn{1}{c|}{\multirow{2}{*}{\textbf{D8}}}&{Vimblastina \(6,0\)mg/m\(^2\)}&{Administrado: (  ) Sim (  ) Não}&{Rubrica}\\
    \multicolumn{1}{c|}{}&{EV em bolo (max 10mg)}&{Data:}&\\
    \hline
    \\
    \hline
    \multicolumn{1}{c|}{\multirow{2}{*}{\textbf{D15}}}&{Vimblastina \(6,0\)mg/m\(^2\)}&{Administrado: (  ) Sim (  ) Não}&{Rubrica}\\
    \multicolumn{1}{c|}{}&{EV em bolo (max 10mg)}&{Data:}&\\
    \hline
    \\
    \hline
    \multicolumn{1}{c|}{\multirow{2}{*}{\textbf{D22}}}&{Vimblastina \(6,0\)mg/m\(^2\)}&{Administrado: (  ) Sim (  ) Não}&{Rubrica}\\
    \multicolumn{1}{c|}{}&{EV em bolo (max 10mg)}&{Data:}&\\
    \hline
    \\
    \hline
    \multicolumn{1}{c|}{\multirow{2}{*}{\textbf{D29}}}&{Vimblastina \(6,0\)mg/m\(^2\)}&{Administrado: (  ) Sim (  ) Não}&{Rubrica}\\
    \multicolumn{1}{c|}{}&{EV em bolo (max 10mg)}&{Data:}&\\
    \hline
    {Exames:}&{Neut (\(>7,5\times10^2\)):}&{Plaq(\(>10^5\)):}&{TGO:}
    \\
    \hline\\
    \hline
    \multicolumn{1}{c|}{\multirow{2}{*}{\textbf{D36}}}&{Vimblastina \(6,0\)mg/m\(^2\)}&{Administrado: (  ) Sim (  ) Não}&{Rubrica}\\
    \multicolumn{1}{c|}{}&{EV em bolo (max 10mg)}&{Data:}&\\
    \hline
    \\
    \hline
    \multicolumn{1}{c|}{\multirow{2}{*}{\textbf{D43}}}&{Vimblastina \(6,0\)mg/m\(^2\)}&{Administrado: (  ) Sim (  ) Não}&{Rubrica}\\
    \multicolumn{1}{c|}{}&{EV em bolo (max 10mg)}&{Data:}&\\
    \hline
    \\
    \hline
    \multicolumn{1}{c|}{\multirow{2}{*}{\textbf{D50}}}&{Vimblastina \(6,0\)mg/m\(^2\)}&{Administrado: (  ) Sim (  ) Não}&{Rubrica}\\
    \multicolumn{1}{c|}{}&{EV em bolo (max 10mg)}&{Data:}&\\
    \hline
    \\
    \hline
    \multicolumn{1}{c|}{\multirow{2}{*}{\textbf{D57}}}&{Vimblastina \(6,0\)mg/m\(^2\)}&{Administrado: (  ) Sim (  ) Não}&{Rubrica}\\
    \multicolumn{1}{c|}{}&{EV em bolo (max 10mg)}&{Data:}&\\
    \hline
    {Exames:}&{Neut (\(>7,5\times10^2\)):}&{Plaq(\(>10^5\)):}&{TGO:}
    \\
    \hline
    \\
    \hline
    \multicolumn{1}{c|}{\multirow{2}{*}{\textbf{D64}}}&{Vimblastina \(6,0\)mg/m\(^2\)}&{Administrado: (  ) Sim (  ) Não}&{Rubrica}\\
    \multicolumn{1}{c|}{}&{EV em bolo (max 10mg)}&{Data:}&\\
    \hline
    \\
    \hline
    \multicolumn{1}{c|}{\multirow{2}{*}{\textbf{D71}}}&{Vimblastina \(6,0\)mg/m\(^2\)}&{Administrado: (  ) Sim (  ) Não}&{Rubrica}\\
    \multicolumn{1}{c|}{}&{EV em bolo (max 10mg)}&{Data:}&\\
    \hline
    \\
    \hline
    \multicolumn{1}{c|}{\multirow{2}{*}{\textbf{D78}}}&{Vimblastina \(6,0\)mg/m\(^2\)}&{Administrado: (  ) Sim (  ) Não}&{Rubrica}\\
    \multicolumn{1}{c|}{}&{EV em bolo (max 10mg)}&{Data:}&\\
    \hline
    \\
    \hline
    \multicolumn{1}{c|}{\multirow{2}{*}{\textbf{D85}}}&{Vimblastina \(6,0\)mg/m\(^2\)}&{Administrado: (  ) Sim (  ) Não}&{Rubrica}\\
    \multicolumn{1}{c|}{}&{EV em bolo (max 10mg)}&{Data:}&\\
    \hline
    {Exames:}&{Neut (\(>7,5\times10^2\)):}&{Plaq(\(>10^5\)):}&{TGO:}
    \\
    \hline
    \\
    \hline
    \multicolumn{1}{c|}{\multirow{2}{*}{\textbf{D92}}}&{Vimblastina \(6,0\)mg/m\(^2\)}&{Administrado: (  ) Sim (  ) Não}&{Rubrica}\\
    \multicolumn{1}{c|}{}&{EV em bolo (max 10mg)}&{Data:}&\\
    \hline
    \\
    \hline
    \multicolumn{1}{c|}{\multirow{2}{*}{\textbf{D99}}}&{Vimblastina \(6,0\)mg/m\(^2\)}&{Administrado: (  ) Sim (  ) Não}&{Rubrica}\\
    \multicolumn{1}{c|}{}&{EV em bolo (max 10mg)}&{Data:}&\\
    \hline
\end{longtable}
\end{center}

\vspace{5mm}
\entrywithlabel[.96\hsize]{\textbf{Nome}}\hfill \\

\entrywithlabel[.45\hsize]{\textbf{Peso}}\hfill  \entrywithlabel[.45\hsize]{\textbf{Estatura}}

\begin{center}
\begin{longtable}{p{1cm}p{5cm}|p{5cm}|p{3cm}}
    \hline
    \multicolumn{1}{c|}{\multirow{2}{*}{\textbf{D106}}}&{Vimblastina \(6,0\)mg/m\(^2\)}&{Administrado: (  ) Sim (  ) Não}&{Rubrica}\\
    \multicolumn{1}{c|}{}&{EV em bolo (max 10mg)}&{Data:}&\\
    \hline\\
    
    \hline
    \multicolumn{1}{c|}{\multirow{2}{*}{\textbf{D113}}}&{Vimblastina \(6,0\)mg/m\(^2\)}&{Administrado: (  ) Sim (  ) Não}&{Rubrica}\\
    \multicolumn{1}{c|}{}&{EV em bolo (max 10mg)}&{Data:}&\\
    \hline
    {Exames:}&{Neut (\(>7,5\times10^2\)):}&{Plaq(\(>10^5\)):}&{TGO:}
    \\
    \hline
    \\
    \hline
    \multicolumn{1}{c|}{\multirow{2}{*}{\textbf{D120}}}&{Vimblastina \(6,0\)mg/m\(^2\)}&{Administrado: (  ) Sim (  ) Não}&{Rubrica}\\
    \multicolumn{1}{c|}{}&{EV em bolo (max 10mg)}&{Data:}&\\
    \hline\\
    \hline
    \multicolumn{1}{c|}{\multirow{2}{*}{\textbf{D127}}}&{Vimblastina \(6,0\)mg/m\(^2\)}&{Administrado: (  ) Sim (  ) Não}&{Rubrica}\\
    \multicolumn{1}{c|}{}&{EV em bolo (max 10mg)}&{Data:}&\\
    \hline
    \\
    \hline
    \multicolumn{1}{c|}{\multirow{2}{*}{\textbf{D134}}}&{Vimblastina \(6,0\)mg/m\(^2\)}&{Administrado: (  ) Sim (  ) Não}&{Rubrica}\\
    \multicolumn{1}{c|}{}&{EV em bolo (max 10mg)}&{Data:}&\\
    \hline
    \\
    \hline
    \multicolumn{1}{c|}{\multirow{2}{*}{\textbf{D141}}}&{Vimblastina \(6,0\)mg/m\(^2\)}&{Administrado: (  ) Sim (  ) Não}&{Rubrica}\\
    \multicolumn{1}{c|}{}&{EV em bolo (max 10mg)}&{Data:}&\\
    \hline
    {Exames:}&{Neut (\(>7,5\times10^2\)):}&{Plaq(\(>10^5\)):}&{TGO:}
    \\
    \hline
    \\
    \hline
    \multicolumn{1}{c|}{\multirow{2}{*}{\textbf{D148}}}&{Vimblastina \(6,0\)mg/m\(^2\)}&{Administrado: (  ) Sim (  ) Não}&{Rubrica}\\
    \multicolumn{1}{c|}{}&{EV em bolo (max 10mg)}&{Data:}&\\
    \hline
    \\
    \hline
    \multicolumn{1}{c|}{\multirow{2}{*}{\textbf{D155}}}&{Vimblastina \(6,0\)mg/m\(^2\)}&{Administrado: (  ) Sim (  ) Não}&{Rubrica}\\
    \multicolumn{1}{c|}{}&{EV em bolo (max 10mg)}&{Data:}&\\
    \hline
    \\
    \hline
    \multicolumn{1}{c|}{\multirow{2}{*}{\textbf{D162}}}&{Vimblastina \(6,0\)mg/m\(^2\)}&{Administrado: (  ) Sim (  ) Não}&{Rubrica}\\
    \multicolumn{1}{c|}{}&{EV em bolo (max 10mg)}&{Data:}&\\
    \hline
    \\
    \hline
    \multicolumn{1}{c|}{\multirow{2}{*}{\textbf{D169}}}&{Vimblastina \(6,0\)mg/m\(^2\)}&{Administrado: (  ) Sim (  ) Não}&{Rubrica}\\
    \multicolumn{1}{c|}{}&{EV em bolo (max 10mg)}&{Data:}&\\
    \hline
    {Exames:}&{Neut (\(>7,5\times10^2\)):}&{Plaq(\(>10^5\)):}&{TGO:}
    \\
    \hline
    \\
    \hline
    \multicolumn{1}{c|}{\multirow{2}{*}{\textbf{D176}}}&{Vimblastina \(6,0\)mg/m\(^2\)}&{Administrado: (  ) Sim (  ) Não}&{Rubrica}\\
    \multicolumn{1}{c|}{}&{EV em bolo (max 10mg)}&{Data:}&\\
    \hline
    \\
    \hline
    \multicolumn{1}{c|}{\multirow{2}{*}{\textbf{D183}}}&{Vimblastina \(6,0\)mg/m\(^2\)}&{Administrado: (  ) Sim (  ) Não}&{Rubrica}\\
    \multicolumn{1}{c|}{}&{EV em bolo (max 10mg)}&{Data:}&\\
    \hline
    \\
    \hline
    \multicolumn{1}{c|}{\multirow{2}{*}{\textbf{D190}}}&{Vimblastina \(6,0\)mg/m\(^2\)}&{Administrado: (  ) Sim (  ) Não}&{Rubrica}\\
    \multicolumn{1}{c|}{}&{EV em bolo (max 10mg)}&{Data:}&\\
    \hline
    \\
    \hline
    \multicolumn{1}{c|}{\multirow{2}{*}{\textbf{D197}}}&{Vimblastina \(6,0\)mg/m\(^2\)}&{Administrado: (  ) Sim (  ) Não}&{Rubrica}\\
    \multicolumn{1}{c|}{}&{EV em bolo (max 10mg)}&{Data:}&\\
    \hline
    {Exames:}&{Neut (\(>7,5\times10^2\)):}&{Plaq(\(>10^5\)):}&{TGO:}
    \\
    \hline
    \\
    \hline
    \multicolumn{1}{c|}{\multirow{2}{*}{\textbf{D204}}}&{Vimblastina \(6,0\)mg/m\(^2\)}&{Administrado: (  ) Sim (  ) Não}&{Rubrica}\\
    \multicolumn{1}{c|}{}&{EV em bolo (max 10mg)}&{Data:}&\\
    \hline
    \\
    \hline
    \multicolumn{1}{c|}{\multirow{2}{*}{\textbf{D211}}}&{Vimblastina \(6,0\)mg/m\(^2\)}&{Administrado: (  ) Sim (  ) Não}&{Rubrica}\\
    \multicolumn{1}{c|}{}&{EV em bolo (max 10mg)}&{Data:}&\\
    \hline
    \\
    \hline
    \multicolumn{1}{c|}{\multirow{2}{*}{\textbf{D218}}}&{Vimblastina \(6,0\)mg/m\(^2\)}&{Administrado: (  ) Sim (  ) Não}&{Rubrica}\\
    \multicolumn{1}{c|}{}&{EV em bolo (max 10mg)}&{Data:}&\\
    \hline
    \\
    \hline
    \multicolumn{1}{c|}{\multirow{2}{*}{\textbf{225}}}&{Vimblastina \(6,0\)mg/m\(^2\)}&{Administrado: (  ) Sim (  ) Não}&{Rubrica}\\
    \multicolumn{1}{c|}{}&{EV em bolo (max 10mg)}&{Data:}&\\
    \hline
    {Exames:}&{Neut (\(>7,5\times10^2\)):}&{Plaq(\(>10^5\)):}&{TGO:}
    \\
    \hline
    \\
    \hline
    \multicolumn{1}{c|}{\multirow{2}{*}{\textbf{D232}}}&{Vimblastina \(6,0\)mg/m\(^2\)}&{Administrado: (  ) Sim (  ) Não}&{Rubrica}\\
    \multicolumn{1}{c|}{}&{EV em bolo (max 10mg)}&{Data:}&\\
    \hline
    \\
    \hline
    \multicolumn{1}{c|}{\multirow{2}{*}{\textbf{D239}}}&{Vimblastina \(6,0\)mg/m\(^2\)}&{Administrado: (  ) Sim (  ) Não}&{Rubrica}\\
    \multicolumn{1}{c|}{}&{EV em bolo (max 10mg)}&{Data:}&\\
    \hline
    \\
    \hline
    \multicolumn{1}{c|}{\multirow{2}{*}{\textbf{D246}}}&{Vimblastina \(6,0\)mg/m\(^2\)}&{Administrado: (  ) Sim (  ) Não}&{Rubrica}\\
    \multicolumn{1}{c|}{}&{EV em bolo (max 10mg)}&{Data:}&\\
    \hline
    \\
    \hline
    \multicolumn{1}{c|}{\multirow{2}{*}{\textbf{D253}}}&{Vimblastina \(6,0\)mg/m\(^2\)}&{Administrado: (  ) Sim (  ) Não}&{Rubrica}\\
    \multicolumn{1}{c|}{}&{EV em bolo (max 10mg)}&{Data:}&\\
    \hline
    {Exames:}&{Neut (\(>7,5\times10^2\)):}&{Plaq(\(>10^5\)):}&{TGO:}
    \\
    \hline
    \\
    \hline
    \multicolumn{1}{c|}{\multirow{2}{*}{\textbf{D260}}}&{Vimblastina \(6,0\)mg/m\(^2\)}&{Administrado: (  ) Sim (  ) Não}&{Rubrica}\\
    \multicolumn{1}{c|}{}&{EV em bolo (max 10mg)}&{Data:}&\\
    \hline
\end{longtable}
\end{center}

\vspace{5mm}
\entrywithlabel[.96\hsize]{\textbf{Nome}}\hfill \\

\entrywithlabel[.45\hsize]{\textbf{Peso}}\hfill  \entrywithlabel[.45\hsize]{\textbf{Estatura}}

\begin{center}
\begin{longtable}{p{1cm}p{5cm}|p{5cm}|p{3cm}}
    \hline
    \multicolumn{1}{c|}{\multirow{2}{*}{\textbf{D267}}}&{Vimblastina \(6,0\)mg/m\(^2\)}&{Administrado: (  ) Sim (  ) Não}&{Rubrica}\\
    \multicolumn{1}{c|}{}&{EV em bolo (max 10mg)}&{Data:}&\\
    \hline
    \\
    \hline
    \multicolumn{1}{c|}{\multirow{2}{*}{\textbf{D274}}}&{Vimblastina \(6,0\)mg/m\(^2\)}&{Administrado: (  ) Sim (  ) Não}&{Rubrica}\\
    \multicolumn{1}{c|}{}&{EV em bolo (max 10mg)}&{Data:}&\\
    \hline
    \\
    \hline
    \multicolumn{1}{c|}{\multirow{2}{*}{\textbf{D281}}}&{Vimblastina \(6,0\)mg/m\(^2\)}&{Administrado: (  ) Sim (  ) Não}&{Rubrica}\\
    \multicolumn{1}{c|}{}&{EV em bolo (max 10mg)}&{Data:}&\\
    \hline
    {Exames:}&{Neut (\(>7,5\times10^2\)):}&{Plaq(\(>10^5\)):}&{TGO:}
    \\
    \hline
    \\
    \hline
    \multicolumn{1}{c|}{\multirow{2}{*}{\textbf{D288}}}&{Vimblastina \(6,0\)mg/m\(^2\)}&{Administrado: (  ) Sim (  ) Não}&{Rubrica}\\
    \multicolumn{1}{c|}{}&{EV em bolo (max 10mg)}&{Data:}&\\
    \hline
    \\
    \hline
    \multicolumn{1}{c|}{\multirow{2}{*}{\textbf{D295}}}&{Vimblastina \(6,0\)mg/m\(^2\)}&{Administrado: (  ) Sim (  ) Não}&{Rubrica}\\
    \multicolumn{1}{c|}{}&{EV em bolo (max 10mg)}&{Data:}&\\
    \hline
    \\
    \hline
    \multicolumn{1}{c|}{\multirow{2}{*}{\textbf{D302}}}&{Vimblastina \(6,0\)mg/m\(^2\)}&{Administrado: (  ) Sim (  ) Não}&{Rubrica}\\
    \multicolumn{1}{c|}{}&{EV em bolo (max 10mg)}&{Data:}&\\
    \hline\\
    \hline
    \multicolumn{1}{c|}{\multirow{2}{*}{\textbf{D309}}}&{Vimblastina \(6,0\)mg/m\(^2\)}&{Administrado: (  ) Sim (  ) Não}&{Rubrica}\\
    \multicolumn{1}{c|}{}&{EV em bolo (max 10mg)}&{Data:}&\\
    \hline
    {Exames:}&{Neut (\(>7,5\times10^2\)):}&{Plaq(\(>10^5\)):}&{TGO:}
    \\
    \hline
    \\
    \hline
    \multicolumn{1}{c|}{\multirow{2}{*}{\textbf{D316}}}&{Vimblastina \(6,0\)mg/m\(^2\)}&{Administrado: (  ) Sim (  ) Não}&{Rubrica}\\
    \multicolumn{1}{c|}{}&{EV em bolo (max 10mg)}&{Data:}&\\
    \hline
    \\
    \hline
    \multicolumn{1}{c|}{\multirow{2}{*}{\textbf{D323}}}&{Vimblastina \(6,0\)mg/m\(^2\)}&{Administrado: (  ) Sim (  ) Não}&{Rubrica}\\
    \multicolumn{1}{c|}{}&{EV em bolo (max 10mg)}&{Data:}&\\
    \hline
    \\
    \hline
    \multicolumn{1}{c|}{\multirow{2}{*}{\textbf{D330}}}&{Vimblastina \(6,0\)mg/m\(^2\)}&{Administrado: (  ) Sim (  ) Não}&{Rubrica}\\
    \multicolumn{1}{c|}{}&{EV em bolo (max 10mg)}&{Data:}&\\
    \hline
    \\
    \hline
    \multicolumn{1}{c|}{\multirow{2}{*}{\textbf{D337}}}&{Vimblastina \(6,0\)mg/m\(^2\)}&{Administrado: (  ) Sim (  ) Não}&{Rubrica}\\
    \multicolumn{1}{c|}{}&{EV em bolo (max 10mg)}&{Data:}&\\
    \hline
    {Exames:}&{Neut (\(>7,5\times10^2\)):}&{Plaq(\(>10^5\)):}&{TGO:}
    \\
    \hline
    \\
    \hline
    \multicolumn{1}{c|}{\multirow{2}{*}{\textbf{344}}}&{Vimblastina \(6,0\)mg/m\(^2\)}&{Administrado: (  ) Sim (  ) Não}&{Rubrica}\\
    \multicolumn{1}{c|}{}&{EV em bolo (max 10mg)}&{Data:}&\\
    \hline
    \\
    \hline
    \multicolumn{1}{c|}{\multirow{2}{*}{\textbf{D351}}}&{Vimblastina \(6,0\)mg/m\(^2\)}&{Administrado: (  ) Sim (  ) Não}&{Rubrica}\\
    \multicolumn{1}{c|}{}&{EV em bolo (max 10mg)}&{Data:}&\\
    \hline
    \\
    \hline
    \multicolumn{1}{c|}{\multirow{2}{*}{\textbf{D358}}}&{Vimblastina \(6,0\)mg/m\(^2\)}&{Administrado: (  ) Sim (  ) Não}&{Rubrica}\\
    \multicolumn{1}{c|}{}&{EV em bolo (max 10mg)}&{Data:}&\\
    \hline
    \\
    \hline
    \multicolumn{1}{c|}{\multirow{2}{*}{\textbf{D365}}}&{Vimblastina \(6,0\)mg/m\(^2\)}&{Administrado: (  ) Sim (  ) Não}&{Rubrica}\\
    \multicolumn{1}{c|}{}&{EV em bolo (max 10mg)}&{Data:}&\\
    \hline
    {Exames:}&{Neut (\(>7,5\times10^2\)):}&{Plaq(\(>10^5\)):}&{TGO:}
    \\
   \hline
\end{longtable}
\textbf{\textit{Final de Protocolo}}
 
\textbf{Solicitar imagem (RNM)}
\end{center}
\subsection{Modificações de dose:} 
Se 750 $\geq$ Neut $\geq$ 500/mm\(^3\), reduzir dose para 5mg/m\textsuperscript{2}. Se Neut < 500mm\(^3\), interromper até subir para 750 ou mais. Se toxicidade hematológica recorrente, reduzir dose para 4mg/m\textsuperscript{2}.

\textbf{ATENÇÃO:} o objetivo deste protocolo é ADIAR O USO DA RT (se não tiver sido feita) até a criança atingir uma idade onde os efeitos adversos da radiação sejam reduzidos, ou controlar doença recidivada após a RT. A principal resposta deste protocolo é ESTABILIZAÇÃO DA DOENÇA. Logo, é inadequado iniciar este esquema de QT em crianças em regime de internação prolongada, dependentes de cuidados hospitalares, visando "melhorar" sua condição clínica. Igualmente, é inadequado iniciar este protocolo em crianças com risco de complicações graves, como naquelas que têm sequelas importantes e muito limitantes.\\

\cleardoublepage
\section{MEDULOBLASTOMA - RISCO PADRÃO -- Adaptado dos ensaios CCG-9961 e ACNS0331}
{\let\thefootnote\relax\footnotetext{Versão Janeiro/2015}}
\textbf{Racional:} no estudo do COG, a QT possibilitou a redução da dose da RT para o neuro-eixo para 2340 cGY, com \textit{boost} para o sítio tumoral completando 54 Gy de dose total\footnote{Packer \textit{et al}, 2006}. O COG está testando agora uma nova redução da RT, com o ensaio fase III ACNS0331. O COG fez modificações na manutenção do protocolo. Utilizamos o esquema de QT segundo o braço controle do ensaio ACNS0331, derivado do CCG-9961.

\textbf{Elegível:} apenas meduloblastoma (fossa posterior), com menos de 1,5cm\textsuperscript{2} de tumor residual (RNM de controle até 21 dias pós-op, preferido 72h após); sem metástases (RNM de neuro-eixo e PL/MO); excluir tumores com anaplasia ou positivos para N-MYC/C-MYC. Tratamento precisa iniciar até 31 dias após cirurgia. Excluir pacientes com menos de 3 anos. NÃO INICIAR ESTE PROTOCOLO EM CRIANÇAS GRAVEMENTE ENFERMAS.

\textbf{Alternativa:} a principal alternativa é a RT para neuro-eixo sem redução de dose (36 Gy) com boost para a fossa posterior de 18-20 Gy, completando 54-56 Gy de dose total. Essa estratégia, na ausência de QT adjuvante, é capaz de evitar recidivas em pacientes de risco padrão.

\vspace{5mm}
\entrywithlabel[.96\hsize]{\textbf{Nome}}\hfill \\

\entrywithlabel[.45\hsize]{\textbf{Peso}}\hfill  \entrywithlabel[.45\hsize]{\textbf{Estatura}}

\subsection{Radioquimioterapia: 7 semanas (43 dias)}

\begin{center}
\begin{longtable}{p{1cm}p{2cm}|p{2cm}|p{1cm}|p{4cm}|p{3cm}}
	\hline
	\multicolumn{6}{c}{\textbf{SEMANA 1}}\\
\hline
    \multicolumn{1}{c|}{\multirow{2}{*}{\textbf{Dia}}}&\multicolumn{2}{c|}{Dose RT}&\multicolumn{1}{c|}{\multirow{2}{*}{Data}}&\multicolumn{1}{c|}{\multirow{2}{*}{Quimioterapia}}&\multicolumn{1}{c}{\multirow{2}{*}{Rubrica}} \\
    \cline{2-3}
    \multicolumn{1}{c|}{\multirow{1}{*}{}}&{Neuro-eixo}&{Fossa poster}&& \\
	\hline
	\multicolumn{1}{c|}{\multirow{1}{*}{\textbf{D1}}}&\multicolumn{1}{c|}{\(1,8\) Gy}&&&{Vincristina \(1,5\) mg/m\(^2\)}&\\
    \multicolumn{1}{c|}{\multirow{1}{*}{\textbf{D2}}}&\multicolumn{1}{c|}{\(1,8\) Gy}&&&{}&\\
    \multicolumn{1}{c|}{\multirow{1}{*}{\textbf{D3}}}&\multicolumn{1}{c|}{\(1,8\) Gy}&&&{}&\\
    \multicolumn{1}{c|}{\multirow{1}{*}{\textbf{D4}}}&\multicolumn{1}{c|}{\(1,8\) Gy}&&&{}&\\
    \multicolumn{1}{c|}{\multirow{1}{*}{\textbf{D5}}}&\multicolumn{1}{c|}{\(1,8\) Gy}&&&{}&\\
    \hline
    \multicolumn{1}{c|}{\multirow{2}{*}{\textbf{Exames}}}&\multicolumn{2}{l|}{Neut (\(>7,5\times10^2\)):}&\multicolumn{2}{l|}{Plaq (\(>7,5\times10^4\)):}&\\
    \cline{2-6}
    \multicolumn{1}{c|}{\multirow{2}{*}{{}}}&\multicolumn{2}{l|}{BT(<1,9mg/dl):}&\multicolumn{2}{l|}{BD(\(<1,5\)mg/dl):}&
    \\
    \hline
\end{longtable}
\pagebreak
\begin{longtable}{p{1cm}p{2cm}|p{2cm}|p{1cm}|p{4cm}|p{3cm}}
	\hline
	\multicolumn{6}{c}{\textbf{SEMANA 2}}\\
\hline
    \multicolumn{1}{c|}{\multirow{2}{*}{\textbf{Dia}}}&\multicolumn{2}{c|}{Dose RT}&\multicolumn{1}{c|}{\multirow{2}{*}{Data}}&\multicolumn{1}{c|}{\multirow{2}{*}{Quimioterapia}}&\multicolumn{1}{c}{\multirow{2}{*}{Rubrica}} \\
    \cline{2-3}
    \multicolumn{1}{c|}{\multirow{1}{*}{}}&{Neuro-eixo}&{Fossa poster}&& \\
	\hline
	\multicolumn{1}{c|}{\multirow{1}{*}{\textbf{D8}}}&\multicolumn{1}{c|}{\(1,8\) Gy}&&&{Vincristina \(1,5\) mg/m\(^2\)}&\\
    \multicolumn{1}{c|}{\multirow{1}{*}{\textbf{D9}}}&\multicolumn{1}{c|}{\(1,8\) Gy}&&&{}&\\
    \multicolumn{1}{c|}{\multirow{1}{*}{\textbf{D10}}}&\multicolumn{1}{c|}{\(1,8\) Gy}&&&{}&\\
    \hline
    \multicolumn{1}{c|}{\multirow{1}{*}{\textbf{D11}}}&\multicolumn{1}{c|}{\(1,8\) Gy}&&&{}&\\
    \multicolumn{1}{c|}{\multirow{1}{*}{\textbf{D12}}}&\multicolumn{1}{c|}{\(1,8\) Gy}&&&{}&\\
    \hline
    \multicolumn{1}{c|}{\multirow{2}{*}{\textbf{Exames}}}&\multicolumn{2}{l|}{Neut (\(>7,5\times10^2\)):}&\multicolumn{2}{l|}{Plaq (\(>7,5\times10^4\)):}&\\
    \cline{2-6}
    \multicolumn{1}{c|}{\multirow{2}{*}{{}}}&\multicolumn{2}{l|}{BT(<1,9mg/dl):}&\multicolumn{2}{l|}{BD(\(<1,5\)mg/dl):}&
    \\
    \hline
\end{longtable}

\begin{longtable}{p{1cm}p{2cm}|p{2cm}|p{1cm}|p{4cm}|p{3cm}}
	\hline
	\multicolumn{6}{c}{\textbf{SEMANA 3}}\\
\hline
    \multicolumn{1}{c|}{\multirow{2}{*}{\textbf{Dia}}}&\multicolumn{2}{c|}{Dose RT}&\multicolumn{1}{c|}{\multirow{2}{*}{Data}}&\multicolumn{1}{c|}{\multirow{2}{*}{Quimioterapia}}&\multicolumn{1}{c}{\multirow{2}{*}{Rubrica}} \\
    \cline{2-3}
    \multicolumn{1}{c|}{\multirow{1}{*}{}}&{Neuro-eixo}&{Fossa poster}&& \\
	\hline
	\multicolumn{1}{c|}{\multirow{1}{*}{\textbf{D15}}}&\multicolumn{1}{c|}{\(1,8\) Gy}&&&{Vincristina \(1,5\) mg/m\(^2\)}&\\
    \multicolumn{1}{c|}{\multirow{1}{*}{\textbf{D16}}}&\multicolumn{1}{c|}{\(1,8\) Gy}&&&{}&\\
    \multicolumn{1}{c|}{\multirow{1}{*}{\textbf{D17}}}&\multicolumn{1}{c|}{\(1,8\) Gy}&&&{}&\\
    \multicolumn{1}{c|}{\multirow{1}{*}{\textbf{D18}}}&\multicolumn{1}{c|}{\(1,8\) Gy}&&&{}&\\
    \multicolumn{1}{c|}{\multirow{1}{*}{\textbf{D19}}}&\multicolumn{1}{c|}{\(1,8\) Gy}&&&{}&\\
    \hline
    \multicolumn{1}{c|}{\multirow{2}{*}{\textbf{Exames}}}&\multicolumn{2}{l|}{Neut (\(>7,5\times10^2\)):}&\multicolumn{2}{l|}{Plaq (\(>7,5\times10^4\)):}&\\
    \cline{2-6}
    \multicolumn{1}{c|}{\multirow{2}{*}{{}}}&\multicolumn{2}{l|}{BT(<1,9mg/dl):}&\multicolumn{2}{l|}{BD(\(<1,5\)mg/dl):}&
    \\
    \hline
\end{longtable}
\begin{longtable}{p{1cm}p{2cm}|p{2cm}|p{1cm}|p{4cm}|p{3cm}}
	\hline
	\multicolumn{6}{c}{\textbf{SEMANA 4}}\\
\hline
    \multicolumn{1}{c|}{\multirow{2}{*}{\textbf{Dia}}}&\multicolumn{2}{c|}{Dose RT}&\multicolumn{1}{c|}{\multirow{2}{*}{Data}}&\multicolumn{1}{c|}{\multirow{2}{*}{Quimioterapia}}&\multicolumn{1}{c}{\multirow{2}{*}{Rubrica}} \\
    \cline{2-3}
    \multicolumn{1}{c|}{\multirow{1}{*}{}}&{Neuro-eixo}&{Fossa poster}&& \\
	\hline
	\multicolumn{1}{c|}{\multirow{1}{*}{\textbf{D22}}}&\multicolumn{1}{c|}{\(1,8\) Gy}&&&{Vincristina \(1,5\) mg/m\(^2\)}&\\
    \multicolumn{1}{c|}{\multirow{1}{*}{\textbf{D23}}}&\multicolumn{1}{c|}{\(1,8\) Gy}&&&{}&\\
    \multicolumn{1}{c|}{\multirow{1}{*}{\textbf{D24}}}&\multicolumn{1}{c|}{\(1,8\) Gy}&&&{}&\\
    \multicolumn{1}{c|}{\multirow{1}{*}{\textbf{D25}}}&\multicolumn{1}{c|}{\(1,8\) Gy}&&&{}&\\
    \multicolumn{1}{c|}{\multirow{1}{*}{\textbf{D26}}}&\multicolumn{1}{c|}{\(1,8\) Gy}&&&{}&\\
    \hline
    \multicolumn{1}{c|}{\multirow{2}{*}{\textbf{Exames}}}&\multicolumn{2}{l|}{Neut (\(>7,5\times10^2\)):}&\multicolumn{2}{l|}{Plaq (\(>7,5\times10^4\)):}&\\
    \cline{2-6}
    \multicolumn{1}{c|}{\multirow{2}{*}{{}}}&\multicolumn{2}{l|}{BT(<1,9mg/dl):}&\multicolumn{2}{l|}{BD(\(<1,5\)mg/dl):}&
    \\
    \hline
\end{longtable}
\end{center}
\pagebreak

\entrywithlabel[.96\hsize]{\textbf{Nome}}\hfill \\

\entrywithlabel[.45\hsize]{\textbf{Peso}}\hfill  \entrywithlabel[.45\hsize]{\textbf{Estatura}}
\begin{center}

\begin{longtable}{p{1cm}p{2cm}|p{2cm}|p{1cm}|p{4cm}|p{3cm}}
	\hline
	\multicolumn{6}{c}{\textbf{SEMANA 5}}\\
\hline
    \multicolumn{1}{c|}{\multirow{2}{*}{\textbf{Dia}}}&\multicolumn{2}{c|}{Dose RT}&\multicolumn{1}{c|}{\multirow{2}{*}{Data}}&\multicolumn{1}{c|}{\multirow{2}{*}{Quimioterapia}}&\multicolumn{1}{c}{\multirow{2}{*}{Rubrica}} \\
    \cline{2-3}
    \multicolumn{1}{c|}{\multirow{1}{*}{}}&{Neuro-eixo}&{Fossa poster}&& \\
	\hline
	\multicolumn{1}{c|}{\multirow{1}{*}{\textbf{D29}}}&\multicolumn{1}{c|}{\(1,8\) Gy}&&&{Vincristina \(1,5\) mg/m\(^2\)}&\\
    \multicolumn{1}{c|}{\multirow{1}{*}{\textbf{D30}}}&\multicolumn{1}{c|}{\(1,8\) Gy}&&&{}&\\
    \multicolumn{1}{c|}{\multirow{1}{*}{\textbf{D31}}}&\multicolumn{1}{c|}{\(1,8\) Gy}&&&{}&\\
    \multicolumn{1}{c|}{\multirow{1}{*}{\textbf{D32}}}&\multicolumn{1}{c|}{\(1,8\) Gy}&&&{}&\\
    \multicolumn{1}{c|}{\multirow{1}{*}{\textbf{D33}}}&\multicolumn{1}{c|}{\(1,8\) Gy}&&&{}&\\
    \hline
    \multicolumn{1}{c|}{\multirow{2}{*}{\textbf{Exames}}}&\multicolumn{2}{l|}{Neut (\(>7,5\times10^2\)):}&\multicolumn{2}{l|}{Plaq (\(>7,5\times10^4\)):}&\\
    \cline{2-6}
    \multicolumn{1}{c|}{\multirow{2}{*}{{}}}&\multicolumn{2}{l|}{BT(<1,9mg/dl):}&\multicolumn{2}{l|}{BD(\(<1,5\)mg/dl):}&
    \\
    \hline
\end{longtable}

\begin{longtable}{p{1cm}p{2cm}|p{2cm}|p{1cm}|p{4cm}|p{3cm}}
	\hline
	\multicolumn{6}{c}{\textbf{SEMANA 6}}\\
\hline
    \multicolumn{1}{c|}{\multirow{2}{*}{\textbf{Dia}}}&\multicolumn{2}{c|}{Dose RT}&\multicolumn{1}{c|}{\multirow{2}{*}{Data}}&\multicolumn{1}{c|}{\multirow{2}{*}{Quimioterapia}}&\multicolumn{1}{c}{\multirow{2}{*}{Rubrica}} \\
    \cline{2-3}
    \multicolumn{1}{c|}{\multirow{1}{*}{}}&{Neuro-eixo}&{Fossa poster}&& \\
	\hline
	\multicolumn{1}{c|}{\multirow{1}{*}{\textbf{D36}}}&\multicolumn{1}{c|}{\(1,8\) Gy}&&&{Vincristina \(1,5\) mg/m\(^2\)}&\\
    \multicolumn{1}{c|}{\multirow{1}{*}{\textbf{D37}}}&\multicolumn{1}{c|}{\(1,8\) Gy}&&&{}&\\
    \multicolumn{1}{c|}{\multirow{1}{*}{\textbf{D38}}}&\multicolumn{1}{c|}{\(1,8\) Gy}&&&{}&\\
    \multicolumn{1}{c|}{\multirow{1}{*}{\textbf{D39}}}&\multicolumn{1}{c|}{\(1,8\) Gy}&&&{}&\\
    \multicolumn{1}{c|}{\multirow{1}{*}{\textbf{D40}}}&\multicolumn{1}{c|}{\(1,8\) Gy}&&&{}&\\
    \hline
    \multicolumn{1}{c|}{\multirow{2}{*}{\textbf{Exames}}}&\multicolumn{2}{l|}{Neut (\(>7,5\times10^2\)):}&\multicolumn{2}{l|}{Plaq (\(>7,5\times10^4\)):}&\\
    \cline{2-6}
    \multicolumn{1}{c|}{\multirow{2}{*}{{}}}&\multicolumn{2}{l|}{BT(<1,9mg/dl):}&\multicolumn{2}{l|}{BD(\(<1,5\)mg/dl):}&
    \\
    \hline
\end{longtable}
\begin{longtable}{p{1cm}p{2cm}|p{2cm}|p{1cm}|p{4cm}|p{3cm}}
	\hline
	\multicolumn{6}{c}{\textbf{SEMANA 7}}\\
\hline
    \multicolumn{1}{c|}{\multirow{2}{*}{\textbf{Dia}}}&\multicolumn{2}{c|}{Dose RT}&\multicolumn{1}{c|}{\multirow{2}{*}{Data}}&\multicolumn{1}{c|}{\multirow{2}{*}{Quimioterapia}}&\multicolumn{1}{c}{\multirow{2}{*}{Rubrica}} \\
    \cline{2-3}
    \multicolumn{1}{c|}{\multirow{1}{*}{}}&{Neuro-eixo}&{Fossa poster}&& \\
	\hline
	\multicolumn{1}{c|}{\multirow{1}{*}{\textbf{D43}}}&\multicolumn{1}{c|}{\(1,8\) Gy}&&&{Vincristina \(1,5\) mg/m\(^2\)}&\\
    \hline
    \multicolumn{1}{c|}{\multirow{2}{*}{\textbf{Exames}}}&\multicolumn{2}{l|}{Neut (\(>7,5\times10^2\)):}&\multicolumn{2}{l|}{Plaq (\(>7,5\times10^4\)):}&\\
    \cline{2-6}
    \multicolumn{1}{c|}{\multirow{2}{*}{{}}}&\multicolumn{2}{l|}{BT(<1,9mg/dl):}&\multicolumn{2}{l|}{BD(\(<1,5\)mg/dl):}&
    \\
    \hline
\end{longtable}
\textbf{Intervalo de 28 dias}\\
\end{center}
\textbf{Máximo de 8 doses de VCR, máximo de 20 dias recebendo RT cranioespinhal, máximo de 51 dias de RT no total}
\pagebreak
\subsection{Manutenção: 04 ciclos A e 04 ciclos B}

\begin{center}
\begin{longtable}{p{1cm}p{5.5cm}|p{1cm}|p{3.5cm}|p{2.5cm}}
	\hline
	\multicolumn{5}{c}{\textbf{CICLO A}}\\
\hline
    \multicolumn{1}{c|}{\multirow{1}{*}{\textbf{Dia}}}&{Dose}&{Data}&{Administrado}&{Rubrica} \\
    \hline
    \multicolumn{1}{c|}{\multirow{1}{*}{\textbf{D71}}}&{Cisplatina \(75\) mg/m\(^2\) EV em 6h}&&{(  ) Sim (  ) Não}&\\
    \multicolumn{1}{c|}{\multirow{1}{*}{\textbf{D72}}}&{Vincristina \(1,5\) mg/m\(^2\), max \(2\) mg}&&{(  ) Sim (  ) Não}&\\
    \multicolumn{1}{c|}{\multirow{1}{*}{\textbf{}}}&&&&\\
    \hline
    \multicolumn{1}{c|}{\multirow{2}{*}{\textbf{Exames}}}&\multicolumn{2}{l|}{Neut(\(>10^3\)):}&{Plaq(\(>10^5\)):}&\\
    \cline{2-5}
    \multicolumn{1}{c|}{\multirow{2}{*}{{}}}&\multicolumn{2}{l|}{ClearCreat:}&{}&{}\\
    \hline
    \\
    \hline
    \multicolumn{1}{c|}{\multirow{1}{*}{\textbf{D78}}}&{Vincristina \(1,5\) mg/m\(^2\), max \(2\) mg}&&{(  ) Sim (  ) Não}&\\
    \hline
    \\
    \hline
    \multicolumn{1}{c|}{\multirow{1}{*}{\textbf{D85}}}&{Vincristina \(1,5\) mg/m\(^2\), max \(2\) mg}&&{(  ) Sim (  ) Não}&\\
    \hline
    \end{longtable}
    \textbf{Intervalo de 28 dias}
    \\
\begin{longtable}{p{1cm}p{5.5cm}|p{1cm}|p{3.5cm}|p{2.5cm}}
    \hline
	\multicolumn{5}{c}{\textbf{CICLO B}}\\
	\hline
    \multicolumn{1}{c|}{\multirow{1}{*}{\textbf{Dia}}}&{Dose}&{Data}&{Administrado}&{Rubrica} \\
    \hline
    \multicolumn{1}{c|}{\multirow{1}{*}{\textbf{D113}}}&{Ciclofosfamida \(1,0\) g/m\(^2\) EV em 6h}&&{(  ) Sim (  ) Não}&\\
    \hline
    \multicolumn{1}{c|}{\multirow{1}{*}{\textbf{D114}}}&{Ciclofosfamida \(1,0\) g/m\(^2\) EV em 6h}&&{(  ) Sim (  ) Não}&\\
    \multicolumn{1}{c|}{\multirow{1}{*}{\textbf{}}}&{Vincristina \(1,5\) mg/m\(^2\), max \(2\) mg}&&{(  ) Sim (  ) Não}&\\
    \hline
    \multicolumn{1}{c|}{\multirow{2}{*}{\textbf{Exames}}}&\multicolumn{2}{l|}{Neut(\(>10^3\)):}&{Plaq(\(>10^5\)):}&\\
    \cline{2-5}
    \multicolumn{1}{c|}{\multirow{2}{*}{{}}}&\multicolumn{2}{l|}{ClearCreat:}&{}&{}\\
    \hline\\
    \hline
    \multicolumn{1}{c|}{\multirow{1}{*}{\textbf{D120}}}&{Vincristina \(1,5\) mg/m\(^2\), max \(2\) mg}&&{(  ) Sim (  ) Não}&\\
    \hline    
\end{longtable}
\textbf{Intervalo de 21 dias}
\begin{longtable}{p{1cm}p{5.5cm}|p{1cm}|p{3.5cm}|p{2.5cm}}
	\hline
	\multicolumn{5}{c}{\textbf{CICLO A}}\\
\hline
    \multicolumn{1}{c|}{\multirow{1}{*}{\textbf{Dia}}}&{Dose}&{Data}&{Administrado}&{Rubrica} \\
    \hline
    \multicolumn{1}{c|}{\multirow{1}{*}{\textbf{D141}}}&{Cisplatina \(75\) mg/m\(^2\) EV em 6h}&&{(  ) Sim (  ) Não}&\\
    \multicolumn{1}{c|}{\multirow{1}{*}{\textbf{D142}}}&{Vincristina \(1,5\) mg/m\(^2\), max \(2\) mg}&&{(  ) Sim (  ) Não}&\\
    \multicolumn{1}{c|}{\multirow{1}{*}{\textbf{}}}&&&&\\
    \hline
    \multicolumn{1}{c|}{\multirow{2}{*}{\textbf{Exames}}}&\multicolumn{2}{l|}{Neut(\(>10^3\)):}&{Plaq(\(>10^5\)):}&\\
    \cline{2-5}
    \multicolumn{1}{c|}{\multirow{2}{*}{{}}}&\multicolumn{2}{l|}{ClearCreat:}&{}&{}\\
    \hline
    \\
    \hline
    \multicolumn{1}{c|}{\multirow{1}{*}{\textbf{D148}}}&{Vincristina \(1,5\) mg/m\(^2\), max \(2\) mg}&&{(  ) Sim (  ) Não}&\\
    \hline
    \\
    \hline
    \multicolumn{1}{c|}{\multirow{1}{*}{\textbf{D155}}}&{Vincristina \(1,5\) mg/m\(^2\), max \(2\) mg}&&{(  ) Sim (  ) Não}&\\
    \hline
    \end{longtable}
    \textbf{Intervalo de 28 dias}
    
\end{center}
\pagebreak

\vspace{5mm}
\entrywithlabel[.96\hsize]{\textbf{Nome}}\hfill \\

\entrywithlabel[.45\hsize]{\textbf{Peso}}\hfill  \entrywithlabel[.45\hsize]{\textbf{Estatura}}

\begin{center}
\begin{longtable}{p{1cm}p{5.5cm}|p{1cm}|p{3.5cm}|p{2.5cm}}
    \hline
	\multicolumn{5}{c}{\textbf{CICLO B}}\\
	\hline
    \multicolumn{1}{c|}{\multirow{1}{*}{\textbf{Dia}}}&{Dose}&{Data}&{Administrado}&{Rubrica} \\
    \hline
    \multicolumn{1}{c|}{\multirow{1}{*}{\textbf{D183}}}&{Ciclofosfamida \(1,0\) g/m\(^2\) EV em 6h}&&{(  ) Sim (  ) Não}&\\
    \multicolumn{1}{c|}{\multirow{1}{*}{\textbf{D184}}}&{Ciclofosfamida \(1,0\) g/m\(^2\) EV em 6h}&&{(  ) Sim (  ) Não}&\\
    \multicolumn{1}{c|}{\multirow{1}{*}{\textbf{}}}&{Vincristina \(1,5\) mg/m\(^2\), max \(2\) mg}&&{(  ) Sim (  ) Não}&\\
    \hline
    \multicolumn{1}{c|}{\multirow{2}{*}{\textbf{Exames}}}&\multicolumn{2}{l|}{Neut(\(>10^3\)):}&{Plaq(\(>10^5\)):}&\\
    \cline{2-5}
    \multicolumn{1}{c|}{\multirow{2}{*}{{}}}&\multicolumn{2}{l|}{ClearCreat:}&{}&{}\\
    \hline\\
    \hline
    \multicolumn{1}{c|}{\multirow{1}{*}{\textbf{D190}}}&{Vincristina \(1,5\) mg/m\(^2\), max \(2\) mg}&&{(  ) Sim (  ) Não}&\\
    \hline    
\end{longtable}
\textbf{Intervalo de 21 dias}
\begin{longtable}{p{1cm}p{5.5cm}|p{1cm}|p{3.5cm}|p{2.5cm}}
	\hline
	\multicolumn{5}{c}{\textbf{CICLO A}}\\
\hline
    \multicolumn{1}{c|}{\multirow{1}{*}{\textbf{Dia}}}&{Dose}&{Data}&{Administrado}&{Rubrica} \\
    \hline
    \multicolumn{1}{c|}{\multirow{1}{*}{\textbf{D211}}}&{Cisplatina \(75\) mg/m\(^2\) EV em 6h}&&{(  ) Sim (  ) Não}&\\
    \multicolumn{1}{c|}{\multirow{1}{*}{\textbf{D212}}}&{Vincristina \(1,5\) mg/m\(^2\), max \(2\) mg}&&{(  ) Sim (  ) Não}&\\
    \multicolumn{1}{c|}{\multirow{1}{*}{\textbf{}}}&&&&\\
    \hline
    \multicolumn{1}{c|}{\multirow{2}{*}{\textbf{Exames}}}&\multicolumn{2}{l|}{Neut(\(>10^3\)):}&{Plaq(\(>10^5\)):}&\\
    \cline{2-5}
    \multicolumn{1}{c|}{\multirow{2}{*}{{}}}&\multicolumn{2}{l|}{ClearCreat:}&{}&{}\\
    \hline\\
    \hline
    \multicolumn{1}{c|}{\multirow{1}{*}{\textbf{D218}}}&{Vincristina \(1,5\) mg/m\(^2\), max \(2\) mg}&&{(  ) Sim (  ) Não}&\\
    \hline\\
    \hline
    \multicolumn{1}{c|}{\multirow{1}{*}{\textbf{D225}}}&{Vincristina \(1,5\) mg/m\(^2\), max \(2\) mg}&&{(  ) Sim (  ) Não}&\\
    \hline
    \end{longtable}
    \textbf{Intervalo de 28 dias}
    \\
\begin{longtable}{p{1cm}p{5.5cm}|p{1cm}|p{3.5cm}|p{2.5cm}}
    \hline
	\multicolumn{5}{c}{\textbf{CICLO B}}\\
	\hline
    \multicolumn{1}{c|}{\multirow{1}{*}{\textbf{Dia}}}&{Dose}&{Data}&{Administrado}&{Rubrica} \\
    \hline
    \multicolumn{1}{c|}{\multirow{1}{*}{\textbf{D253}}}&{Ciclofosfamida \(1,0\) g/m\(^2\) EV em 6h}&&{(  ) Sim (  ) Não}&\\
    \multicolumn{1}{c|}{\multirow{1}{*}{\textbf{D254}}}&{Ciclofosfamida \(1,0\) g/m\(^2\) EV em 6h}&&{(  ) Sim (  ) Não}&\\
    \multicolumn{1}{c|}{\multirow{1}{*}{\textbf{}}}&{Vincristina \(1,5\) mg/m\(^2\), max \(2\) mg}&&{(  ) Sim (  ) Não}&\\
    \hline
    \multicolumn{1}{c|}{\multirow{2}{*}{\textbf{Exames}}}&\multicolumn{2}{l|}{Neut(\(>10^3\)):}&{Plaq(\(>10^5\)):}&\\
    \cline{2-5}
    \multicolumn{1}{c|}{\multirow{2}{*}{{}}}&\multicolumn{2}{l|}{ClearCreat:}&{}&{}\\
    \hline
    \\
    \hline
    \multicolumn{1}{c|}{\multirow{1}{*}{\textbf{D260}}}&{Vincristina \(1,5\) mg/m\(^2\), max \(2\) mg}&&{(  ) Sim (  ) Não}&\\
    \hline    
\end{longtable}
\textbf{Intervalo de 21 dias}
\pagebreak
\begin{longtable}{p{1cm}p{5.5cm}|p{1cm}|p{3.5cm}|p{2.5cm}}
	\hline
	\multicolumn{5}{c}{\textbf{CICLO A}}\\
\hline
    \multicolumn{1}{c|}{\multirow{1}{*}{\textbf{Dia}}}&{Dose}&{Data}&{Administrado}&{Rubrica} \\
    \hline
    \multicolumn{1}{c|}{\multirow{1}{*}{\textbf{D281}}}&{Cisplatina \(75\) mg/m\(^2\) EV em 6h}&&{(  ) Sim (  ) Não}&\\
    \multicolumn{1}{c|}{\multirow{1}{*}{\textbf{D282}}}&{Vincristina \(1,5\) mg/m\(^2\), max \(2\) mg}&&{(  ) Sim (  ) Não}&\\
%    \multicolumn{1}{c|}{\multirow{1}{*}{\textbf{}}}&&&&\\
    \hline
    \multicolumn{1}{c|}{\multirow{2}{*}{\textbf{Exames}}}&\multicolumn{2}{l|}{Neut(\(>10^3\)):}&{Plaq(\(>10^5\)):}&\\
    \cline{2-5}
    \multicolumn{1}{c|}{\multirow{2}{*}{{}}}&\multicolumn{2}{l|}{ClearCreat:}&{}&{}\\
    \hline
    \\
    \hline
    \multicolumn{1}{c|}{\multirow{1}{*}{\textbf{D288}}}&{Vincristina \(1,5\) mg/m\(^2\), max \(2\) mg}&&{(  ) Sim (  ) Não}&\\
    \hline
    \\
    \hline
    \multicolumn{1}{c|}{\multirow{1}{*}{\textbf{D295}}}&{Vincristina \(1,5\) mg/m\(^2\), max \(2\) mg}&&{(  ) Sim (  ) Não}&\\
    \hline
    \end{longtable}
    \textbf{Intervalo de 28 dias}
\begin{longtable}{p{1cm}p{5.5cm}|p{1cm}|p{3.5cm}|p{2.5cm}}
    \hline
	\multicolumn{5}{c}{\textbf{CICLO B}}\\
	\hline
    \multicolumn{1}{c|}{\multirow{1}{*}{\textbf{Dia}}}&{Dose}&{Data}&{Administrado}&{Rubrica} \\
    \hline
    \multicolumn{1}{c|}{\multirow{1}{*}{\textbf{D323}}}&{Ciclofosfamida \(1,0\) g/m\(^2\) EV em 6h}&&{(  ) Sim (  ) Não}&\\
    \multicolumn{1}{c|}{\multirow{1}{*}{\textbf{D324}}}&{Ciclofosfamida \(1,0\) g/m\(^2\) EV em 6h}&&{(  ) Sim (  ) Não}&\\
    \multicolumn{1}{c|}{\multirow{1}{*}{\textbf{}}}&{Vincristina \(1,5\) mg/m\(^2\), max \(2\) mg}&&{(  ) Sim (  ) Não}&\\
    \hline
    \multicolumn{1}{c|}{\multirow{2}{*}{\textbf{Exames}}}&\multicolumn{2}{l|}{Neut (\(>10^3\)):}&{Plaq (\(>10^5\)):}&\\
    \cline{2-5}
    \multicolumn{1}{c|}{\multirow{2}{*}{{}}}&\multicolumn{2}{l|}{ClearCreat:}&{}&{}\\
    \hline
    \\
    \hline
    \multicolumn{1}{c|}{\multirow{1}{*}{\textbf{D330}}}&{Vincristina \(1,5\) mg/m\(^2\), max \(2\) mg}&&{(  ) Sim (  ) Não}&\\
    \hline    
\end{longtable}
\textbf{FIM DE PROTOCOLO}

\end{center}

\subsection{Modificações de dose:} 
Se tiver que adiar a CTX por neutropenia, reduzir em 50\% a dose, mesmo após recuperação.\\
Toxicidade grau 3-4 pela VCR, suspender dose seguinte. Reiniciar com dose normal. Recorrência: reduzir dose.
Se ocorrer redução de 20dB ou mais em frequências auditivas baixas (500-2000Hz), reduzir CDDP em 50\%. Se ocorrer redução de 30dB na faixa de 4000-8000 Hz), reduzir CDDP em 50\%. Ototoxicidade grau IV: interromper CDDP até nível de lesão retornar ao grau II.

\textbf{Avaliação:} imagem a cada 3 ciclos (3 meses), se progressão, interromper protocolo.\\

\textbf{ATENÇÃO:} o objetivo deste protocolo é REDUZIR A DOSE DA RT PARA O NEURO-EIXO, visando reduzir os efeitos adversos da radiação, sem aumentar a taxa de recidiva. Logo, é inadequado iniciar este esquema de QT em crianças com risco de complicações graves, como naquelas que têm sequelas importantes e muito limitantes.\\
\cleardoublepage
\section{TUMORES MALIGNOS DO SNC EM MENORES DE 3 ANOS -- Adaptado do ensaio CCG 9921}
{\let\thefootnote\relax\footnotetext{Versão Janeiro/2015}}
\small
\textbf{Racional:} no estudo do COG, a QT foi capaz de adiar e até tornar desnecessária a RT. Essa tem sido a principal estratégia de tratamento na maioria dos ensaios clínicos em crianças com esse perfil\footnote{Geyer, 2005}. Pacientes com sPNET e ATRT têm prognóstico bem inferior que os outros.

\textbf{Elegível:} gliomas de alto grau, ependimoma, tumores embrionários, tumores de células germinativas. Independente se metástase. Estadiamento: citologia LCR e imagem do neuro-eixo (RNM) para ependimomas e tumores embrionários (meduloblastoma, PNET, ATRT, pineoblastoma, outros); marcadores para TCG. NÃO INICIAR ESTE PROTOCOLO EM CRIANÇAS GRAVEMENTE ENFERMAS.

\textbf{Alternativa:} não existe tratamento padrão para crianças menores de 3 anos com tumores cerebrais malignos. Os pacientes com ressecção incompleta têm um prognóstico insatisfatório e sobrevida livre de progressão prolongada reduzida.

\vspace{5mm}
\entrywithlabel[.96\hsize]{\textbf{Nome}}\hfill \\

\entrywithlabel[.45\hsize]{\textbf{Peso}}\hfill  \entrywithlabel[.45\hsize]{\textbf{Estatura}}

\subsection{Indução: 5 ciclos (VCEC)}
\renewcommand{\arraystretch}{1.5}
\begin{center}
\begin{longtable}{p{1cm}c|p{5cm}|p{1.5cm}p{1.5cm}|c|c}
	\hline
	\multicolumn{7}{c}{Ciclo 1} \\
	\hline
	\multicolumn{1}{c|}{\multirow{1}{*}{\textbf{Dia}}}&{Data}&{}&\multicolumn{1}{c|}{Leuco}&\multicolumn{1}{c|}{Plaq}&{Administrado}&{Rubrica} \\
    \hline
    \multicolumn{1}{c|}{\multirow{3}{*}{\textbf{1}}}&&{Vincristina \(0,05\) mg/kg}&\multicolumn{1}{c|}{\(>10^3/mm^3\)}&\multicolumn{1}{c|}{\(>10^5/mm^3\)}&{(  ) Sim (  ) Não}&\\
    \cline{4-5}
    \multicolumn{1}{c|}{}&&{Etoposido \(1,5\) mg/kg/dia}&\multicolumn{1}{c|}{}&&{(  ) Sim (  ) Não}&\\
    \cline{4-5}
    \multicolumn{1}{c|}{}&\multirow{1}{*}{}&{Cisplatina \(3,5mg/kg\)}&&&{(  ) Sim (  ) Não}&\\
    \hline
    \multicolumn{1}{c|}{\multirow{3}{*}{\textbf{2}}}&&{Ciclofosfamida \(55\) mg/kg/dia}&{}&&{(  ) Sim (  ) Não}&\\
    \multicolumn{1}{c|}{}&&{MESNA \(55\) mg/kg/dia \(\times 0,1 \:e\: 5\)h}&&&{(  ) Sim (  ) Não}&\\
    \multicolumn{1}{c|}{}&&{Etoposido \(1,5\) mg/kg/dia}&&&{(  ) Sim (  ) Não}&\\
    \hline
    \multicolumn{1}{c|}{\multirow{3}{*}{\textbf{3}}}&&{Ciclofosfamida \(55\) mg/kg/dia}&{}&&{(  ) Sim (  ) Não}&\\
    \multicolumn{1}{c|}{}&&{MESNA \(55\) mg/kg/dia \(\times 0,1 \:e\: 5\)h}&&&{(  ) Sim (  ) Não}&\\
    \multicolumn{1}{c|}{}&\multirow{1}{*}{}&{Etoposido \(1,5\) mg/kg/dia}&{}&&{(  ) Sim (  ) Não}&\\
    \hline
    \multicolumn{1}{c|}{\textbf{4-13}}&&{G-CSF \(5 \mu\)g/kg/dia }&&&{(  ) Sim (  ) Não}&\\
    \hline
\end{longtable}
\begin{longtable}{p{1cm}p{0.8cm}|p{5cm}|p{1.7cm}p{2cm}|c|c}
	\hline
	\multicolumn{1}{c|}{\multirow{1}{*}{\textbf{Dia}}}&{Data}&{}&{}&&{Administrado}&{Rubrica} \\
    \hline
    \multicolumn{1}{c|}{\textbf{8}}&&{Vincristina \(1,5\) mg/m\(^2\)/dia}&\multicolumn{1}{c}{}&&{(  ) Sim (  ) Não}&\\
    \hline
    \multicolumn{1}{c|}{\textbf{15}}&&{Vincristina \(1,5\) mg/m\(^2\)/dia}&\multicolumn{1}{c}{}&&{(  ) Sim (  ) Não}&\\
    \hline
\end{longtable}

\begin{longtable}{p{1cm}c|p{5cm}|p{1.5cm}p{1.5cm}|c|c}
	\hline
	\multicolumn{7}{c}{Ciclo 2} \\
	\hline
	\multicolumn{1}{c|}{\multirow{1}{*}{\textbf{Dia}}}&{Data}&{}&\multicolumn{1}{c|}{Leuco}&\multicolumn{1}{c|}{Plaq}&{Administrado}&{Rubrica} \\
    \hline
    \multicolumn{1}{c|}{\multirow{3}{*}{\textbf{22}}}&&{Vincristina \(0,05\) mg/kg}&\multicolumn{1}{c|}{\(>10^3/mm^3\)}&\multicolumn{1}{c|}{\(>10^5/mm^3\)}&{(  ) Sim (  ) Não}&\\
    \cline{4-5}
    \multicolumn{1}{c|}{}&&{Etoposido \(1,5\) mg/kg/dia}&\multicolumn{1}{c|}{}&&{(  ) Sim (  ) Não}&\\
    \cline{4-5}
    \multicolumn{1}{c|}{}&\multirow{1}{*}{}&{Cisplatina \(3,5mg/kg\)}&&&{(  ) Sim (  ) Não}&\\
    \hline
    \multicolumn{1}{c|}{\multirow{3}{*}{\textbf{23}}}&&{Ciclofosfamida \(55\) mg/kg/dia}&{}&&{(  ) Sim (  ) Não}&\\
    \multicolumn{1}{c|}{}&&{MESNA \(55\) mg/kg/dia \(\times 0,1 \:e\: 5\)h}&&&{(  ) Sim (  ) Não}&\\
    \multicolumn{1}{c|}{}&&{Etoposido \(1,5\) mg/kg/dia}&&&{(  ) Sim (  ) Não}&\\
    \hline
    \multicolumn{1}{c|}{\multirow{3}{*}{\textbf{24}}}&&{Ciclofosfamida \(55\) mg/kg/dia}&{}&&{(  ) Sim (  ) Não}&\\
    \multicolumn{1}{c|}{}&&{MESNA \(55\) mg/kg/dia \(\times 0,1 \:e\: 5\)h}&&&{(  ) Sim (  ) Não}&\\
    \multicolumn{1}{c|}{}&\multirow{1}{*}{}&{Etoposido \(1,5\) mg/kg/dia}&{}&&{(  ) Sim (  ) Não}&\\
    \hline
    \multicolumn{1}{c|}{\textbf{25-34}}&&{G-CSF \(5 \mu\)g/kg/dia }&&&{(  ) Sim (  ) Não}&\\
    \hline
\end{longtable}
\begin{longtable}{p{1cm}p{0.9cm}|p{5cm}|p{1.7cm}p{2cm}|c|c}
	\hline
	\multicolumn{1}{c|}{\multirow{1}{*}{\textbf{Dia}}}&{Data}&{}&{}&&{Administrado}&{Rubrica} \\
    \hline
    \multicolumn{1}{c|}{\textbf{29}}&&{Vincristina \(1,5\) mg/m\(^2\)/dia}&\multicolumn{1}{c}{}&&{(  ) Sim (  ) Não}&\\
    \hline
    \multicolumn{1}{c|}{\textbf{36}}&&{Vincristina \(1,5\) mg/m\(^2\)/dia}&\multicolumn{1}{c}{}&&{(  ) Sim (  ) Não}&\\
    \hline
\end{longtable}
\begin{longtable}{p{1cm}c|p{5cm}|p{1.5cm}p{1.5cm}|c|c}
	\hline
	\multicolumn{7}{c}{Ciclo 3} \\
	\hline
	\multicolumn{1}{c|}{\multirow{1}{*}{\textbf{Dia}}}&{Data}&{}&\multicolumn{1}{c|}{Leuco}&\multicolumn{1}{c|}{Plaq}&{Administrado}&{Rubrica} \\
    \hline
    \multicolumn{1}{c|}{\multirow{3}{*}{\textbf{43}}}&&{Vincristina \(0,05\) mg/kg}&\multicolumn{1}{c|}{\(>10^3/mm^3\)}&\multicolumn{1}{c|}{\(>10^5/mm^3\)}&{(  ) Sim (  ) Não}&\\
    \cline{4-5}
    \multicolumn{1}{c|}{}&&{Etoposido \(1,5\) mg/kg/dia}&\multicolumn{1}{c|}{}&&{(  ) Sim (  ) Não}&\\
    \cline{4-5}
    \multicolumn{1}{c|}{}&\multirow{1}{*}{}&{Cisplatina \(3,5mg/kg\)}&&&{(  ) Sim (  ) Não}&\\
    \hline
    \multicolumn{1}{c|}{\multirow{3}{*}{\textbf{44}}}&&{Ciclofosfamida \(55\) mg/kg/dia}&{}&&{(  ) Sim (  ) Não}&\\
    \multicolumn{1}{c|}{}&&{MESNA \(55\) mg/kg/dia \(\times 0,1 \:e\: 5\)h}&&&{(  ) Sim (  ) Não}&\\
    \multicolumn{1}{c|}{}&&{Etoposido \(1,5\) mg/kg/dia}&&&{(  ) Sim (  ) Não}&\\
    \hline
    \multicolumn{1}{c|}{\multirow{3}{*}{\textbf{45}}}&&{Ciclofosfamida \(55\) mg/kg/dia}&{}&&{(  ) Sim (  ) Não}&\\
    \multicolumn{1}{c|}{}&&{MESNA \(55\) mg/kg/dia \(\times 0,1 \:e\: 5\)h}&&&{(  ) Sim (  ) Não}&\\
    \multicolumn{1}{c|}{}&\multirow{1}{*}{}&{Etoposido \(1,5\) mg/kg/dia}&{}&&{(  ) Sim (  ) Não}&\\
    \hline
    \multicolumn{1}{c|}{\textbf{46-55}}&&{G-CSF \(5 \mu\)g/kg/dia }&&&{(  ) Sim (  ) Não}&\\
    \hline
\end{longtable}
\end{center}
\clearpage

\vspace{5mm}
\entrywithlabel[.96\hsize]{\textbf{Nome}}\hfill \\

\entrywithlabel[.45\hsize]{\textbf{Peso}}\hfill  \entrywithlabel[.45\hsize]{\textbf{Estatura}}

\begin{center}
\begin{longtable}{p{1cm}c|p{4.5cm}|p{1.7cm}p{2cm}|c|c}
	\hline
	\multicolumn{1}{c|}{\multirow{1}{*}{\textbf{Dia}}}&{Data}&{}&{}&&{Administrado}&{Rubrica} \\
    \hline
    \multicolumn{1}{c|}{\textbf{50}}&&{Vincristina \(1,5\) mg/m\(^2\)/dia}&\multicolumn{1}{c}{}&&{(  ) Sim (  ) Não}&\\
    \hline
    \multicolumn{1}{c|}{\textbf{57}}&&{Vincristina \(1,5\) mg/m\(^2\)/dia}&\multicolumn{1}{c}{}&&{(  ) Sim (  ) Não}&\\
    \hline
\end{longtable}

\begin{longtable}{p{1cm}c|p{5cm}|p{1.5cm}p{1.5cm}|c|c}
	\hline
	\multicolumn{7}{c}{Ciclo 4} \\
	\hline
	\multicolumn{1}{c|}{\multirow{1}{*}{\textbf{Dia}}}&{Data}&{}&\multicolumn{1}{c|}{Leuco}&\multicolumn{1}{c|}{Plaq}&{Administrado}&{Rubrica} \\
    \hline
    \multicolumn{1}{c|}{\multirow{2}{*}{\textbf{64}}}&&{Etoposido \(1,5\) mg/kg/dia}&\multicolumn{1}{c|}{}&&{(  ) Sim (  ) Não}&\\
    \cline{4-5}
    \multicolumn{1}{c|}{}&\multirow{1}{*}{}&{Cisplatina \(3,5mg/kg\)}&&&{(  ) Sim (  ) Não}&\\
    \hline
    \multicolumn{1}{c|}{\multirow{3}{*}{\textbf{65}}}&&{Ciclofosfamida \(55\) mg/kg/dia}&{}&&{(  ) Sim (  ) Não}&\\
    \multicolumn{1}{c|}{}&&{MESNA \(55\) mg/kg/dia \(\times 0,1 \:e\: 5\)h}&&&{(  ) Sim (  ) Não}&\\
    \multicolumn{1}{c|}{}&&{Etoposido \(1,5\) mg/kg/dia}&&&{(  ) Sim (  ) Não}&\\
    \hline
    \multicolumn{1}{c|}{\multirow{3}{*}{\textbf{66}}}&&{Ciclofosfamida \(55\) mg/kg/dia}&{}&&{(  ) Sim (  ) Não}&\\
    \multicolumn{1}{c|}{}&&{MESNA \(55\) mg/kg/dia \(\times 0,1 \:e\: 5\)h}&&&{(  ) Sim (  ) Não}&\\
    \multicolumn{1}{c|}{}&\multirow{1}{*}{}&{Etoposido \(1,5\) mg/kg/dia}&{}&&{(  ) Sim (  ) Não}&\\
    \hline
    \multicolumn{1}{c|}{\textbf{67-76}}&&{G-CSF \(5 \mu\)g/kg/dia }&&&{(  ) Sim (  ) Não}&\\
    \hline
\end{longtable}
\begin{longtable}{p{1cm}c|p{5cm}|p{1.5cm}p{1.5cm}|c|c}
	\hline
	\multicolumn{7}{c}{Ciclo 5} \\
	\hline
	\multicolumn{1}{c|}{\multirow{1}{*}{\textbf{Dia}}}&{Data}&{}&\multicolumn{1}{c|}{Leuco}&\multicolumn{1}{c|}{Plaq}&{Administrado}&{Rubrica} \\
    \hline
    \multicolumn{1}{c|}{\multirow{2}{*}{\textbf{85}}}&&{Etoposido \(1,5\) mg/kg/dia}&\multicolumn{1}{c|}{}&&{(  ) Sim (  ) Não}&\\
    \cline{4-5}
    \multicolumn{1}{c|}{}&\multirow{1}{*}{}&{Cisplatina \(3,5mg/kg\)}&&&{(  ) Sim (  ) Não}&\\
    \hline
    \multicolumn{1}{c|}{\multirow{3}{*}{\textbf{86}}}&&{Ciclofosfamida \(55\) mg/kg/dia}&{}&&{(  ) Sim (  ) Não}&\\
    \multicolumn{1}{c|}{}&&{MESNA \(55\) mg/kg/dia \(\times 0,1 \:e\: 5\)h}&&&{(  ) Sim (  ) Não}&\\
    \multicolumn{1}{c|}{}&&{Etoposido \(1,5\) mg/kg/dia}&&&{(  ) Sim (  ) Não}&\\
    \hline
    \multicolumn{1}{c|}{\multirow{3}{*}{\textbf{87}}}&&{Ciclofosfamida \(55\) mg/kg/dia}&{}&&{(  ) Sim (  ) Não}&\\
    \multicolumn{1}{c|}{}&&{MESNA \(55\) mg/kg/dia \(\times 0,1 \:e\: 5\)h}&&&{(  ) Sim (  ) Não}&\\
    \multicolumn{1}{c|}{}&\multirow{1}{*}{}&{Etoposido \(1,5\) mg/kg/dia}&{}&&{(  ) Sim (  ) Não}&\\
    \hline
    \multicolumn{1}{c|}{\textbf{88-97}}&&{G-CSF \(5 \mu\)g/kg/dia }&&&{(  ) Sim (  ) Não}&\\
    \hline
\end{longtable}
\textbf{REAVALIAR}
\end{center}
\textbf{Menos de 36 meses de idade ao terminar indução:} Sem doença residual – ir para manutenção. Doença residual: considerar \textit{second look surgery}.\\
\textbf{Mais de 36 meses de idade ao terminar a indução:} Sem doença residual, nem metástase – manutenção. Doença residual/metástase – RT antes da manutenção. Reiniciar QT 4 semanas após o fim da RT e completar a manutenção.

\subsection{Manutenção: 08 ciclos}
\begin{center}
\begin{longtable}{p{1cm}c|p{5cm}|p{1.5cm}p{1.5cm}|c|c}
	\hline
	\multicolumn{7}{c}{Ciclo 1} \\
	\hline
	\multicolumn{1}{c|}{\multirow{1}{*}{\textbf{Dia}}}&{Data}&{}&\multicolumn{1}{c|}{Leuco}&\multicolumn{1}{c|}{Plaq}&{Administrado}&{Rubrica} \\
    \hline
    \multicolumn{1}{c|}{\multirow{3}{*}{\textbf{1}}}&\multirow{2}{*}{}&{Carboplatina \(10\) mg/kg}&\multicolumn{1}{c|}{\(>10^3\)}&\multicolumn{1}{c|}{\(>10^5\)}&{(  ) Sim (  ) Não}&\\
    \cline{4-5}
    \multicolumn{1}{c|}{}&&{Vincristina \(0,05\) mg/kg}&\multicolumn{1}{c|}{}&&{(  ) Sim (  ) Não}&\\
    \cline{4-5}
    \multicolumn{1}{c|}{}&\multirow{1}{*}{}&{Etoposido \(1,5\) mg/kg/dia}&{}&&{(  ) Sim (  ) Não}&\\
    \cline{1-3}\cline{6-6}
    \multicolumn{1}{c|}{\textbf{2}}&\multirow{1}{*}{}&{Etoposido \(1,5\) mg/kg/dia}&{}&&{(  ) Sim (  ) Não}&\\
    \hline
\end{longtable}
\begin{longtable}{p{1cm}c|p{5cm}|p{1.5cm}p{1.5cm}|c|c}
	\hline
	\multicolumn{1}{c|}{\multirow{1}{*}{\textbf{Dia}}}&{Data}&{}&{}&&{Administrado}&{Rubrica} \\
    \hline
    \multicolumn{1}{c|}{\textbf{8}}&&{Vincristina \(0,05\) mg/kg}&\multicolumn{1}{c}{}&&{(  ) Sim (  ) Não}&\\
    \hline
    \multicolumn{1}{c|}{\textbf{15}}&&{Vincristina \(0,05\) mg/kg}&\multicolumn{1}{c}{}&&{(  ) Sim (  ) Não}&\\
    \hline
    \multicolumn{1}{c|}{\textbf{22}}&&{Vincristina \(0,05\) mg/kg}&\multicolumn{1}{c}{}&&{(  ) Sim (  ) Não}&\\
    \hline
\end{longtable}

\begin{longtable}{p{1cm}c|p{5cm}|p{1.5cm}p{1.5cm}|c|c}
	\hline
	\multicolumn{1}{c|}{\multirow{1}{*}{\textbf{Dia}}}&{Data}&{}&\multicolumn{1}{c|}{Leuco}&\multicolumn{1}{c|}{Plaq}&{Administrado}&{Rubrica} \\
    \hline
    \multicolumn{1}{c|}{\multirow{3}{*}{\textbf{29}}}&&{Ciclofosfamida \(55\) mg/kg/dia}&\multicolumn{1}{c|}{}&&{(  ) Sim (  ) Não}&\\
    \cline{4-5}
    \multicolumn{1}{c|}{}&&{MESNA \(55\) mg/kg/dia \(\times 0,1 \:e\: 5\)h}&&&{(  ) Sim (  ) Não}&\\
    \multicolumn{1}{c|}{}&&{Etoposido \(1,5\) mg/kg/dia}&&&{(  ) Sim (  ) Não}&\\
    \hline
    \multicolumn{1}{c|}{\multirow{1}{*}{\textbf{30}}}&&{Etoposido \(1,5\) mg/kg/dia)}&{}&&{(  ) Sim (  ) Não}&\\
    \hline
\end{longtable}
\end{center}
\begin{center}
\begin{longtable}{p{1cm}c|p{5cm}|p{1.5cm}p{1.5cm}|c|c}
	\hline
	\multicolumn{7}{c}{Ciclo 2} \\
	\hline
	\multicolumn{1}{c|}{\multirow{1}{*}{\textbf{Dia}}}&{Data}&{}&\multicolumn{1}{c|}{Leuco}&\multicolumn{1}{c|}{Plaq}&{Administrado}&{Rubrica} \\
    \hline
    \multicolumn{1}{c|}{\multirow{3}{*}{\textbf{50}}}&\multirow{2}{*}{}&{Carboplatina \(10\) mg/kg}&\multicolumn{1}{c|}{\(>10^3\)}&\multicolumn{1}{c|}{\(>10^5\)}&{(  ) Sim (  ) Não}&\\
    \cline{4-5}
    \multicolumn{1}{c|}{}&&{Vincristina \(0,05\) mg/kg}&\multicolumn{1}{c|}{}&&{(  ) Sim (  ) Não}&\\
    \cline{4-5}
    \multicolumn{1}{c|}{}&\multirow{1}{*}{}&{Etoposido \(1,5\) mg/kg/dia}&{}&&{(  ) Sim (  ) Não}&\\
    \cline{1-3}\cline{6-6}
    \multicolumn{1}{c|}{\textbf{51}}&\multirow{1}{*}{}&{Etoposido \(1,5\) mg/kg/dia}&{}&&{(  ) Sim (  ) Não}&\\
    \hline
\end{longtable}
\end{center}

\vspace{5mm}
\entrywithlabel[.96\hsize]{\textbf{Nome}}\hfill \\

\entrywithlabel[.45\hsize]{\textbf{Peso}}\hfill  \entrywithlabel[.45\hsize]{\textbf{Estatura}}

\begin{center}
\begin{longtable}{p{1cm}c|p{5cm}|p{1.5cm}p{1.5cm}|c|c}
	\hline
	\multicolumn{1}{c|}{\multirow{1}{*}{\textbf{Dia}}}&{Data}&{}&{}&&{Administrado}&{Rubrica} \\
    \hline
    \multicolumn{1}{c|}{\textbf{57}}&&{Vincristina \(0,05\) mg/kg}&\multicolumn{1}{c}{}&&{(  ) Sim (  ) Não}&\\
    \hline
    \multicolumn{1}{c|}{\textbf{64}}&&{Vincristina \(0,05\) mg/kg}&\multicolumn{1}{c}{}&&{(  ) Sim (  ) Não}&\\
    \hline
    \multicolumn{1}{c|}{\textbf{71}}&&{Vincristina \(0,05\) mg/kg}&\multicolumn{1}{c}{}&&{(  ) Sim (  ) Não}&\\
    \hline
\end{longtable}
\begin{longtable}{p{1cm}c|p{5cm}|p{1.5cm}p{1.5cm}|c|c}
	\hline
	\multicolumn{1}{c|}{\multirow{1}{*}{\textbf{Dia}}}&{Data}&{}&\multicolumn{1}{c|}{Leuco}&\multicolumn{1}{c|}{Plaq}&{Administrado}&{Rubrica} \\
    \hline
    \multicolumn{1}{c|}{\multirow{3}{*}{\textbf{78}}}&&{Ciclofosfamida \(55\) mg/kg/dia}&\multicolumn{1}{c|}{}&&{(  ) Sim (  ) Não}&\\
    \cline{4-5}
    \multicolumn{1}{c|}{}&&{MESNA \(55\) mg/kg/dia \(\times 0,1 \:e\: 5\)h}&&&{(  ) Sim (  ) Não}&\\
    \multicolumn{1}{c|}{}&&{Etoposido \(1,5\) mg/kg/dia}&&&{(  ) Sim (  ) Não}&\\
    \hline
    \multicolumn{1}{c|}{\multirow{1}{*}{\textbf{79}}}&&{Etoposido \(1,5\) mg/kg/dia)}&{}&&{(  ) Sim (  ) Não}&\\
    \hline
\end{longtable}

\begin{longtable}{p{1cm}c|p{5cm}|p{1.5cm}p{1.5cm}|c|c}
	\hline
	\multicolumn{7}{c}{Ciclo 3} \\
	\hline
	\multicolumn{1}{c|}{\multirow{1}{*}{\textbf{Dia}}}&{Data}&{}&\multicolumn{1}{c|}{Leuco}&\multicolumn{1}{c|}{Plaq}&{Administrado}&{Rubrica} \\
    \hline
    \multicolumn{1}{c|}{\multirow{3}{*}{\textbf{99}}}&\multirow{2}{*}{}&{Carboplatina \(10\) mg/kg}&\multicolumn{1}{c|}{\(>10^3\)}&\multicolumn{1}{c|}{\(>10^5\)}&{(  ) Sim (  ) Não}&\\
    \cline{4-5}
    \multicolumn{1}{c|}{}&&{Vincristina \(0,05\) mg/kg}&\multicolumn{1}{c|}{}&&{(  ) Sim (  ) Não}&\\
    \cline{4-5}
    \multicolumn{1}{c|}{}&\multirow{1}{*}{}&{Etoposido \(1,5\) mg/kg/dia}&{}&&{(  ) Sim (  ) Não}&\\
    \cline{1-3}\cline{6-6}
    \multicolumn{1}{c|}{\textbf{100}}&\multirow{1}{*}{}&{Etoposido \(1,5\) mg/kg/dia}&{}&&{(  ) Sim (  ) Não}&\\
    \hline
\end{longtable}
\begin{longtable}{p{1cm}c|p{5cm}|p{1.5cm}p{1.5cm}|c|c}
	\hline
	\multicolumn{1}{c|}{\multirow{1}{*}{\textbf{Dia}}}&{Data}&{}&{}&&{Administrado}&{Rubrica} \\
    \hline
    \multicolumn{1}{c|}{\textbf{106}}&&{Vincristina \(0,05\) mg/kg}&\multicolumn{1}{c}{}&&{(  ) Sim (  ) Não}&\\
    \hline
    \multicolumn{1}{c|}{\textbf{113}}&&{Vincristina \(0,05\) mg/kg}&\multicolumn{1}{c}{}&&{(  ) Sim (  ) Não}&\\
    \hline
    \multicolumn{1}{c|}{\textbf{120}}&&{Vincristina \(0,05\) mg/kg}&\multicolumn{1}{c}{}&&{(  ) Sim (  ) Não}&\\
    \hline
\end{longtable}

\begin{longtable}{p{1cm}c|p{5cm}|p{1.5cm}p{1.5cm}|c|c}
	\hline
	\multicolumn{1}{c|}{\multirow{1}{*}{\textbf{Dia}}}&{Data}&{}&\multicolumn{1}{c|}{Leuco}&\multicolumn{1}{c|}{Plaq}&{Administrado}&{Rubrica} \\
    \hline
    \multicolumn{1}{c|}{\multirow{3}{*}{\textbf{127}}}&&{Ciclofosfamida \(55\) mg/kg/dia}&\multicolumn{1}{c|}{}&&{(  ) Sim (  ) Não}&\\
    \cline{4-5}
    \multicolumn{1}{c|}{}&&{MESNA \(55\) mg/kg/dia \(\times 0,1 \:e\: 5\)h}&&&{(  ) Sim (  ) Não}&\\
    \multicolumn{1}{c|}{}&&{Etoposido \(1,5\) mg/kg/dia}&&&{(  ) Sim (  ) Não}&\\
    \hline
    \multicolumn{1}{c|}{\multirow{1}{*}{\textbf{128}}}&&{Etoposido \(1,5\) mg/kg/dia)}&{}&&{(  ) Sim (  ) Não}&\\
    \hline
\end{longtable}
\begin{longtable}{p{1cm}c|p{5cm}|p{1.5cm}p{1.5cm}|c|c}
	\hline
	\multicolumn{7}{c}{Ciclo 4} \\
	\hline
	\multicolumn{1}{c|}{\multirow{1}{*}{\textbf{Dia}}}&{Data}&{}&\multicolumn{1}{c|}{Leuco}&\multicolumn{1}{c|}{Plaq}&{Administrado}&{Rubrica} \\
    \hline
    \multicolumn{1}{c|}{\multirow{3}{*}{\textbf{148}}}&\multirow{2}{*}{}&{Carboplatina \(10\) mg/kg}&\multicolumn{1}{c|}{\(>10^3\)}&\multicolumn{1}{c|}{\(>10^5\)}&{(  ) Sim (  ) Não}&\\
    \cline{4-5}
    \multicolumn{1}{c|}{}&&{Vincristina \(0,05\) mg/kg}&\multicolumn{1}{c|}{}&&{(  ) Sim (  ) Não}&\\
    \cline{4-5}
    \multicolumn{1}{c|}{}&\multirow{1}{*}{}&{Etoposido \(1,5\) mg/kg/dia}&{}&&{(  ) Sim (  ) Não}&\\
    \cline{1-3}\cline{6-6}
    \multicolumn{1}{c|}{\textbf{149}}&\multirow{1}{*}{}&{Etoposido \(1,5\) mg/kg/dia}&{}&&{(  ) Sim (  ) Não}&\\
    \hline
\end{longtable}
\begin{longtable}{p{1cm}c|p{5cm}|p{1.5cm}p{1.5cm}|c|c}
	\hline
	\multicolumn{1}{c|}{\multirow{1}{*}{\textbf{Dia}}}&{Data}&{}&{}&&{Administrado}&{Rubrica} \\
    \hline
    \multicolumn{1}{c|}{\textbf{155}}&&{Vincristina \(0,05\) mg/kg}&\multicolumn{1}{c}{}&&{(  ) Sim (  ) Não}&\\
    \hline
    \multicolumn{1}{c|}{\textbf{162}}&&{Vincristina \(0,05\) mg/kg}&\multicolumn{1}{c}{}&&{(  ) Sim (  ) Não}&\\
    \hline
    \multicolumn{1}{c|}{\textbf{169}}&&{Vincristina \(0,05\) mg/kg}&\multicolumn{1}{c}{}&&{(  ) Sim (  ) Não}&\\
    \hline
\end{longtable}

\begin{longtable}{p{1cm}c|p{5cm}|p{1.5cm}p{1.5cm}|c|c}
	\hline
	\multicolumn{1}{c|}{\multirow{1}{*}{\textbf{Dia}}}&{Data}&{}&\multicolumn{1}{c|}{Leuco}&\multicolumn{1}{c|}{Plaq}&{Administrado}&{Rubrica} \\
    \hline
    \multicolumn{1}{c|}{\multirow{3}{*}{\textbf{176}}}&&{Ciclofosfamida \(55\) mg/kg/dia}&\multicolumn{1}{c|}{}&&{(  ) Sim (  ) Não}&\\
    \cline{4-5}
    \multicolumn{1}{c|}{}&&{MESNA \(55\) mg/kg/dia \(\times 0,1 \:e\: 5\)h}&&&{(  ) Sim (  ) Não}&\\
    \multicolumn{1}{c|}{}&&{Etoposido \(1,5\) mg/kg/dia}&&&{(  ) Sim (  ) Não}&\\
    \hline
    \multicolumn{1}{c|}{\multirow{1}{*}{\textbf{177}}}&&{Etoposido \(1,5\) mg/kg/dia)}&{}&&{(  ) Sim (  ) Não}&\\
    \hline
\end{longtable}
\end{center}
\begin{center}
\begin{longtable}{p{1cm}c|p{5cm}|p{1.5cm}p{1.5cm}|c|c}
	\hline
	\multicolumn{7}{c}{Ciclo 5} \\
	\hline
	\multicolumn{1}{c|}{\multirow{1}{*}{\textbf{Dia}}}&{Data}&{}&\multicolumn{1}{c|}{Leuco}&\multicolumn{1}{c|}{Plaq}&{Administrado}&{Rubrica} \\
    \hline
    \multicolumn{1}{c|}{\multirow{3}{*}{\textbf{197}}}&\multirow{2}{*}{}&{Carboplatina \(10\) mg/kg}&\multicolumn{1}{c|}{\(>10^3\)}&\multicolumn{1}{c|}{\(>10^5\)}&{(  ) Sim (  ) Não}&\\
    \cline{4-5}
    \multicolumn{1}{c|}{}&&{Vincristina \(0,05\) mg/kg}&\multicolumn{1}{c|}{}&&{(  ) Sim (  ) Não}&\\
    \cline{4-5}
    \multicolumn{1}{c|}{}&\multirow{1}{*}{}&{Etoposido \(1,5\) mg/kg/dia}&{}&&{(  ) Sim (  ) Não}&\\
    \cline{1-3}\cline{6-6}
    \multicolumn{1}{c|}{\textbf{198}}&\multirow{1}{*}{}&{Etoposido \(1,5\) mg/kg/dia}&{}&&{(  ) Sim (  ) Não}&\\
    \hline
\end{longtable}
\begin{longtable}{p{1cm}c|p{5cm}|p{1.5cm}p{1.5cm}|c|c}
	\hline
	\multicolumn{1}{c|}{\multirow{1}{*}{\textbf{Dia}}}&{Data}&{}&{}&&{Administrado}&{Rubrica} \\
    \hline
    \multicolumn{1}{c|}{\textbf{204}}&&{Vincristina \(0,05\) mg/kg}&\multicolumn{1}{c}{}&&{(  ) Sim (  ) Não}&\\
    \hline
    \multicolumn{1}{c|}{\textbf{211}}&&{Vincristina \(0,05\) mg/kg}&\multicolumn{1}{c}{}&&{(  ) Sim (  ) Não}&\\
    \hline
    \multicolumn{1}{c|}{\textbf{218}}&&{Vincristina \(0,05\) mg/kg}&\multicolumn{1}{c}{}&&{(  ) Sim (  ) Não}&\\
    \hline
\end{longtable}
\end{center}

\vspace{5mm}
\entrywithlabel[.96\hsize]{\textbf{Nome}}\hfill \\

\entrywithlabel[.45\hsize]{\textbf{Peso}}\hfill  \entrywithlabel[.45\hsize]{\textbf{Estatura}}

\begin{center}
\begin{longtable}{p{1cm}c|p{5cm}|p{1.5cm}p{1.5cm}|c|c}
	\hline
	\multicolumn{1}{c|}{\multirow{1}{*}{\textbf{Dia}}}&{Data}&{}&\multicolumn{1}{c|}{Leuco}&\multicolumn{1}{c|}{Plaq}&{Administrado}&{Rubrica} \\
    \hline
    \multicolumn{1}{c|}{\multirow{3}{*}{\textbf{225}}}&&{Ciclofosfamida \(55\) mg/kg/dia}&\multicolumn{1}{c|}{}&&{(  ) Sim (  ) Não}&\\
    \cline{4-5}
    \multicolumn{1}{c|}{}&&{MESNA \(55\) mg/kg/dia \(\times 0,1 \:e\: 5\)h}&&&{(  ) Sim (  ) Não}&\\
    \multicolumn{1}{c|}{}&&{Etoposido \(1,5\) mg/kg/dia}&&&{(  ) Sim (  ) Não}&\\
    \hline
    \multicolumn{1}{c|}{\multirow{1}{*}{\textbf{226}}}&&{Etoposido \(1,5\) mg/kg/dia)}&{}&&{(  ) Sim (  ) Não}&\\
    \hline
\end{longtable}
\end{center}
\begin{center}
\begin{longtable}{p{1cm}c|p{5cm}|p{1.5cm}p{1.5cm}|c|c}
	\hline
	\multicolumn{7}{c}{Ciclo 6} \\
	\hline
	\multicolumn{1}{c|}{\multirow{1}{*}{\textbf{Dia}}}&{Data}&{}&\multicolumn{1}{c|}{Leuco}&\multicolumn{1}{c|}{Plaq}&{Administrado}&{Rubrica} \\
    \hline
    \multicolumn{1}{c|}{\multirow{3}{*}{\textbf{246}}}&\multirow{2}{*}{}&{Carboplatina \(10\) mg/kg}&\multicolumn{1}{c|}{\(>10^3\)}&\multicolumn{1}{c|}{\(>10^5\)}&{(  ) Sim (  ) Não}&\\
    \cline{4-5}
    \multicolumn{1}{c|}{}&&{Vincristina \(0,05\) mg/kg}&\multicolumn{1}{c|}{}&&{(  ) Sim (  ) Não}&\\
    \cline{4-5}
    \multicolumn{1}{c|}{}&\multirow{1}{*}{}&{Etoposido \(1,5\) mg/kg/dia}&{}&&{(  ) Sim (  ) Não}&\\
    \cline{1-3}\cline{6-6}
    \multicolumn{1}{c|}{\textbf{247}}&\multirow{1}{*}{}&{Etoposido \(1,5\) mg/kg/dia}&{}&&{(  ) Sim (  ) Não}&\\
    \hline
\end{longtable}
\begin{longtable}{p{1cm}c|p{5cm}|p{1.5cm}p{1.5cm}|c|c}
	\hline
	\multicolumn{1}{c|}{\multirow{1}{*}{\textbf{Dia}}}&{Data}&{}&{}&&{Administrado}&{Rubrica} \\
    \hline
    \multicolumn{1}{c|}{\textbf{253}}&&{Vincristina \(0,05\) mg/kg}&\multicolumn{1}{c}{}&&{(  ) Sim (  ) Não}&\\
    \hline
    \multicolumn{1}{c|}{\textbf{260}}&&{Vincristina \(0,05\) mg/kg}&\multicolumn{1}{c}{}&&{(  ) Sim (  ) Não}&\\
    \hline
    \multicolumn{1}{c|}{\textbf{267}}&&{Vincristina \(0,05\) mg/kg}&\multicolumn{1}{c}{}&&{(  ) Sim (  ) Não}&\\
    \hline
\end{longtable}

\begin{longtable}{p{1cm}c|p{5cm}|p{1.5cm}p{1.5cm}|c|c}
	\hline
	\multicolumn{1}{c|}{\multirow{1}{*}{\textbf{Dia}}}&{Data}&{}&\multicolumn{1}{c|}{Leuco}&\multicolumn{1}{c|}{Plaq}&{Administrado}&{Rubrica} \\
    \hline
    \multicolumn{1}{c|}{\multirow{3}{*}{\textbf{274}}}&&{Ciclofosfamida \(55\) mg/kg/dia}&\multicolumn{1}{c|}{}&&{(  ) Sim (  ) Não}&\\
    \cline{4-5}
    \multicolumn{1}{c|}{}&&{MESNA \(55\) mg/kg/dia \(\times 0,1 \:e\: 5\)h}&&&{(  ) Sim (  ) Não}&\\
    \multicolumn{1}{c|}{}&&{Etoposido \(1,5\) mg/kg/dia}&&&{(  ) Sim (  ) Não}&\\
    \hline
    \multicolumn{1}{c|}{\multirow{1}{*}{\textbf{275}}}&&{Etoposido \(1,5\) mg/kg/dia)}&{}&&{(  ) Sim (  ) Não}&\\
    \hline
\end{longtable}
\end{center}
\begin{center}
\begin{longtable}{p{1cm}c|p{5cm}|p{1.5cm}p{1.5cm}|c|c}
	\hline
	\multicolumn{7}{c}{Ciclo 7} \\
	\hline
	\multicolumn{1}{c|}{\multirow{1}{*}{\textbf{Dia}}}&{Data}&{}&\multicolumn{1}{c|}{Leuco}&\multicolumn{1}{c|}{Plaq}&{Administrado}&{Rubrica} \\
    \hline
    \multicolumn{1}{c|}{\multirow{3}{*}{\textbf{295}}}&\multirow{2}{*}{}&{Carboplatina \(10\) mg/kg}&\multicolumn{1}{c|}{\(>10^3\)}&\multicolumn{1}{c|}{\(>10^5\)}&{(  ) Sim (  ) Não}&\\
    \cline{4-5}
    \multicolumn{1}{c|}{}&&{Vincristina \(0,05\) mg/kg}&\multicolumn{1}{c|}{}&&{(  ) Sim (  ) Não}&\\
    \cline{4-5}
    \multicolumn{1}{c|}{}&\multirow{1}{*}{}&{Etoposido \(1,5\) mg/kg/dia}&{}&&{(  ) Sim (  ) Não}&\\
    \cline{1-3}\cline{6-6}
    \multicolumn{1}{c|}{\textbf{296}}&\multirow{1}{*}{}&{Etoposido \(1,5\) mg/kg/dia}&{}&&{(  ) Sim (  ) Não}&\\
    \hline
\end{longtable}
\begin{longtable}{p{1cm}c|p{5cm}|p{1.5cm}p{1.5cm}|c|c}
	\hline
	\multicolumn{1}{c|}{\multirow{1}{*}{\textbf{Dia}}}&{Data}&{}&{}&&{Administrado}&{Rubrica} \\
    \hline
    \multicolumn{1}{c|}{\textbf{302}}&&{Vincristina \(0,05\) mg/kg}&\multicolumn{1}{c}{}&&{(  ) Sim (  ) Não}&\\
    \hline
    \multicolumn{1}{c|}{\textbf{309}}&&{Vincristina \(0,05\) mg/kg}&\multicolumn{1}{c}{}&&{(  ) Sim (  ) Não}&\\
    \hline
    \multicolumn{1}{c|}{\textbf{316}}&&{Vincristina \(0,05\) mg/kg}&\multicolumn{1}{c}{}&&{(  ) Sim (  ) Não}&\\
    \hline
\end{longtable}

\begin{longtable}{p{1cm}c|p{5cm}|p{1.5cm}p{1.5cm}|c|c}
	\hline
	\multicolumn{1}{c|}{\multirow{1}{*}{\textbf{Dia}}}&{Data}&{}&\multicolumn{1}{c|}{Leuco}&\multicolumn{1}{c|}{Plaq}&{Administrado}&{Rubrica} \\
    \hline
    \multicolumn{1}{c|}{\multirow{3}{*}{\textbf{323}}}&&{Ciclofosfamida \(55\) mg/kg/dia}&\multicolumn{1}{c|}{}&&{(  ) Sim (  ) Não}&\\
    \cline{4-5}
    \multicolumn{1}{c|}{}&&{MESNA \(55\) mg/kg/dia \(\times 0,1 \:e\: 5\)h}&&&{(  ) Sim (  ) Não}&\\
    \multicolumn{1}{c|}{}&&{Etoposido \(1,5\) mg/kg/dia}&&&{(  ) Sim (  ) Não}&\\
    \hline
    \multicolumn{1}{c|}{\multirow{1}{*}{\textbf{324}}}&&{Etoposido \(1,5\) mg/kg/dia)}&{}&&{(  ) Sim (  ) Não}&\\
    \hline
\end{longtable}
\end{center}
\begin{center}
\begin{longtable}{p{1cm}c|p{5cm}|p{1.5cm}p{1.5cm}|c|c}
	\hline
	\multicolumn{7}{c}{Ciclo 8} \\
	\hline
	\multicolumn{1}{c|}{\multirow{1}{*}{\textbf{Dia}}}&{Data}&{}&\multicolumn{1}{c|}{Leuco}&\multicolumn{1}{c|}{Plaq}&{Administrado}&{Rubrica} \\
    \hline
    \multicolumn{1}{c|}{\multirow{3}{*}{\textbf{344}}}&\multirow{2}{*}{}&{Carboplatina \(10\) mg/kg}&\multicolumn{1}{c|}{\(>10^3\)}&\multicolumn{1}{c|}{\(>10^5\)}&{(  ) Sim (  ) Não}&\\
    \cline{4-5}
    \multicolumn{1}{c|}{}&&{Vincristina \(0,05\) mg/kg}&\multicolumn{1}{c|}{}&&{(  ) Sim (  ) Não}&\\
    \cline{4-5}
    \multicolumn{1}{c|}{}&\multirow{1}{*}{}&{Etoposido \(1,5\) mg/kg/dia}&{}&&{(  ) Sim (  ) Não}&\\
    \cline{1-3}\cline{6-6}
    \multicolumn{1}{c|}{\textbf{345}}&\multirow{1}{*}{}&{Etoposido \(1,5\) mg/kg/dia}&{}&&{(  ) Sim (  ) Não}&\\
    \hline
\end{longtable}
\begin{longtable}{p{1cm}c|p{5cm}|p{1.5cm}p{1.5cm}|c|c}
	\hline
	\multicolumn{1}{c|}{\multirow{1}{*}{\textbf{Dia}}}&{Data}&{}&{}&&{Administrado}&{Rubrica} \\
    \hline
    \multicolumn{1}{c|}{\textbf{351}}&&{Vincristina \(0,05\) mg/kg}&\multicolumn{1}{c}{}&&{(  ) Sim (  ) Não}&\\
    \hline
    \multicolumn{1}{c|}{\textbf{358}}&&{Vincristina \(0,05\) mg/kg}&\multicolumn{1}{c}{}&&{(  ) Sim (  ) Não}&\\
    \hline
    \multicolumn{1}{c|}{\textbf{365}}&&{Vincristina \(0,05\) mg/kg}&\multicolumn{1}{c}{}&&{(  ) Sim (  ) Não}&\\
    \hline
\end{longtable}
\end{center}

\vspace{5mm}
\entrywithlabel[.96\hsize]{\textbf{Nome}}\hfill \\

\entrywithlabel[.45\hsize]{\textbf{Peso}}\hfill  \entrywithlabel[.45\hsize]{\textbf{Estatura}}

\begin{center}
\begin{longtable}{p{1cm}c|p{5cm}|p{1.5cm}p{1.5cm}|c|c}
	\hline
	\multicolumn{1}{c|}{\multirow{1}{*}{\textbf{Dia}}}&{Data}&{}&\multicolumn{1}{c|}{Leuco}&\multicolumn{1}{c|}{Plaq}&{Administrado}&{Rubrica} \\
    \hline
    \multicolumn{1}{c|}{\multirow{3}{*}{\textbf{372}}}&&{Ciclofosfamida \(55\) mg/kg/dia}&\multicolumn{1}{c|}{}&&{(  ) Sim (  ) Não}&\\
    \cline{4-5}
    \multicolumn{1}{c|}{}&&{MESNA \(55\) mg/kg/dia \(\times 0,1 \:e\: 5\)h}&&&{(  ) Sim (  ) Não}&\\
    \multicolumn{1}{c|}{}&&{Etoposido \(1,5\) mg/kg/dia}&&&{(  ) Sim (  ) Não}&\\
    \hline
    \multicolumn{1}{c|}{\multirow{1}{*}{\textbf{373}}}&&{Etoposido \(1,5\) mg/kg/dia)}&{}&&{(  ) Sim (  ) Não}&\\
    \hline
\end{longtable}
\textit{\textbf{Reavaliar com imagem – Re-operação se possível}}

\textit{\textbf{Encaminhar para Radioterapia}}

\textit{\textbf{Final de Protocolo}}
\end{center}
\subsection{Modificações de dose:} 

Adiar se L < 1000/mm\(^3\) ou P < 100000/mm\(^3\). Se atraso maior que 7 dias, reduzir dose de ciclofosfamida em 20%.
Toxicidade grau 3-4 pela VCR, suspender dose seguinte. Reiniciar com dose normal. Recorrência: reduzir dose.
Bilirrunina total de 1,5-1,9 mg/dl, reduzir VCR para 1,0 mg/m2; se bilirrubina > 1,9 mg/dl, suspender uma dose de VCR.

G-CSF: se ocorrer atraso maior que 1 semana no início do próximo ciclo, fazer G-CSF imediatamente após a droga que causou neutropenia. Se ocorrer infecção grave com neutropenia, tratar a infecção e iniciar G-CSF imediatamente e fazer no próximo ciclo. Se novo episódio infeccioso ocorrer apesar de usar G-CSF, reduzir dose em 25\% da droga causadora da neutropenia.

Se o clearance de creatinina <50\% basal ou <60, suspender CDDP. No ciclo seguinte, se exames normalizados, fazer 50\% da dose de CDDP. Aumente novamente para 100\% somente no terceiro ciclo, se exames mantiverem-se normais.
Se ocorrer redução de 20dB ou mais em freqüências auditivas baixas (500-2000Hz), reduzir carboplatina em 50\%. Se ocorrer redução de 30dB na faixa de 4000-8000 Hz), reduzir carboplatina em 50\%. Ototoxicidade grau IV: interromper carboplatina até nível de lesão retornar ao grau II.

\cleardoublepage
\chapter{Protocolos \textit{off-label} (não padronizados)}
\cleardoublepage
\section{GLIOMA DE BAIXO GRAU: RECORRÊNCIA APÓS MÚLTIPLOS TRATAMENTOS (protocolo \textit{off-label})}
{\let\thefootnote\relax\footnotetext{Versão Janeiro/2015}}
\textbf{Racional:} em 2007, Nicholson et al publicaram um ensaio fase II do COG de pacientes pediátricos com tumores recorrentes tratados com ciclos mensais de temozolomida. A série incluía 113 pacientes com tumores cerebrais, sendo que 22 tinham astrocitomas de baixo grau recorrentes e 8 tinham outros tumors de baixo grau. Neste subgrupo específico, os pacientes mostraram uma sobrevida livre de progressão prolongada, com cerca de 40\% dos pacientes mostrando estabilidade de doença.

\textbf{Elegível:} pacientes com astrocitoma pilocítico, pilomixóide, difuso (ou fibrilar), oligodendroglioma, ganglioglioma, tumores mistos, tumores de vias ópticas/hipotálamo, tumores focais de tronco. Todos os pacientes devem ter feito, pelo menos, 2 (dois) esquemas de QT previamente (recorrência múltipla). Pacientes com contra-indicação à RT (menores de 5 anos e portadores de NF-1) podem ser incluídos. Somente iniciar o protocolo após a assinatura do TERMO DE CONSENTIMENTO INFORMADO, conforme acordado. NÃO INICIAR ESTE PROTOCOLO EM CRIANÇAS GRAVEMENTE ENFERMAS.

\textbf{Alternativa:} a conduta expectante é uma opção, uma vez que, via de regra, o crescimento destes tumores é lento e sua progressão demora anos, ou mesmo décadas. Pacientes de maior risco, como aqueles com lesões de vias ópticas ou hipotálamo, síndrome diencefálica ou com lesões de crescimento rápido devem ser tratados sem grande demora. Se possível, uma nova ressecção cirúrgica deve ser avaliada. A principal alternativa adjuvante para pacientes com mais de 5 anos e sem NF-1 é a RT local. Pacientes com astrocitomas difusos têm maior risco de transformação maligna após RT. 

\vspace{5mm}
\entrywithlabel[.96\hsize]{\textbf{Nome}}\hfill \\

\entrywithlabel[.45\hsize]{\textbf{Peso}}\hfill  \entrywithlabel[.45\hsize]{\textbf{Estatura}}

\subsection{Quimioterapia: 10 ciclos}
\begin{center}
\begin{longtable}{p{1cm}p{4cm}|p{1cm}|p{5cm}|p{3cm}}
	\hline
	\multicolumn{5}{c}{\textbf{CICLO 1}}\\
\hline
    \multicolumn{1}{c|}{\multirow{1}{*}{\textbf{Dia}}}&{Dose}&{Data}&{Administrado}&{Rubrica} \\
    \hline
    \multicolumn{1}{c|}{\multirow{1}{*}{\textbf{D1}}}&{Temozolomida \(200\) mg/m\(^2\)}&&{(  ) Sim (  ) Não}&\\
    \multicolumn{1}{c|}{\multirow{1}{*}{\textbf{D2}}}&{Temozolomida \(200\) mg/m\(^2\)}&&{(  ) Sim (  ) Não}&\\
    \multicolumn{1}{c|}{\multirow{1}{*}{\textbf{D3}}}&{Temozolomida \(200\) mg/m\(^2\)}&&{(  ) Sim (  ) Não}&\\
    \multicolumn{1}{c|}{\multirow{1}{*}{\textbf{D4}}}&{Temozolomida \(200\) mg/m\(^2\)}&&{(  ) Sim (  ) Não}&\\
    \multicolumn{1}{c|}{\multirow{1}{*}{\textbf{D5}}}&{Temozolomida \(200\) mg/m\(^2\)}&&{(  ) Sim (  ) Não}&\\
    \hline
    \multicolumn{1}{c|}{\multirow{2}{*}{\textbf{Exames}}}&\multicolumn{2}{l|}{Neut (\(>7,5\times10^2\)):}&{Plaq (\(>7,5\times10^4\)):}&{TGO:}\\
    \cline{2-5}
    \multicolumn{1}{c|}{\multirow{2}{*}{{}}}&\multicolumn{2}{l|}{Creat(\(<1,5\) vezes)}&{BT(\(<1,5\) vezes):}&{TGP:}
    \\
    \hline
\end{longtable}
\textbf{Intervalo de 14 dias.}
\begin{longtable}{p{5cm}|p{5cm}|p{5cm}}
    \hline
    \textbf{Exames (data):}&{Neut (\(>7,5\times10^2\)):}&{Plaq (\(>7,5\times10^4\)):}\\
    \hline
\end{longtable}
\textbf{Intervalo de 14 dias.}
\\
\end{center}
\begin{center}
\begin{longtable}{p{1cm}p{4cm}|p{1cm}|p{5cm}|p{3cm}}
	\hline
	\multicolumn{5}{c}{\textbf{CICLO 2}}\\
\hline
    \multicolumn{1}{c|}{\multirow{1}{*}{\textbf{Dia}}}&{Dose}&{Data}&{Administrado}&{Rubrica} \\
    \hline
    \multicolumn{1}{c|}{\multirow{1}{*}{\textbf{D1}}}&{Temozolomida \(200\) mg/m\(^2\)}&&{(  ) Sim (  ) Não}&\\
    \multicolumn{1}{c|}{\multirow{1}{*}{\textbf{D2}}}&{Temozolomida \(200\) mg/m\(^2\)}&&{(  ) Sim (  ) Não}&\\
    \multicolumn{1}{c|}{\multirow{1}{*}{\textbf{D3}}}&{Temozolomida \(200\) mg/m\(^2\)}&&{(  ) Sim (  ) Não}&\\
    \multicolumn{1}{c|}{\multirow{1}{*}{\textbf{D4}}}&{Temozolomida \(200\) mg/m\(^2\)}&&{(  ) Sim (  ) Não}&\\
    \multicolumn{1}{c|}{\multirow{1}{*}{\textbf{D5}}}&{Temozolomida \(200\) mg/m\(^2\)}&&{(  ) Sim (  ) Não}&\\
    \hline
    \multicolumn{1}{c|}{\multirow{2}{*}{\textbf{Exames}}}&\multicolumn{2}{l|}{Neut (\(>7,5\times10^2\)):}&{Plaq (\(>7,5\times10^4\)):}&{TGO:}\\
    \cline{2-5}
    \multicolumn{1}{c|}{\multirow{2}{*}{{}}}&\multicolumn{2}{l|}{Creat(\(<1,5\) vezes)}&{BT(\(<1,5\) vezes):}&{TGP:}
    \\
    \hline
\end{longtable}
\textbf{Intervalo de 14 dias.}
\begin{longtable}{p{5cm}|p{5cm}|p{5cm}}
    \hline
    \textbf{Exames (data):}&{Neut (\(>7,5\times10^2\)):}&{Plaq (\(>7,5\times10^4\)):}
    \\
    \hline
\end{longtable}
\textbf{Intervalo de 14 dias.}
\end{center}

\begin{center}
\begin{longtable}{p{1cm}p{4cm}|p{1cm}|p{5cm}|p{3cm}}
	\hline
	\multicolumn{5}{c}{\textbf{CICLO 3}}\\
\hline
    \multicolumn{1}{c|}{\multirow{1}{*}{\textbf{Dia}}}&{Dose}&{Data}&{Administrado}&{Rubrica} \\
    \hline
    \multicolumn{1}{c|}{\multirow{1}{*}{\textbf{D1}}}&{Temozolomida \(200\) mg/m\(^2\)}&&{(  ) Sim (  ) Não}&\\
    \multicolumn{1}{c|}{\multirow{1}{*}{\textbf{D2}}}&{Temozolomida \(200\) mg/m\(^2\)}&&{(  ) Sim (  ) Não}&\\
    \multicolumn{1}{c|}{\multirow{1}{*}{\textbf{D3}}}&{Temozolomida \(200\) mg/m\(^2\)}&&{(  ) Sim (  ) Não}&\\
    \multicolumn{1}{c|}{\multirow{1}{*}{\textbf{D4}}}&{Temozolomida \(200\) mg/m\(^2\)}&&{(  ) Sim (  ) Não}&\\
    \multicolumn{1}{c|}{\multirow{1}{*}{\textbf{D5}}}&{Temozolomida \(200\) mg/m\(^2\)}&&{(  ) Sim (  ) Não}&\\
    \hline
    \multicolumn{1}{c|}{\multirow{2}{*}{\textbf{Exames}}}&\multicolumn{2}{l|}{Neut (\(>7,5\times10^2\)):}&{Plaq (\(>7,5\times10^4\)):}&{TGO:}\\
    \cline{2-5}
    \multicolumn{1}{c|}{\multirow{2}{*}{{}}}&\multicolumn{2}{l|}{Creat(\(<1,5\) vezes)}&{BT(\(<1,5\) vezes):}&{TGP:}
    \\
    \hline
\end{longtable}
\textbf{Intervalo de 14 dias.}
\begin{longtable}{p{5cm}|p{5cm}|p{5cm}}
    \hline
    \textbf{Exames (data):}&{Neut (\(>7,5\times10^2\)):}&{Plaq (\(>7,5\times10^4\)):}
    \\
    \hline
\end{longtable}
\textbf{Intervalo de 14 dias.}
\end{center}
\pagebreak

\vspace{5mm}
\entrywithlabel[.96\hsize]{\textbf{Nome}}\hfill \\

\entrywithlabel[.45\hsize]{\textbf{Peso}}\hfill  \entrywithlabel[.45\hsize]{\textbf{Estatura}}

\begin{center}
\begin{longtable}{p{1cm}p{4cm}|p{1cm}|p{5cm}|p{3cm}}
	\hline
	\multicolumn{5}{c}{\textbf{CICLO 4}}\\
\hline
    \multicolumn{1}{c|}{\multirow{1}{*}{\textbf{Dia}}}&{Dose}&{Data}&{Administrado}&{Rubrica} \\
    \hline
    \multicolumn{1}{c|}{\multirow{1}{*}{\textbf{D1}}}&{Temozolomida \(200\) mg/m\(^2\)}&&{(  ) Sim (  ) Não}&\\
    \multicolumn{1}{c|}{\multirow{1}{*}{\textbf{D2}}}&{Temozolomida \(200\) mg/m\(^2\)}&&{(  ) Sim (  ) Não}&\\
    \multicolumn{1}{c|}{\multirow{1}{*}{\textbf{D3}}}&{Temozolomida \(200\) mg/m\(^2\)}&&{(  ) Sim (  ) Não}&\\
    \multicolumn{1}{c|}{\multirow{1}{*}{\textbf{D4}}}&{Temozolomida \(200\) mg/m\(^2\)}&&{(  ) Sim (  ) Não}&\\
    \multicolumn{1}{c|}{\multirow{1}{*}{\textbf{D5}}}&{Temozolomida \(200\) mg/m\(^2\)}&&{(  ) Sim (  ) Não}&\\
    \hline
    \multicolumn{1}{c|}{\multirow{2}{*}{\textbf{Exames}}}&\multicolumn{2}{l|}{Neut (\(>7,5\times10^2\)):}&{Plaq (\(>7,5\times10^4\)):}&{TGO:}\\
    \cline{2-5}
    \multicolumn{1}{c|}{\multirow{2}{*}{{}}}&\multicolumn{2}{l|}{Creat(\(<1,5\) vezes)}&{BT(\(<1,5\) vezes):}&{TGP:}
    \\
    \hline
\end{longtable}
\textbf{Intervalo de 14 dias.}
\begin{longtable}{p{5cm}|p{5cm}|p{5cm}}
    \hline
    \textbf{Exames (data):}&{Neut (\(>7,5\times10^2\)):}&{Plaq (\(>7,5\times10^4\)):}
    \\
    \hline
\end{longtable}
\textbf{Intervalo de 14 dias.}
\end{center}

\begin{center}
\begin{longtable}{p{1cm}p{4cm}|p{1cm}|p{5cm}|p{3cm}}
	\hline
	\multicolumn{5}{c}{\textbf{CICLO 5}}\\
\hline
    \multicolumn{1}{c|}{\multirow{1}{*}{\textbf{Dia}}}&{Dose}&{Data}&{Administrado}&{Rubrica} \\
    \hline
    \multicolumn{1}{c|}{\multirow{1}{*}{\textbf{D1}}}&{Temozolomida \(200\) mg/m\(^2\)}&&{(  ) Sim (  ) Não}&\\
    \multicolumn{1}{c|}{\multirow{1}{*}{\textbf{D2}}}&{Temozolomida \(200\) mg/m\(^2\)}&&{(  ) Sim (  ) Não}&\\
    \multicolumn{1}{c|}{\multirow{1}{*}{\textbf{D3}}}&{Temozolomida \(200\) mg/m\(^2\)}&&{(  ) Sim (  ) Não}&\\
    \multicolumn{1}{c|}{\multirow{1}{*}{\textbf{D4}}}&{Temozolomida \(200\) mg/m\(^2\)}&&{(  ) Sim (  ) Não}&\\
    \multicolumn{1}{c|}{\multirow{1}{*}{\textbf{D5}}}&{Temozolomida \(200\) mg/m\(^2\)}&&{(  ) Sim (  ) Não}&\\
    \hline
    \multicolumn{1}{c|}{\multirow{2}{*}{\textbf{Exames}}}&\multicolumn{2}{l|}{Neut (\(>7,5\times10^2\)):}&{Plaq (\(>7,5\times10^4\)):}&{TGO:}\\
    \cline{2-5}
    \multicolumn{1}{c|}{\multirow{2}{*}{{}}}&\multicolumn{2}{l|}{Creat(\(<1,5\) vezes)}&{BT(\(<1,5\) vezes):}&{TGP:}
    \\
    \hline
\end{longtable}
\textbf{Intervalo de 14 dias.}
\begin{longtable}{p{5cm}|p{5cm}|p{5cm}}
    \hline
    \textbf{Exames (data):}&{Neut (\(>7,5\times10^2\)):}&{Plaq (\(>7,5\times10^4\)):}
    \\
    \hline
\end{longtable}
\textbf{Intervalo de 14 dias.}
\end{center}
\clearpage
\begin{center}
\begin{longtable}{p{1cm}p{4cm}|p{1cm}|p{5cm}|p{3cm}}
	\hline
	\multicolumn{5}{c}{\textbf{CICLO 6}}\\
\hline
    \multicolumn{1}{c|}{\multirow{1}{*}{\textbf{Dia}}}&{Dose}&{Data}&{Administrado}&{Rubrica} \\
    \hline
    \multicolumn{1}{c|}{\multirow{1}{*}{\textbf{D1}}}&{Temozolomida \(200\) mg/m\(^2\)}&&{(  ) Sim (  ) Não}&\\
    \multicolumn{1}{c|}{\multirow{1}{*}{\textbf{D2}}}&{Temozolomida \(200\) mg/m\(^2\)}&&{(  ) Sim (  ) Não}&\\
    \multicolumn{1}{c|}{\multirow{1}{*}{\textbf{D3}}}&{Temozolomida \(200\) mg/m\(^2\)}&&{(  ) Sim (  ) Não}&\\
    \multicolumn{1}{c|}{\multirow{1}{*}{\textbf{D4}}}&{Temozolomida \(200\) mg/m\(^2\)}&&{(  ) Sim (  ) Não}&\\
    \multicolumn{1}{c|}{\multirow{1}{*}{\textbf{D5}}}&{Temozolomida \(200\) mg/m\(^2\)}&&{(  ) Sim (  ) Não}&\\
    \hline
    \multicolumn{1}{c|}{\multirow{2}{*}{\textbf{Exames}}}&\multicolumn{2}{l|}{Neut (\(>7,5\times10^2\)):}&{Plaq (\(>7,5\times10^4\)):}&{TGO:}\\
    \cline{2-5}
    \multicolumn{1}{c|}{\multirow{2}{*}{{}}}&\multicolumn{2}{l|}{Creat(\(<1,5\) vezes)}&{BT(\(<1,5\) vezes):}&{TGP:}
    \\
    \hline
\end{longtable}
\textbf{Intervalo de 14 dias.}
\begin{longtable}{p{5cm}|p{5cm}|p{5cm}}
    \hline
    \textbf{Exames (data):}&{Neut (\(>7,5\times10^2\)):}&{Plaq (\(>7,5\times10^4\)):}
    \\
    \hline
\end{longtable}
\textbf{Intervalo de 14 dias.}
\end{center}

\begin{center}
\begin{longtable}{p{1cm}p{4cm}|p{1cm}|p{5cm}|p{3cm}}
	\hline
	\multicolumn{5}{c}{\textbf{CICLO 7}}\\
\hline
    \multicolumn{1}{c|}{\multirow{1}{*}{\textbf{Dia}}}&{Dose}&{Data}&{Administrado}&{Rubrica} \\
    \hline
    \multicolumn{1}{c|}{\multirow{1}{*}{\textbf{D1}}}&{Temozolomida \(200\) mg/m\(^2\)}&&{(  ) Sim (  ) Não}&\\
    \multicolumn{1}{c|}{\multirow{1}{*}{\textbf{D2}}}&{Temozolomida \(200\) mg/m\(^2\)}&&{(  ) Sim (  ) Não}&\\
    \multicolumn{1}{c|}{\multirow{1}{*}{\textbf{D3}}}&{Temozolomida \(200\) mg/m\(^2\)}&&{(  ) Sim (  ) Não}&\\
    \multicolumn{1}{c|}{\multirow{1}{*}{\textbf{D4}}}&{Temozolomida \(200\) mg/m\(^2\)}&&{(  ) Sim (  ) Não}&\\
    \multicolumn{1}{c|}{\multirow{1}{*}{\textbf{D5}}}&{Temozolomida \(200\) mg/m\(^2\)}&&{(  ) Sim (  ) Não}&\\
    \hline
    \multicolumn{1}{c|}{\multirow{2}{*}{\textbf{Exames}}}&\multicolumn{2}{l|}{Neut (\(>7,5\times10^2\)):}&{Plaq (\(>7,5\times10^4\)):}&{TGO:}\\
    \cline{2-5}
    \multicolumn{1}{c|}{\multirow{2}{*}{{}}}&\multicolumn{2}{l|}{Creat(\(<1,5\) vezes)}&{BT(\(<1,5\) vezes):}&{TGP:}
    \\
    \hline
\end{longtable}
\textbf{Intervalo de 14 dias.}
\begin{longtable}{p{5cm}|p{5cm}|p{5cm}}
    \hline
    \textbf{Exames (data):}&{Neut (\(>7,5\times10^2\)):}&{Plaq (\(>7,5\times10^4\)):}
    \\
    \hline
\end{longtable}
\textbf{Intervalo de 14 dias.}
\end{center}

\pagebreak
\vspace{5mm}
\entrywithlabel[.96\hsize]{\textbf{Nome}}\hfill \\

\entrywithlabel[.45\hsize]{\textbf{Peso}}\hfill  \entrywithlabel[.45\hsize]{\textbf{Estatura}}

\begin{center}
\begin{longtable}{p{1cm}p{4cm}|p{1cm}|p{5cm}|p{3cm}}
	\hline
	\multicolumn{5}{c}{\textbf{CICLO 8}}\\
\hline
    \multicolumn{1}{c|}{\multirow{1}{*}{\textbf{Dia}}}&{Dose}&{Data}&{Administrado}&{Rubrica} \\
    \hline
    \multicolumn{1}{c|}{\multirow{1}{*}{\textbf{D1}}}&{Temozolomida \(200\) mg/m\(^2\)}&&{(  ) Sim (  ) Não}&\\
    \multicolumn{1}{c|}{\multirow{1}{*}{\textbf{D2}}}&{Temozolomida \(200\) mg/m\(^2\)}&&{(  ) Sim (  ) Não}&\\
    \multicolumn{1}{c|}{\multirow{1}{*}{\textbf{D3}}}&{Temozolomida \(200\) mg/m\(^2\)}&&{(  ) Sim (  ) Não}&\\
    \multicolumn{1}{c|}{\multirow{1}{*}{\textbf{D4}}}&{Temozolomida \(200\) mg/m\(^2\)}&&{(  ) Sim (  ) Não}&\\
    \multicolumn{1}{c|}{\multirow{1}{*}{\textbf{D5}}}&{Temozolomida \(200\) mg/m\(^2\)}&&{(  ) Sim (  ) Não}&\\
    \hline
    \multicolumn{1}{c|}{\multirow{2}{*}{\textbf{Exames}}}&\multicolumn{2}{l|}{Neut (\(>7,5\times10^2\)):}&{Plaq (\(>7,5\times10^4\)):}&{TGO:}\\
    \cline{2-5}
    \multicolumn{1}{c|}{\multirow{2}{*}{{}}}&\multicolumn{2}{l|}{Creat(\(<1,5\) vezes)}&{BT(\(<1,5\) vezes):}&{TGP:}
    \\
    \hline
\end{longtable}
\textbf{Intervalo de 14 dias.}
\begin{longtable}{p{5cm}|p{5cm}|p{5cm}}
    \hline
    \textbf{Exames (data):}&{Neut (\(>7,5\times10^2\)):}&{Plaq (\(>7,5\times10^4\)):}
    \\
    \hline
\end{longtable}
\textbf{Intervalo de 14 dias.}
\end{center}

\begin{center}
\begin{longtable}{p{1cm}p{4cm}|p{1cm}|p{5cm}|p{3cm}}
	\hline
	\multicolumn{5}{c}{\textbf{CICLO 9}}\\
\hline
    \multicolumn{1}{c|}{\multirow{1}{*}{\textbf{Dia}}}&{Dose}&{Data}&{Administrado}&{Rubrica} \\
    \hline
    \multicolumn{1}{c|}{\multirow{1}{*}{\textbf{D1}}}&{Temozolomida \(200\) mg/m\(^2\)}&&{(  ) Sim (  ) Não}&\\
    \multicolumn{1}{c|}{\multirow{1}{*}{\textbf{D2}}}&{Temozolomida \(200\) mg/m\(^2\)}&&{(  ) Sim (  ) Não}&\\
    \multicolumn{1}{c|}{\multirow{1}{*}{\textbf{D3}}}&{Temozolomida \(200\) mg/m\(^2\)}&&{(  ) Sim (  ) Não}&\\
    \multicolumn{1}{c|}{\multirow{1}{*}{\textbf{D4}}}&{Temozolomida \(200\) mg/m\(^2\)}&&{(  ) Sim (  ) Não}&\\
    \multicolumn{1}{c|}{\multirow{1}{*}{\textbf{D5}}}&{Temozolomida \(200\) mg/m\(^2\)}&&{(  ) Sim (  ) Não}&\\
    \hline
    \multicolumn{1}{c|}{\multirow{2}{*}{\textbf{Exames}}}&\multicolumn{2}{l|}{Neut (\(>7,5\times10^2\)):}&{Plaq (\(>7,5\times10^4\)):}&{TGO:}\\
    \cline{2-5}
    \multicolumn{1}{c|}{\multirow{2}{*}{{}}}&\multicolumn{2}{l|}{Creat(\(<1,5\) vezes)}&{BT(\(<1,5\) vezes):}&{TGP:}
    \\
    \hline
\end{longtable}
\textbf{Intervalo de 14 dias.}
\begin{longtable}{p{5cm}|p{5cm}|p{5cm}}
    \hline
    \textbf{Exames (data):}&{Neut (\(>7,5\times10^2\)):}&{Plaq (\(>7,5\times10^4\)):}
    \\
    \hline
\end{longtable}
\textbf{Intervalo de 14 dias.}
\end{center}
\clearpage
\begin{center}
\begin{longtable}{p{1cm}p{4cm}|p{1cm}|p{5cm}|p{3cm}}
	\hline
	\multicolumn{5}{c}{\textbf{CICLO 10}}\\
\hline
    \multicolumn{1}{c|}{\multirow{1}{*}{\textbf{Dia}}}&{Dose}&{Data}&{Administrado}&{Rubrica} \\
    \hline
    \multicolumn{1}{c|}{\multirow{1}{*}{\textbf{D1}}}&{Temozolomida \(200\) mg/m\(^2\)}&&{(  ) Sim (  ) Não}&\\
    \multicolumn{1}{c|}{\multirow{1}{*}{\textbf{D2}}}&{Temozolomida \(200\) mg/m\(^2\)}&&{(  ) Sim (  ) Não}&\\
    \multicolumn{1}{c|}{\multirow{1}{*}{\textbf{D3}}}&{Temozolomida \(200\) mg/m\(^2\)}&&{(  ) Sim (  ) Não}&\\
    \multicolumn{1}{c|}{\multirow{1}{*}{\textbf{D4}}}&{Temozolomida \(200\) mg/m\(^2\)}&&{(  ) Sim (  ) Não}&\\
    \multicolumn{1}{c|}{\multirow{1}{*}{\textbf{D5}}}&{Temozolomida \(200\) mg/m\(^2\)}&&{(  ) Sim (  ) Não}&\\
    \hline
    \multicolumn{1}{c|}{\multirow{2}{*}{\textbf{Exames}}}&\multicolumn{2}{l|}{Neut (\(>7,5\times10^2\)):}&{Plaq (\(>7,5\times10^4\)):}&{TGO:}\\
    \cline{2-5}
    \multicolumn{1}{c|}{\multirow{2}{*}{{}}}&\multicolumn{2}{l|}{Creat(\(<1,5\) vezes)}&{BT(\(<1,5\) vezes):}&{TGP:}
    \\
    \hline
\end{longtable}
\textbf{Intervalo de 14 dias.}
\begin{longtable}{p{5cm}|p{5cm}|p{5cm}}
    \hline
    \textbf{Exames (data):}&{Neut (\(>7,5\times10^2\)):}&{Plaq (\(>7,5\times10^4\)):}
    \\
    \hline
\end{longtable}
 
\textbf{FIM DE PROTOCOLO}

\end{center}
\subsection{Modificações de dose:} 
Se atraso maior que 7 dias por toxicidade, reduzir os ciclos subsequentes para 150 mg/m\textsuperscript{2}/dia.

\textbf{Avaliação:} imagem a cada 3 ciclos (3 meses), se progressão, interromper protocolo.

\textbf{APRESENTAÇÕES DE TEMOZOLOMIDA NO HIAS:} cápsulas de 100mg

\textbf{ADVERTÊNCIA:} SMZ+TMP não deve ser administrada juntamente com a temozolomida!

\textbf{ATENÇÃO:} o objetivo deste protocolo é ADIAR O USO DA RT (se não tiver sido feita) até a criança atingir uma idade onde os efeitos adversos da radiação sejam reduzidos, ou controlar doença recidivada após a RT. A principal resposta deste protocolo é ESTABILIZAÇÃO DA DOENÇA. Logo, é inadequado iniciar este esquema de QT em crianças em regime de internação prolongada, dependentes de cuidados hospitalares, visando "melhorar" sua condição clínica. Igualmente, é inadequado iniciar este protocolo em crianças com risco de complicações graves, como naquelas que têm sequelas importantes e muito limitantes.

\textbf{ATENÇÃO:} este protocolo é \textit{off-label} (não padronizado) e não tem eficácia comprovada quando comparado com tratamento padrão sem QT. Dessa forma, é inadequado iniciar este protocolo em crianças com risco de complicações graves, como naquelas que têm sequelas importantes e muito limitantes.
\cleardoublepage

\section{PNET - ALTO RISCO -- Adaptado dos ensaios COG-A99701 e ACNS0332 - protocolo \textit{off-label}}
{\let\thefootnote\relax\footnotetext{Versão Janeiro/2015}}
\textbf{Racional:} no estudo piloto do COG\footnote{Jakacki \textit{et al}, 2012}, a QT durante a RT possibilitou a melhora da sobrevida de pacientes com tumor residual ou metástase. O COG está testando agora essa estratégia no ensaio fase III ACNS0332. Não existe tratamento quimioterápico padrão para estes pacientes, porém os resultados do COG são os melhores publicados até o momento. Reforço deve ser feito também sobre as metástases espinhais, até 45Gy dose total acima do cone medular e 50,4 Gy abaixo dele. Reavaliar com imagens 4 semanas após terminar RT.

\textbf{Elegível:} meduloblastoma (fossa posterior) e PNET, com mais de 1,5cm\textsuperscript{2} de tumor residual (RNM de controle até 21 dias pós-op, preferido 72h após); e/ou com metástases (RNM de neuro-eixo e PL/MO); incluir tumores com anaplasia ou positivos para N-MYC/C-MYC. Excluir pacientes com marcador INI-1 negativo ou menores de 3 anos. Tratamento precisa iniciar até 31 dias após cirurgia. NÃO INICIAR ESTE PROTOCOLO EM CRIANÇAS GRAVEMENTE ENFERMAS.

\textbf{Alternativa:} não existe alternativa de QT amplamente aceita para este grupo de pacientes. Invariavelmente, os pacientes com doença metastática e fatores de risco molecular têm prognóstico insatisfatório, com reduzida sobrevida livre de progressão prolongada. 

\vspace{5mm}
\entrywithlabel[.96\hsize]{\textbf{Nome}}\hfill \\

\entrywithlabel[.45\hsize]{\textbf{Peso}}\hfill  \entrywithlabel[.45\hsize]{\textbf{Estatura}}

\subsection{Radioquimioterapia: 7 semanas (43 dias)}

\begin{center}
\begin{longtable}{p{1cm}p{2cm}|p{2cm}|p{1cm}|p{4cm}|p{3cm}}
	\hline
	\multicolumn{6}{c}{\textbf{SEMANA 1}}\\
\hline
    \multicolumn{1}{c|}{\multirow{2}{*}{\textbf{Dia}}}&\multicolumn{2}{c|}{Dose RT}&\multicolumn{1}{c|}{\multirow{2}{*}{Data}}&\multicolumn{1}{c|}{\multirow{2}{*}{Quimioterapia}}&\multicolumn{1}{c}{\multirow{2}{*}{Rubrica}} \\
    \cline{2-3}
    \multicolumn{1}{c|}{\multirow{1}{*}{}}&{Neuro-eixo}&{Fossa poster}&& \\
	\hline
	\multicolumn{1}{c|}{\multirow{1}{*}{\textbf{D1}}}&\multicolumn{1}{c|}{\(1,8\) Gy}&&&{Vincristina \(1,5\) mg/m\(^2\)}&\\
	\multicolumn{1}{c|}{\multirow{1}{*}{\textbf{}}}&\multicolumn{1}{c|}{}&&&{Carboplatina 35mg/m\textsuperscript{2}}&\\
    \multicolumn{1}{c|}{\multirow{1}{*}{\textbf{D2}}}&\multicolumn{1}{c|}{\(1,8\) Gy}&&&{Carboplatina 35mg/m\textsuperscript{2}}&\\
    \multicolumn{1}{c|}{\multirow{1}{*}{\textbf{D3}}}&\multicolumn{1}{c|}{\(1,8\) Gy}&&&{Carboplatina 35mg/m\textsuperscript{2}}&\\
    \multicolumn{1}{c|}{\multirow{1}{*}{\textbf{D4}}}&\multicolumn{1}{c|}{\(1,8\) Gy}&&&{Carboplatina 35mg/m\textsuperscript{2}}&\\
    \multicolumn{1}{c|}{\multirow{1}{*}{\textbf{D5}}}&\multicolumn{1}{c|}{\(1,8\) Gy}&&&{Carboplatina 35mg/m\textsuperscript{2}}&\\
    \hline
    \multicolumn{1}{c|}{\multirow{2}{*}{\textbf{Exames}}}&\multicolumn{2}{l|}{Neut (\(>7,5\times10^2\)):}&\multicolumn{2}{l|}{Plaq (\(>7,5\times10^4\)):}&\\
    \cline{2-6}
    \multicolumn{1}{c|}{\multirow{2}{*}{{}}}&\multicolumn{2}{l|}{BT(<1,9mg/dl):}&\multicolumn{2}{l|}{BD(\(<1,5\)mg/dl):}&\\
    \hline
\end{longtable}
\clearpage
\begin{longtable}{p{1cm}p{2cm}|p{2cm}|p{1cm}|p{4cm}|p{3cm}}
	\hline
	\multicolumn{6}{c}{\textbf{SEMANA 2}}\\
\hline
    \multicolumn{1}{c|}{\multirow{2}{*}{\textbf{Dia}}}&\multicolumn{2}{c|}{Dose RT}&\multicolumn{1}{c|}{\multirow{2}{*}{Data}}&\multicolumn{1}{c|}{\multirow{2}{*}{Quimioterapia}}&\multicolumn{1}{c}{\multirow{2}{*}{Rubrica}} \\
    \cline{2-3}
    \multicolumn{1}{c|}{\multirow{1}{*}{}}&{Neuro-eixo}&{Fossa poster}&& \\
	\hline
	\multicolumn{1}{c|}{\multirow{1}{*}{\textbf{D8}}}&\multicolumn{1}{c|}{\(1,8\) Gy}&&&{Vincristina \(1,5\) mg/m\(^2\)}&\\
	\multicolumn{1}{c|}{\multirow{1}{*}{\textbf{}}}&\multicolumn{1}{c|}{}&&&{Carboplatina 35mg/m\textsuperscript{2}}&\\
    \multicolumn{1}{c|}{\multirow{1}{*}{\textbf{D9}}}&\multicolumn{1}{c|}{\(1,8\) Gy}&&&{Carboplatina 35mg/m\textsuperscript{2}}&\\
    \multicolumn{1}{c|}{\multirow{1}{*}{\textbf{D10}}}&\multicolumn{1}{c|}{\(1,8\) Gy}&&&{Carboplatina 35mg/m\textsuperscript{2}}&\\
    \multicolumn{1}{c|}{\multirow{1}{*}{\textbf{D11}}}&\multicolumn{1}{c|}{\(1,8\) Gy}&&&{Carboplatina 35mg/m\textsuperscript{2}}&\\
    \multicolumn{1}{c|}{\multirow{1}{*}{\textbf{D12}}}&\multicolumn{1}{c|}{\(1,8\) Gy}&&&{Carboplatina 35mg/m\textsuperscript{2}}&\\
    \hline
    \multicolumn{1}{c|}{\multirow{2}{*}{\textbf{Exames}}}&\multicolumn{2}{l|}{Neut (\(>7,5\times10^2\)):}&\multicolumn{2}{l|}{Plaq (\(>7,5\times10^4\)):}&\\
    \cline{2-6}
    \multicolumn{1}{c|}{\multirow{2}{*}{{}}}&\multicolumn{2}{l|}{BT(<1,9mg/dl):}&\multicolumn{2}{l|}{BD(\(<1,5\)mg/dl):}&\\
    \hline
\end{longtable}

\begin{longtable}{p{1cm}p{2cm}|p{2cm}|p{1cm}|p{4cm}|p{3cm}}
	\hline
	\multicolumn{6}{c}{\textbf{SEMANA 3}}\\
\hline
    \multicolumn{1}{c|}{\multirow{2}{*}{\textbf{Dia}}}&\multicolumn{2}{c|}{Dose RT}&\multicolumn{1}{c|}{\multirow{2}{*}{Data}}&\multicolumn{1}{c|}{\multirow{2}{*}{Quimioterapia}}&\multicolumn{1}{c}{\multirow{2}{*}{Rubrica}} \\
    \cline{2-3}
    \multicolumn{1}{c|}{\multirow{1}{*}{}}&{Neuro-eixo}&{Fossa poster}&& \\
	\hline
	\multicolumn{1}{c|}{\multirow{1}{*}{\textbf{D15}}}&\multicolumn{1}{c|}{\(1,8\) Gy}&&&{Vincristina \(1,5\) mg/m\(^2\)}&\\
	\multicolumn{1}{c|}{\multirow{1}{*}{\textbf{}}}&\multicolumn{1}{c|}{}&&&{Carboplatina 35mg/m\textsuperscript{2}}&\\
    \multicolumn{1}{c|}{\multirow{1}{*}{\textbf{D16}}}&\multicolumn{1}{c|}{\(1,8\) Gy}&&&{Carboplatina 35mg/m\textsuperscript{2}}&\\
    \multicolumn{1}{c|}{\multirow{1}{*}{\textbf{D17}}}&\multicolumn{1}{c|}{\(1,8\) Gy}&&&{Carboplatina 35mg/m\textsuperscript{2}}&\\
    \multicolumn{1}{c|}{\multirow{1}{*}{\textbf{D18}}}&\multicolumn{1}{c|}{\(1,8\) Gy}&&&{Carboplatina 35mg/m\textsuperscript{2}}&\\
    \multicolumn{1}{c|}{\multirow{1}{*}{\textbf{D19}}}&\multicolumn{1}{c|}{\(1,8\) Gy}&&&{Carboplatina 35mg/m\textsuperscript{2}}&\\
    \hline
    \multicolumn{1}{c|}{\multirow{2}{*}{\textbf{Exames}}}&\multicolumn{2}{l|}{Neut (\(>7,5\times10^2\)):}&\multicolumn{2}{l|}{Plaq (\(>7,5\times10^4\)):}&\\
    \cline{2-6}
    \multicolumn{1}{c|}{\multirow{2}{*}{{}}}&\multicolumn{2}{l|}{BT(<1,9mg/dl):}&\multicolumn{2}{l|}{BD(\(<1,5\)mg/dl):}&
    \\
    \hline
\end{longtable}
\clearpage
\begin{longtable}{p{1cm}p{2cm}|p{2cm}|p{1cm}|p{4cm}|p{3cm}}
	\hline
	\multicolumn{6}{c}{\textbf{SEMANA 4}}\\
\hline
    \multicolumn{1}{c|}{\multirow{2}{*}{\textbf{Dia}}}&\multicolumn{2}{c|}{Dose RT}&\multicolumn{1}{c|}{\multirow{2}{*}{Data}}&\multicolumn{1}{c|}{\multirow{2}{*}{Quimioterapia}}&\multicolumn{1}{c}{\multirow{2}{*}{Rubrica}} \\
    \cline{2-3}
    \multicolumn{1}{c|}{\multirow{1}{*}{}}&{Neuro-eixo}&{Fossa poster}&& \\
	\hline
	\multicolumn{1}{c|}{\multirow{1}{*}{\textbf{D22}}}&\multicolumn{1}{c|}{\(1,8\) Gy}&&&{Vincristina \(1,5\) mg/m\(^2\)}&\\
	\multicolumn{1}{c|}{\multirow{1}{*}{\textbf{}}}&\multicolumn{1}{c|}{}&&&{Carboplatina 35mg/m\textsuperscript{2}}&\\
    \multicolumn{1}{c|}{\multirow{1}{*}{\textbf{D23}}}&\multicolumn{1}{c|}{\(1,8\) Gy}&&&{Carboplatina 35mg/m\textsuperscript{2}}&\\
    \multicolumn{1}{c|}{\multirow{1}{*}{\textbf{D24}}}&\multicolumn{1}{c|}{\(1,8\) Gy}&&&{Carboplatina 35mg/m\textsuperscript{2}}&\\
    \multicolumn{1}{c|}{\multirow{1}{*}{\textbf{D25}}}&\multicolumn{1}{c|}{\(1,8\) Gy}&&&{Carboplatina 35mg/m\textsuperscript{2}}&\\
    \multicolumn{1}{c|}{\multirow{1}{*}{\textbf{D26}}}&\multicolumn{1}{c|}{\(1,8\) Gy}&&&{Carboplatina 35mg/m\textsuperscript{2}}&\\
    \hline
    \multicolumn{1}{c|}{\multirow{2}{*}{\textbf{Exames}}}&\multicolumn{2}{l|}{Neut (\(>7,5\times10^2\)):}&\multicolumn{2}{l|}{Plaq (\(>7,5\times10^4\)):}&\\
    \cline{2-6}
    \multicolumn{1}{c|}{\multirow{2}{*}{{}}}&\multicolumn{2}{l|}{BT(<1,9mg/dl):}&\multicolumn{2}{l|}{BD(\(<1,5\)mg/dl):}&
    \\
    \hline
\end{longtable}
\begin{longtable}{p{1cm}p{2cm}|p{2cm}|p{1cm}|p{4cm}|p{3cm}}
	\hline
	\multicolumn{6}{c}{\textbf{SEMANA 5}}\\
\hline
    \multicolumn{1}{c|}{\multirow{2}{*}{\textbf{Dia}}}&\multicolumn{2}{c|}{Dose RT}&\multicolumn{1}{c|}{\multirow{2}{*}{Data}}&\multicolumn{1}{c|}{\multirow{2}{*}{Quimioterapia}}&\multicolumn{1}{c}{\multirow{2}{*}{Rubrica}} \\
    \cline{2-3}
    \multicolumn{1}{c|}{\multirow{1}{*}{}}&{Neuro-eixo}&{Fossa poster}&& \\
	\hline
	\multicolumn{1}{c|}{\multirow{1}{*}{\textbf{D29}}}&\multicolumn{1}{c|}{\(1,8\) Gy}&&&{Vincristina \(1,5\) mg/m\(^2\)}&\\
	\multicolumn{1}{c|}{\multirow{1}{*}{\textbf{}}}&\multicolumn{1}{c|}{}&&&{Carboplatina 35mg/m\textsuperscript{2}}&\\
    \multicolumn{1}{c|}{\multirow{1}{*}{\textbf{D30}}}&\multicolumn{1}{c|}{\(1,8\) Gy}&&&{Carboplatina 35mg/m\textsuperscript{2}}&\\
    \multicolumn{1}{c|}{\multirow{1}{*}{\textbf{D31}}}&\multicolumn{1}{c|}{\(1,8\) Gy}&&&{Carboplatina 35mg/m\textsuperscript{2}}&\\
    \multicolumn{1}{c|}{\multirow{1}{*}{\textbf{D32}}}&\multicolumn{1}{c|}{\(1,8\) Gy}&&&{Carboplatina 35mg/m\textsuperscript{2}}&\\
    \multicolumn{1}{c|}{\multirow{1}{*}{\textbf{D33}}}&\multicolumn{1}{c|}{\(1,8\) Gy}&&&{Carboplatina 35mg/m\textsuperscript{2}}&\\
    \hline
    \multicolumn{1}{c|}{\multirow{2}{*}{\textbf{Exames}}}&\multicolumn{2}{l|}{Neut (\(>7,5\times10^2\)):}&\multicolumn{2}{l|}{Plaq (\(>7,5\times10^4\)):}&\\
    \cline{2-6}
    \multicolumn{1}{c|}{\multirow{2}{*}{{}}}&\multicolumn{2}{l|}{BT(<1,9mg/dl):}&\multicolumn{2}{l|}{BD(\(<1,5\)mg/dl):}&
    \\
    \hline
\end{longtable}
\clearpage
\begin{longtable}{p{1cm}p{2cm}|p{2cm}|p{1cm}|p{4cm}|p{3cm}}
	\hline
	\multicolumn{6}{c}{\textbf{SEMANA 6}}\\
\hline
    \multicolumn{1}{c|}{\multirow{2}{*}{\textbf{Dia}}}&\multicolumn{2}{c|}{Dose RT}&\multicolumn{1}{c|}{\multirow{2}{*}{Data}}&\multicolumn{1}{c|}{\multirow{2}{*}{Quimioterapia}}&\multicolumn{1}{c}{\multirow{2}{*}{Rubrica}} \\
    \cline{2-3}
    \multicolumn{1}{c|}{\multirow{1}{*}{}}&{Neuro-eixo}&{Fossa poster}&& \\
	\hline
	\multicolumn{1}{c|}{\multirow{1}{*}{\textbf{D36}}}&\multicolumn{1}{c|}{\(1,8\) Gy}&&&{Vincristina \(1,5\) mg/m\(^2\)}&\\
	\multicolumn{1}{c|}{\multirow{1}{*}{\textbf{}}}&\multicolumn{1}{c|}{}&&&{Carboplatina 35mg/m\textsuperscript{2}}&\\
    \multicolumn{1}{c|}{\multirow{1}{*}{\textbf{D37}}}&\multicolumn{1}{c|}{\(1,8\) Gy}&&&{Carboplatina 35mg/m\textsuperscript{2}}&\\
    \multicolumn{1}{c|}{\multirow{1}{*}{\textbf{D38}}}&\multicolumn{1}{c|}{\(1,8\) Gy}&&&{Carboplatina 35mg/m\textsuperscript{2}}&\\
    \multicolumn{1}{c|}{\multirow{1}{*}{\textbf{D39}}}&\multicolumn{1}{c|}{\(1,8\) Gy}&&&{Carboplatina 35mg/m\textsuperscript{2}}&\\
    \multicolumn{1}{c|}{\multirow{1}{*}{\textbf{D40}}}&\multicolumn{1}{c|}{\(1,8\) Gy}&&&{Carboplatina 35mg/m\textsuperscript{2}}&\\
    \hline
    \multicolumn{1}{c|}{\multirow{2}{*}{\textbf{Exames}}}&\multicolumn{2}{l|}{Neut (\(>7,5\times10^2\)):}&\multicolumn{2}{l|}{Plaq (\(>7,5\times10^4\)):}&\\
    \cline{2-6}
    \multicolumn{1}{c|}{\multirow{2}{*}{{}}}&\multicolumn{2}{l|}{BT(<1,9mg/dl):}&\multicolumn{2}{l|}{BD(\(<1,5\)mg/dl):}&
    \\
    \hline
\end{longtable}
\begin{longtable}{p{1cm}p{2cm}|p{2cm}|p{1cm}|p{4cm}|p{3cm}}
	\hline
	\multicolumn{6}{c}{\textbf{SEMANA 7}}\\
\hline
    \multicolumn{1}{c|}{\multirow{2}{*}{\textbf{Dia}}}&\multicolumn{2}{c|}{Dose RT}&\multicolumn{1}{c|}{\multirow{2}{*}{Data}}&\multicolumn{1}{c|}{\multirow{2}{*}{Quimioterapia}}&\multicolumn{1}{c}{\multirow{2}{*}{Rubrica}} \\
    \cline{2-3}
    \multicolumn{1}{c|}{\multirow{1}{*}{}}&{Neuro-eixo}&{Fossa poster}&& \\
	\hline
	\multicolumn{1}{c|}{\multirow{1}{*}{\textbf{D43}}}&\multicolumn{1}{c|}{\(1,8\) Gy}&&&{Vincristina \(1,5\) mg/m\(^2\)}&\\
    \hline
    \multicolumn{1}{c|}{\multirow{2}{*}{\textbf{Exames}}}&\multicolumn{2}{l|}{Neut (\(>7,5\times10^2\)):}&\multicolumn{2}{l|}{Plaq (\(>7,5\times10^4\)):}&\\
    \cline{2-6}
    \multicolumn{1}{c|}{\multirow{2}{*}{{}}}&\multicolumn{2}{l|}{BT(<1,9mg/dl):}&\multicolumn{2}{l|}{BD(\(<1,5\)mg/dl):}&
    \\
    \hline
\end{longtable}
\textbf{Intervalo de 42 dias}\\
\end{center}

\subsection{Manutenção: 06 ciclos}

\begin{center}
    \begin{longtable}{p{1cm}p{5cm}|p{1cm}|p{4cm}|p{2.5cm}}
    \hline
	\multicolumn{5}{c}{\textbf{CICLO 1}}\\
	\hline
    \multicolumn{1}{c|}{\multirow{1}{*}{\textbf{Dia}}}&{Dose}&{Data}&{Administrado}&{Rubrica} \\
    \hline
    \multicolumn{1}{c|}{\multirow{1}{*}{\textbf{D85}}}&{Ciclofosfamida \(1,0\) g/m\(^2\) EV em 6h}&&{(  ) Sim (  ) Não}&\\
    \multicolumn{1}{c|}{\multirow{1}{*}{\textbf{D86}}}&{Ciclofosfamida \(1,0\) g/m\(^2\) EV em 6h}&&{(  ) Sim (  ) Não}&\\
    \multicolumn{1}{c|}{\multirow{1}{*}{\textbf{}}}&{Vincristina \(1,5\) mg/m\(^2\), max \(2\) mg}&&{(  ) Sim (  ) Não}&\\
    \hline
    \multicolumn{1}{c|}{\multirow{2}{*}{\textbf{Exames}}}&\multicolumn{2}{l|}{Neut(\(>10^3\)):}&{Plaq(\(>10^5\)):}&\\
    \cline{2-5}
    \multicolumn{1}{c|}{\multirow{2}{*}{{}}}&\multicolumn{2}{l|}{ClearCreat:}&{}&{}\\
    \hline
    \\
    \hline
    \multicolumn{1}{c|}{\multirow{1}{*}{\textbf{D92}}}&{Vincristina \(1,5\) mg/m\(^2\), max \(2\) mg}&&{(  ) Sim (  ) Não}&\\
    \hline    
\end{longtable}
\textbf{Intervalo de 21 dias}
    \begin{longtable}{p{1cm}p{5cm}|p{1cm}|p{4cm}|p{2.5cm}}
    \hline
	\multicolumn{5}{c}{\textbf{CICLO 2}}\\
	\hline
    \multicolumn{1}{c|}{\multirow{1}{*}{\textbf{Dia}}}&{Dose}&{Data}&{Administrado}&{Rubrica} \\
    \hline
    \multicolumn{1}{c|}{\multirow{1}{*}{\textbf{D113}}}&{Ciclofosfamida \(1,0\) g/m\(^2\) EV em 6h}&&{(  ) Sim (  ) Não}&\\
    \multicolumn{1}{c|}{\multirow{1}{*}{\textbf{D114}}}&{Ciclofosfamida \(1,0\) g/m\(^2\) EV em 6h}&&{(  ) Sim (  ) Não}&\\
    \multicolumn{1}{c|}{\multirow{1}{*}{\textbf{}}}&{Vincristina \(1,5\) mg/m\(^2\), max \(2\) mg}&&{(  ) Sim (  ) Não}&\\
    \hline
    \multicolumn{1}{c|}{\multirow{2}{*}{\textbf{Exames}}}&\multicolumn{2}{l|}{Neut(\(>10^3\)):}&{Plaq(\(>10^5\)):}&\\
    \cline{2-5}
    \multicolumn{1}{c|}{\multirow{2}{*}{{}}}&\multicolumn{2}{l|}{ClearCreat:}&{}&{}\\
    \hline
    \\
    \hline
    \multicolumn{1}{c|}{\multirow{1}{*}{\textbf{D120}}}&{Vincristina \(1,5\) mg/m\(^2\), max \(2\) mg}&&{(  ) Sim (  ) Não}&\\
    \hline    
\end{longtable}
\textbf{Intervalo de 21 dias}
    \begin{longtable}{p{1cm}p{5cm}|p{1cm}|p{4cm}|p{2.5cm}}
    \hline
	\multicolumn{5}{c}{\textbf{CICLO 3}}\\
	\hline
    \multicolumn{1}{c|}{\multirow{1}{*}{\textbf{Dia}}}&{Dose}&{Data}&{Administrado}&{Rubrica} \\
    \hline
    \multicolumn{1}{c|}{\multirow{1}{*}{\textbf{141}}}&{Ciclofosfamida \(1,0\) g/m\(^2\) EV em 6h}&&{(  ) Sim (  ) Não}&\\
    \multicolumn{1}{c|}{\multirow{1}{*}{\textbf{D142}}}&{Ciclofosfamida \(1,0\) g/m\(^2\) EV em 6h}&&{(  ) Sim (  ) Não}&\\
    \multicolumn{1}{c|}{\multirow{1}{*}{\textbf{}}}&{Vincristina \(1,5\) mg/m\(^2\), max \(2\) mg}&&{(  ) Sim (  ) Não}&\\
    \hline
    \multicolumn{1}{c|}{\multirow{2}{*}{\textbf{Exames}}}&\multicolumn{2}{l|}{Neut(\(>10^3\)):}&{Plaq(\(>10^5\)):}&\\
    \cline{2-5}
    \multicolumn{1}{c|}{\multirow{2}{*}{{}}}&\multicolumn{2}{l|}{ClearCreat:}&{}&{}\\
    \hline
    \\
    \hline
    \multicolumn{1}{c|}{\multirow{1}{*}{\textbf{D148}}}&{Vincristina \(1,5\) mg/m\(^2\), max \(2\) mg}&&{(  ) Sim (  ) Não}&\\
    \hline    
\end{longtable}
\textbf{Intervalo de 21 dias}
\    \begin{longtable}{p{1cm}p{5cm}|p{1cm}|p{4cm}|p{2.5cm}}
    \hline
	\multicolumn{5}{c}{\textbf{CICLO 4}}\\
	\hline
    \multicolumn{1}{c|}{\multirow{1}{*}{\textbf{Dia}}}&{Dose}&{Data}&{Administrado}&{Rubrica} \\
    \hline
    \multicolumn{1}{c|}{\multirow{1}{*}{\textbf{D169}}}&{Ciclofosfamida \(1,0\) g/m\(^2\) EV em 6h}&&{(  ) Sim (  ) Não}&\\
    \multicolumn{1}{c|}{\multirow{1}{*}{\textbf{D170}}}&{Ciclofosfamida \(1,0\) g/m\(^2\) EV em 6h}&&{(  ) Sim (  ) Não}&\\
    \multicolumn{1}{c|}{\multirow{1}{*}{\textbf{}}}&{Vincristina \(1,5\) mg/m\(^2\), max \(2\) mg}&&{(  ) Sim (  ) Não}&\\
    \hline
    \multicolumn{1}{c|}{\multirow{2}{*}{\textbf{Exames}}}&\multicolumn{2}{l|}{Neut(\(>10^3\)):}&{Plaq(\(>10^5\)):}&\\
    \cline{2-5}
    \multicolumn{1}{c|}{\multirow{2}{*}{{}}}&\multicolumn{2}{l|}{ClearCreat:}&{}&{}\\
    \hline\\
    \hline
    \multicolumn{1}{c|}{\multirow{1}{*}{\textbf{D176}}}&{Vincristina \(1,5\) mg/m\(^2\), max \(2\) mg}&&{(  ) Sim (  ) Não}&\\
    \hline    
\end{longtable}
\textbf{Intervalo de 21 dias}
    \begin{longtable}{p{1cm}p{5cm}|p{1cm}|p{4cm}|p{2.5cm}}
    \hline
	\multicolumn{5}{c}{\textbf{CICLO 5}}\\
	\hline
    \multicolumn{1}{c|}{\multirow{1}{*}{\textbf{Dia}}}&{Dose}&{Data}&{Administrado}&{Rubrica} \\
    \hline
    \multicolumn{1}{c|}{\multirow{1}{*}{\textbf{D197}}}&{Ciclofosfamida \(1,0\) g/m\(^2\) EV em 6h}&&{(  ) Sim (  ) Não}&\\
    \multicolumn{1}{c|}{\multirow{1}{*}{\textbf{D198}}}&{Ciclofosfamida \(1,0\) g/m\(^2\) EV em 6h}&&{(  ) Sim (  ) Não}&\\
    \multicolumn{1}{c|}{\multirow{1}{*}{\textbf{}}}&{Vincristina \(1,5\) mg/m\(^2\), max \(2\) mg}&&{(  ) Sim (  ) Não}&\\
    \hline
    \multicolumn{1}{c|}{\multirow{2}{*}{\textbf{Exames}}}&\multicolumn{2}{l|}{Neut(\(>10^3\)):}&{Plaq(\(>10^5\)):}&\\
    \cline{2-5}
    \multicolumn{1}{c|}{\multirow{2}{*}{{}}}&\multicolumn{2}{l|}{ClearCreat:}&{}&{}\\
    \hline
    \\
    \hline
    \multicolumn{1}{c|}{\multirow{1}{*}{\textbf{D204}}}&{Vincristina \(1,5\) mg/m\(^2\), max \(2\) mg}&&{(  ) Sim (  ) Não}&\\
    \hline    
\end{longtable}
\textbf{Intervalo de 21 dias}
\    \begin{longtable}{p{1cm}p{5cm}|p{1cm}|p{4cm}|p{2.5cm}}
    \hline
	\multicolumn{5}{c}{\textbf{CICLO 6}}\\
	\hline
    \multicolumn{1}{c|}{\multirow{1}{*}{\textbf{Dia}}}&{Dose}&{Data}&{Administrado}&{Rubrica} \\
    \hline
    \multicolumn{1}{c|}{\multirow{1}{*}{\textbf{D225}}}&{Ciclofosfamida \(1,0\) g/m\(^2\) EV em 6h}&&{(  ) Sim (  ) Não}&\\
    \multicolumn{1}{c|}{\multirow{1}{*}{\textbf{D226}}}&{Ciclofosfamida \(1,0\) g/m\(^2\) EV em 6h}&&{(  ) Sim (  ) Não}&\\
    \multicolumn{1}{c|}{\multirow{1}{*}{\textbf{}}}&{Vincristina \(1,5\) mg/m\(^2\), max \(2\) mg}&&{(  ) Sim (  ) Não}&\\
    \hline
    \multicolumn{1}{c|}{\multirow{2}{*}{\textbf{Exames}}}&\multicolumn{2}{l|}{Neut(\(>10^3\)):}&{Plaq(\(>10^5\)):}&\\
    \cline{2-5}
    \multicolumn{1}{c|}{\multirow{2}{*}{{}}}&\multicolumn{2}{l|}{ClearCreat:}&{}&{}\\
    \hline
    \\
    \hline
    \multicolumn{1}{c|}{\multirow{1}{*}{\textbf{D232}}}&{Vincristina \(1,5\) mg/m\(^2\), max \(2\) mg}&&{(  ) Sim (  ) Não}&\\
    \hline    
\end{longtable}
\textbf{FIM DE PROTOCOLO}

\end{center}
\subsection{Modificações de dose:} 
Se tiver que adiar a CTX por neutropenia, reduzir em 25\% a dose, mesmo após recuperação.Toxicidade grau 3-4 pela VCR, suspender dose seguinte. Reiniciar com dose normal. Recorrência: reduzir dose.

\textbf{Avaliação:} imagem a cada 3 ciclos (3 meses), se progressão, interromper protocolo.

\textbf{ATENÇÃO:} este protocolo é \textit{off-label} (não padronizado) e não tem eficácia comprovada. Dessa forma, é inadequado iniciar este protocolo em crianças com risco de complicações graves, como naquelas que têm sequelas importantes e muito limitantes.

\cleardoublepage

\section{GLIOMA DE ALTO GRAU E DIPG -- Adaptado do ensaio ACNS0423 - protocolo \textit{off-label}}
{\let\thefootnote\relax\footnotetext{Versão Janeiro/2017}}
\textbf{Racional:} no estudo do COG, a temozolomida (TMZ) associada à lomustina mostraram superioridade em relação ao controle histórico (ACNS0126)\footnote{Jakacki \textit{et al}, 2016}. Deve-se levar em conta a disponibilidade e custo do esquema. Não foi realizada comparação com os dados do estudo CCG-945.

\textbf{Elegível:} extensão da ressecção cirúrgica (RNM de controle até 21 dias pós-op, preferido 72h após); não é necessário pesquisar metástases de rotina (RNM de neuro-eixo e PL/MO). Tratamento precisa iniciar até 42 dias após cirurgia. Inclui glioblastoma multiforme, astrocitoma anaplásico, oligodendroglioma anaplásico, gliossarcoma, gliomas difusos de linha média H3K27M+ (DIPG). Pacientes com mais de 3 anos de idade. Somente iniciar o protocolo após a assinatura do TERMO DE CONSENTIMENTO INFORMADO, conforme acordado. NÃO INICIAR ESTE PROTOCOLO EM CRIANÇAS GRAVEMENTE ENFERMAS.

\textbf{Alternativa:} não existe esquema de QT amplamente aceito para tratar crianças com gliomas de alto grau, incluindo gliomas pontinos intrínsecos difusos (DIPG). O tratamento padrão é RT craniana. Os pacientes com mais de 3 anos têm um prognóstico insatisfatório e sobrevida mediana de pouco mais de 12 meses.

\entrywithlabel[.96\hsize]{\textbf{Nome}}\hfill \\

\entrywithlabel[.45\hsize]{\textbf{Peso}}\hfill  \entrywithlabel[.45\hsize]{\textbf{Estatura}}\\

\subsection{Indução: 6 semanas (radioquimioterapia)}

\renewcommand{\arraystretch}{1.5}

\begin{center}
\begin{longtable}{p{1cm}c|c|c|c|c|c}
	\hline
\multicolumn{1}{c|}{\multirow{1}{*}{\textbf{Semana}}}&{Dose}&{Data}&{Neut}&{Plaq}&{Administrado}&{Rubrica} \\
    \hline
    \multicolumn{1}{c|}{\multirow{1}{*}{\textbf{1}}}&{Temozolomida 90mg/m\textsuperscript{2}/dia Seg-Sex}&{}&&&{(  ) Sim (  ) Não}&\\
    \cline{3-5}
    \multicolumn{1}{c|}{\multirow{1}{*}{{\textbf{2}}}}&{Temozolomida 90mg/m\textsuperscript{2}/dia Seg-Sex}&{}&&&{(  ) Sim (  ) Não}&\\
    \cline{3-5}
    \multicolumn{1}{c|}{\multirow{1}{*}{{\textbf{3}}}}&{Temozolomida 90mg/m\textsuperscript{2}/dia Seg-Sex}&{}&&&{(  ) Sim (  ) Não}&\\
    \cline{3-5}
    \multicolumn{1}{c|}{\multirow{1}{*}{{\textbf{4}}}}&{Temozolomida 90mg/m\textsuperscript{2}/dia Seg-Sex}&{}&&&{(  ) Sim (  ) Não}&\\
    \cline{3-5}
    \multicolumn{1}{c|}{\multirow{1}{*}{{\textbf{5}}}}&{Temozolomida 90mg/m\textsuperscript{2}/dia Seg-Sex}&{}&&&{(  ) Sim (  ) Não}&\\
    \cline{3-5}
    \multicolumn{1}{c|}{\multirow{1}{*}{{\textbf{6}}}}&{Temozolomida 90mg/m\textsuperscript{2}/dia Seg-Sex}&{}&&&{(  ) Sim (  ) Não}&\\
    \hline
\end{longtable}
\textbf{Intervalo de 28 dias.}
\end{center}
\clearpage
\subsection{Manutenção: 06 ciclos}

\begin{center}
\begin{longtable}{p{1cm}p{4cm}|p{1cm}|p{5cm}|p{3cm}}
	\hline
	\multicolumn{5}{c}{\textbf{CICLO 1}}\\
\hline
    \multicolumn{1}{c|}{\multirow{1}{*}{\textbf{Dia}}}&{Dose}&{Data}&{Administrado}&{Rubrica} \\
    \hline
    \multicolumn{1}{c|}{\multirow{1}{*}{\textbf{D1}}}&{Temozolomida \(160\) mg/m\(^2\)}&&{(  ) Sim (  ) Não}&\\
    \multicolumn{1}{c|}{\multirow{1}{*}{\textbf{}}}&{Lomustina \(90\) mg/m\(^2\)}&&{(  ) Sim (  ) Não}&\\
    \multicolumn{1}{c|}{\multirow{1}{*}{\textbf{D2}}}&{Temozolomida \(160\) mg/m\(^2\)}&&{(  ) Sim (  ) Não}&\\
    \multicolumn{1}{c|}{\multirow{1}{*}{\textbf{D3}}}&{Temozolomida \(160\) mg/m\(^2\)}&&{(  ) Sim (  ) Não}&\\
    \multicolumn{1}{c|}{\multirow{1}{*}{\textbf{D4}}}&{Temozolomida \(160\) mg/m\(^2\)}&&{(  ) Sim (  ) Não}&\\
    \multicolumn{1}{c|}{\multirow{1}{*}{\textbf{D5}}}&{Temozolomida \(160\) mg/m\(^2\)}&&{(  ) Sim (  ) Não}&\\
    \hline
    \multicolumn{1}{c|}{\multirow{2}{*}{\textbf{Exames}}}&\multicolumn{2}{l|}{Neut (\(>7,5\times10^2\)):}&{Plaq (\(>7,5\times10^4\)):}&{TGO:}\\
    \cline{2-5}
    \multicolumn{1}{c|}{\multirow{2}{*}{{}}}&\multicolumn{2}{l|}{Creat(\(<1,5\) vezes)}&{BT(\(<1,5\) vezes):}&{TGP:}
    \\
    \hline
\end{longtable}
\textbf{Intervalo de 14 dias.}
\begin{longtable}{p{5cm}|p{5cm}|p{5cm}}
    \hline
    \textbf{Exames (data):}&{Neut (\(>7,5\times10^2\)):}&{Plaq (\(>7,5\times10^4\)):}
    \\
    \hline
\end{longtable}
\textbf{Intervalo de 28 dias.}
\\[1.5cm]
\end{center}

\begin{center}
\begin{longtable}{p{1cm}p{4cm}|p{1cm}|p{5cm}|p{3cm}}
	\hline
	\multicolumn{5}{c}{\textbf{CICLO 2}}\\
\hline
    \multicolumn{1}{c|}{\multirow{1}{*}{\textbf{Dia}}}&{Dose}&{Data}&{Administrado}&{Rubrica} \\
    \hline
    \multicolumn{1}{c|}{\multirow{1}{*}{\textbf{D1}}}&{Temozolomida \(160\) mg/m\(^2\)}&&{(  ) Sim (  ) Não}&\\
    \multicolumn{1}{c|}{\multirow{1}{*}{\textbf{}}}&{Lomustina \(90\) mg/m\(^2\)}&&{(  ) Sim (  ) Não}&\\
    \multicolumn{1}{c|}{\multirow{1}{*}{\textbf{D2}}}&{Temozolomida \(160\) mg/m\(^2\)}&&{(  ) Sim (  ) Não}&\\
    \multicolumn{1}{c|}{\multirow{1}{*}{\textbf{D3}}}&{Temozolomida \(160\) mg/m\(^2\)}&&{(  ) Sim (  ) Não}&\\
    \multicolumn{1}{c|}{\multirow{1}{*}{\textbf{D4}}}&{Temozolomida \(160\) mg/m\(^2\)}&&{(  ) Sim (  ) Não}&\\
    \multicolumn{1}{c|}{\multirow{1}{*}{\textbf{D5}}}&{Temozolomida \(160\) mg/m\(^2\)}&&{(  ) Sim (  ) Não}&\\
    \hline
    \multicolumn{1}{c|}{\multirow{2}{*}{\textbf{Exames}}}&\multicolumn{2}{l|}{Neut (\(>7,5\times10^2\)):}&{Plaq (\(>7,5\times10^4\)):}&{TGO:}\\
    \cline{2-5}
    \multicolumn{1}{c|}{\multirow{2}{*}{{}}}&\multicolumn{2}{l|}{Creat(\(<1,5\) vezes)}&{BT(\(<1,5\) vezes):}&{TGP:}
    \\
    \hline
\end{longtable}
\textbf{Intervalo de 14 dias.}
\begin{longtable}{p{5cm}|p{5cm}|p{5cm}}
    \hline
    \textbf{Exames (data):}&{Neut (\(>7,5\times10^2\)):}&{Plaq (\(>7,5\times10^4\)):}
    \\
    \hline
\end{longtable}
\textbf{Intervalo de 28 dias.}
\\[1.5cm]
\end{center}
\clearpage
\begin{center}
\begin{longtable}{p{1cm}p{4cm}|p{1cm}|p{5cm}|p{3cm}}
	\hline
	\multicolumn{5}{c}{\textbf{CICLO 3}}\\
\hline
    \multicolumn{1}{c|}{\multirow{1}{*}{\textbf{Dia}}}&{Dose}&{Data}&{Administrado}&{Rubrica} \\
    \hline
    \multicolumn{1}{c|}{\multirow{1}{*}{\textbf{D1}}}&{Temozolomida \(160\) mg/m\(^2\)}&&{(  ) Sim (  ) Não}&\\
    \multicolumn{1}{c|}{\multirow{1}{*}{\textbf{}}}&{Lomustina \(90\) mg/m\(^2\)}&&{(  ) Sim (  ) Não}&\\
    \multicolumn{1}{c|}{\multirow{1}{*}{\textbf{D2}}}&{Temozolomida \(160\) mg/m\(^2\)}&&{(  ) Sim (  ) Não}&\\
    \multicolumn{1}{c|}{\multirow{1}{*}{\textbf{D3}}}&{Temozolomida \(160\) mg/m\(^2\)}&&{(  ) Sim (  ) Não}&\\
    \multicolumn{1}{c|}{\multirow{1}{*}{\textbf{D4}}}&{Temozolomida \(160\) mg/m\(^2\)}&&{(  ) Sim (  ) Não}&\\
    \multicolumn{1}{c|}{\multirow{1}{*}{\textbf{D5}}}&{Temozolomida \(160\) mg/m\(^2\)}&&{(  ) Sim (  ) Não}&\\
    \hline
    \multicolumn{1}{c|}{\multirow{2}{*}{\textbf{Exames}}}&\multicolumn{2}{l|}{Neut (\(>7,5\times10^2\)):}&{Plaq (\(>7,5\times10^4\)):}&{TGO:}\\
    \cline{2-5}
    \multicolumn{1}{c|}{\multirow{2}{*}{{}}}&\multicolumn{2}{l|}{Creat(\(<1,5\) vezes)}&{BT(\(<1,5\) vezes):}&{TGP:}
    \\
    \hline
\end{longtable}
\textbf{Intervalo de 14 dias.}
\begin{longtable}{p{5cm}|p{5cm}|p{5cm}}
    \hline
    \textbf{Exames (data):}&{Neut (\(>7,5\times10^2\)):}&{Plaq (\(>7,5\times10^4\)):}
    \\
    \hline
\end{longtable}
\textbf{Intervalo de 28 dias.}
\\[1.5cm]
\end{center}

\begin{center}
\begin{longtable}{p{1cm}p{4cm}|p{1cm}|p{5cm}|p{3cm}}
	\hline
	\multicolumn{5}{c}{\textbf{CICLO 4}}\\
\hline
    \multicolumn{1}{c|}{\multirow{1}{*}{\textbf{Dia}}}&{Dose}&{Data}&{Administrado}&{Rubrica} \\
    \hline
    \multicolumn{1}{c|}{\multirow{1}{*}{\textbf{D1}}}&{Temozolomida \(160\) mg/m\(^2\)}&&{(  ) Sim (  ) Não}&\\
    \multicolumn{1}{c|}{\multirow{1}{*}{\textbf{}}}&{Lomustina \(90\) mg/m\(^2\)}&&{(  ) Sim (  ) Não}&\\
    \multicolumn{1}{c|}{\multirow{1}{*}{\textbf{D2}}}&{Temozolomida \(160\) mg/m\(^2\)}&&{(  ) Sim (  ) Não}&\\
    \multicolumn{1}{c|}{\multirow{1}{*}{\textbf{D3}}}&{Temozolomida \(160\) mg/m\(^2\)}&&{(  ) Sim (  ) Não}&\\
    \multicolumn{1}{c|}{\multirow{1}{*}{\textbf{D4}}}&{Temozolomida \(160\) mg/m\(^2\)}&&{(  ) Sim (  ) Não}&\\
    \multicolumn{1}{c|}{\multirow{1}{*}{\textbf{D5}}}&{Temozolomida \(160\) mg/m\(^2\)}&&{(  ) Sim (  ) Não}&\\
    \hline
    \multicolumn{1}{c|}{\multirow{2}{*}{\textbf{Exames}}}&\multicolumn{2}{l|}{Neut (\(>7,5\times10^2\)):}&{Plaq (\(>7,5\times10^4\)):}&{TGO:}\\
    \cline{2-5}
    \multicolumn{1}{c|}{\multirow{2}{*}{{}}}&\multicolumn{2}{l|}{Creat(\(<1,5\) vezes)}&{BT(\(<1,5\) vezes):}&{TGP:}
    \\
    \hline
\end{longtable}
\textbf{Intervalo de 14 dias.}
\begin{longtable}{p{5cm}|p{5cm}|p{5cm}}
    \hline
    \textbf{Exames (data):}&{Neut (\(>7,5\times10^2\)):}&{Plaq (\(>7,5\times10^4\)):}
    \\
    \hline
\end{longtable}
\textbf{Intervalo de 28 dias.}
\\[1.5cm]
\end{center}
\clearpage
\begin{center}
\begin{longtable}{p{1cm}p{4cm}|p{1cm}|p{5cm}|p{3cm}}
	\hline
	\multicolumn{5}{c}{\textbf{CICLO 5}}\\
\hline
    \multicolumn{1}{c|}{\multirow{1}{*}{\textbf{Dia}}}&{Dose}&{Data}&{Administrado}&{Rubrica} \\
    \hline
    \multicolumn{1}{c|}{\multirow{1}{*}{\textbf{D1}}}&{Temozolomida \(160\) mg/m\(^2\)}&&{(  ) Sim (  ) Não}&\\
    \multicolumn{1}{c|}{\multirow{1}{*}{\textbf{}}}&{Lomustina \(90\) mg/m\(^2\)}&&{(  ) Sim (  ) Não}&\\
    \multicolumn{1}{c|}{\multirow{1}{*}{\textbf{D2}}}&{Temozolomida \(160\) mg/m\(^2\)}&&{(  ) Sim (  ) Não}&\\
    \multicolumn{1}{c|}{\multirow{1}{*}{\textbf{D3}}}&{Temozolomida \(160\) mg/m\(^2\)}&&{(  ) Sim (  ) Não}&\\
    \multicolumn{1}{c|}{\multirow{1}{*}{\textbf{D4}}}&{Temozolomida \(160\) mg/m\(^2\)}&&{(  ) Sim (  ) Não}&\\
    \multicolumn{1}{c|}{\multirow{1}{*}{\textbf{D5}}}&{Temozolomida \(160\) mg/m\(^2\)}&&{(  ) Sim (  ) Não}&\\
    \hline
    \multicolumn{1}{c|}{\multirow{2}{*}{\textbf{Exames}}}&\multicolumn{2}{l|}{Neut (\(>7,5\times10^2\)):}&{Plaq (\(>7,5\times10^4\)):}&{TGO:}\\
    \cline{2-5}
    \multicolumn{1}{c|}{\multirow{2}{*}{{}}}&\multicolumn{2}{l|}{Creat(\(<1,5\) vezes)}&{BT(\(<1,5\) vezes):}&{TGP:}
    \\
    \hline
\end{longtable}
\textbf{Intervalo de 14 dias.}
\begin{longtable}{p{5cm}|p{5cm}|p{5cm}}
    \hline
    \textbf{Exames (data):}&{Neut (\(>7,5\times10^2\)):}&{Plaq (\(>7,5\times10^4\)):}
    \\
    \hline
\end{longtable}
\textbf{Intervalo de 28 dias.}
\\[1.5cm]
\end{center}

\begin{center}
\begin{longtable}{p{1cm}p{4cm}|p{1cm}|p{5cm}|p{3cm}}
	\hline
	\multicolumn{5}{c}{\textbf{CICLO 6}}\\
\hline
    \multicolumn{1}{c|}{\multirow{1}{*}{\textbf{Dia}}}&{Dose}&{Data}&{Administrado}&{Rubrica} \\
    \hline
    \multicolumn{1}{c|}{\multirow{1}{*}{\textbf{D1}}}&{Temozolomida \(160\) mg/m\(^2\)}&&{(  ) Sim (  ) Não}&\\
    \multicolumn{1}{c|}{\multirow{1}{*}{\textbf{}}}&{Lomustina \(90\) mg/m\(^2\)}&&{(  ) Sim (  ) Não}&\\
    \multicolumn{1}{c|}{\multirow{1}{*}{\textbf{D2}}}&{Temozolomida \(160\) mg/m\(^2\)}&&{(  ) Sim (  ) Não}&\\
    \multicolumn{1}{c|}{\multirow{1}{*}{\textbf{D3}}}&{Temozolomida \(160\) mg/m\(^2\)}&&{(  ) Sim (  ) Não}&\\
    \multicolumn{1}{c|}{\multirow{1}{*}{\textbf{D4}}}&{Temozolomida \(160\) mg/m\(^2\)}&&{(  ) Sim (  ) Não}&\\
    \multicolumn{1}{c|}{\multirow{1}{*}{\textbf{D5}}}&{Temozolomida \(160\) mg/m\(^2\)}&&{(  ) Sim (  ) Não}&\\
    \hline
    \multicolumn{1}{c|}{\multirow{2}{*}{\textbf{Exames}}}&\multicolumn{2}{l|}{Neut (\(>7,5\times10^2\)):}&{Plaq (\(>7,5\times10^4\)):}&{TGO:}\\
    \cline{2-5}
    \multicolumn{1}{c|}{\multirow{2}{*}{{}}}&\multicolumn{2}{l|}{Creat(\(<1,5\) vezes)}&{BT(\(<1,5\) vezes):}&{TGP:}
    \\
    \hline
\end{longtable}
\textbf{Intervalo de 14 dias.}
\begin{longtable}{p{5cm}|p{5cm}|p{5cm}}
    \hline
    \textbf{Exames (data):}&{Neut (\(>7,5\times10^2\)):}&{Plaq (\(>7,5\times10^4\)):}
    \\
    \hline
\end{longtable}
\textbf{Intervalo de 28 dias.}
\\[1.5cm]
\end{center}
\clearpage


\subsection{Modificações de dose:} 
Se atraso maior que 7 dias por toxicidade, reduzir os ciclos subsequentes para 150 mg/m\textsuperscript{2}/dia

\textbf{Avaliação:} imagem a cada 3 ciclos (3 meses), se progressão, interromper protocolo.

\textbf{APRESENTAÇÕES DE TEMOZOLOMIDA NO HIAS:} cápsulas de 100mg

\textbf{ADVERTÊNCIA:} SMZ+TMP não deve ser administrada juntamente com a temozolomida!

\textbf{ATENÇÃO:} este protocolo é \textit{off-label} (não padronizado) e não tem eficácia comprovada quando comparado com tratamento padrão sem QT. Dessa forma, é inadequado iniciar este protocolo em crianças com risco de complicações graves, como naquelas que têm sequelas importantes e muito limitantes.\\

\cleardoublepage

\section{GLIOMA DE ALTO GRAU E DIPG -- Adaptado do ensaio HIT-GBM-D - protocolo \textit{off-label}}
{\let\thefootnote\relax\footnotetext{Versão Janeiro/2015}}

\textbf{Racional:} no estudo fase II do GPOH, a manutenção simplificada com prednisona, vincristina e lomustina foi tão eficaz quanto a manutenção intensiva do HIT-GBM-C \footnote{Wolff et al, 2011}. Optamos por não usar a lomustina, apenas a manutenção com prednisona e VCR. O braço experimental com MTX não foi utilizado, apenas o braço controle.

\textbf{Elegível:} extensão da ressecção cirúrgica (RNM de controle 24-48h pós-op, máximo 72h após); citologia do LCR 10-14 dias pós-op. Iniciar 2 semanas após a cirurgia. Inclui glioblastoma multiforme, astrocitoma anaplásico, oligodendroglioma anaplásico, gliossarcoma. Pacientes com mais de 3 anos de idade. Somente iniciar o protocolo após a assinatura do TERMO DE CONSENTIMENTO INFORMADO, conforme acordado. NÃO INICIAR ESTE PROTOCOLO EM CRIANÇAS GRAVEMENTE ENFERMAS.

\textbf{Alternativa:} não existe esquema de QT amplamente aceito para tratar crianças com gliomas de alto grau, incluindo gliomas pontinos intrínsecos difusos (DIPG). O tratamento padrão é RT craniana. Os pacientes com mais de 3 anos têm um prognóstico insatisfatório e sobrevida mediana de pouco mais de 12 meses.

\vspace{5mm}
\entrywithlabel[.96\hsize]{\textbf{Nome}}\hfill \\

\entrywithlabel[.45\hsize]{\textbf{Peso}}\hfill  \entrywithlabel[.45\hsize]{\textbf{Estatura}}

\subsection{Indução: 6 semanas (radioquimioterapia)}
\textbf{Radioterapia:} 54 Gy para crianças 3-5 anos e 59,4 Gy para crianças a partir de 6 anos. DIPG: dose máxima 54 Gy.

\textbf{Síndrome pós-RT:} sonolência, fadiga e alterações do EEG podem ocorrer semanas a meses após a RT. Trata-se de uma síndrome reversível, o tratamento não deve ser modificado.

\textbf{Atraso no início da RT:} janela de vincristina semanal 1,5 mg/m\textsuperscript{2}, máximo 2mg, até iniciar RT.

\renewcommand{\arraystretch}{1.5}

\begin{center}
\begin{longtable}{p{1cm}c|p{4cm}|p{2cm}p{2cm}|c|c}
	\hline
	\multicolumn{7}{c}{PEV} \\
	\hline
	\multicolumn{1}{c|}{\multirow{1}{*}{\textbf{Dia}}}&{Data}&{}&\multicolumn{1}{c|}{Leuco}&\multicolumn{1}{c|}{Plaq}&{Administrado}&{Rubrica} \\
    \hline
    \multicolumn{1}{c|}{\multirow{2}{*}{\textbf{1}}}&\multirow{2}{*}{}&{Cisplatina \(20\) mg/m\(^2\)/dia}&\multicolumn{1}{c|}{\(>2000\)}&\multicolumn{1}{c|}{\(>10^5\)}&{(  ) Sim (  ) Não}&\\
    \cline{4-5}
    \multicolumn{1}{c|}{}&&{Etoposido \(100\) mg/m\(^2\)/dia)}&\multicolumn{1}{c|}{}&&{(  ) Sim (  ) Não}&\\
    \cline{2-6}
    \multicolumn{1}{c|}{\multirow{2}{*}{\textbf{2}}}&\multirow{2}{*}{}&{Cisplatina \(20\) mg/m\(^2\)/dia}&{}&&{(  ) Sim (  ) Não}&\\
    \multicolumn{1}{c|}{}&&{Etoposido \(100\) mg/m\(^2\)/dia)}&&&{(  ) Sim (  ) Não}&\\
    \cline{2-3}\cline{6-6}
    \multicolumn{1}{c|}{\multirow{2}{*}{\textbf{3}}}&\multirow{2}{*}{}&{Cisplatina \(20\) mg/m\(^2\)/dia}&{}&&{(  ) Sim (  ) Não}&\\
    \multicolumn{1}{c|}{}&&{Etoposido \(100\) mg/m\(^2\)/dia)}&&&{(  ) Sim (  ) Não}&\\
    \cline{2-3}\cline{6-6}
    \multicolumn{1}{c|}{\multirow{1}{*}{{\textbf{4}}}}&{}&{Cisplatina \(20\) mg/m\(^2\)/dia}&{}&&{(  ) Sim (  ) Não}&\\
    \cline{2-3}\cline{6-6}
    \multicolumn{1}{c|}{\multirow{2}{*}{\textbf{5}}}&\multirow{2}{*}{}&{Cisplatina \(20\) mg/m\(^2\)/dia}&{}&&{(  ) Sim (  ) Não}&\\
    \multicolumn{1}{c|}{}&&{Vincristina \(1,5\) mg/m\(^2\)/dia}&&&{(  ) Sim (  ) Não}&\\
    \hline
\end{longtable}
\begin{longtable}{p{1cm}c|p{4cm}|p{2cm}p{2cm}|c|c}
	\hline
	\multicolumn{7}{c}{Vincristina semanal} \\
	\hline
	\multicolumn{1}{c|}{\multirow{1}{*}{\textbf{Dia}}}&{Data}&{}&{}&&{Administrado}&{Rubrica} \\
    \hline
    \multicolumn{1}{c|}{\textbf{12}}&&{Vincristina \(1,5\) mg/m\(^2\)/dia}&\multicolumn{1}{c}{}&&{(  ) Sim (  ) Não}&\\
    \cline{1-3}\cline{6-6}
    \multicolumn{1}{c|}{\textbf{19}}&&{Vincristina \(1,5\) mg/m\(^2\)/dia}&\multicolumn{1}{c}{}&&{(  ) Sim (  ) Não}&\\
    \cline{1-3}\cline{6-6}
    \multicolumn{1}{c|}{\textbf{26}}&&{Vincristina \(1,5\) mg/m\(^2\)/dia}&\multicolumn{1}{c}{}&&{(  ) Sim (  ) Não}&\\
    \hline
\end{longtable}
\begin{longtable}{p{1cm}c|p{4cm}|p{2cm}p{2cm}|c|c}
	\hline
	\multicolumn{7}{c}{PEIV} \\
	\hline
	\multicolumn{1}{c|}{\multirow{1}{*}{\textbf{Dia}}}&{Data}&{}&\multicolumn{1}{c|}{Leuco}&\multicolumn{1}{c|}{Plaq}&{Administrado}&{Rubrica} \\
    \hline
    \multicolumn{1}{c|}{\multirow{3}{*}{\textbf{35}}}&\multirow{3}{*}{}&{Cisplatina \(20\) mg/m\(^2\)/dia}&\multicolumn{1}{c|}{\(>2000\)}&\multicolumn{1}{c|}{\(>10^5\)}&{(  ) Sim (  ) Não}&\\
    \cline{4-5}
    \multicolumn{1}{c|}{}&&{Etoposido \(100\) mg/m\(^2\)/dia)}&\multicolumn{1}{c|}{}&&{(  ) Sim (  ) Não}&\\
    \multicolumn{1}{c|}{}&&{Ifosfamida \(1500\) mg/m\(^2\)/dia}&\multicolumn{1}{c|}{}&&{(  ) Sim (  ) Não}&\\
    \cline{2-6}
    \multicolumn{1}{c|}{\multirow{3}{*}{\textbf{36}}}&\multirow{3}{*}{}&{Cisplatina \(20\) mg/m\(^2\)/dia}&{}&&{(  ) Sim (  ) Não}&\\
    \multicolumn{1}{c|}{}&&{Etoposido \(100\) mg/m\(^2\)/dia)}&&&{(  ) Sim (  ) Não}&\\
    \multicolumn{1}{c|}{}&&{Ifosfamida \(1500\) mg/m\(^2\)/dia}&&&{(  ) Sim (  ) Não}&\\
    \cline{2-3}\cline{6-6}
    \multicolumn{1}{c|}{\multirow{3}{*}{\textbf{37}}}&\multirow{3}{*}{}&{Cisplatina \(20\) mg/m\(^2\)/dia}&{}&&{(  ) Sim (  ) Não}&\\
    \multicolumn{1}{c|}{}&&{Etoposido \(100\) mg/m\(^2\)/dia)}&&&{(  ) Sim (  ) Não}&\\
    \multicolumn{1}{c|}{}&&{Ifosfamida \(1500\) mg/m\(^2\)/dia}&&&{(  ) Sim (  ) Não}&\\
    \cline{2-3}\cline{6-6}
    \multicolumn{1}{c|}{\multirow{2}{*}{{\textbf{38}}}}&\multirow{2}{*}{}&{Cisplatina \(20\) mg/m\(^2\)/dia}&{}&&{(  ) Sim (  ) Não}&\\
    \multicolumn{1}{c|}{}&&{Ifosfamida \(1500\) mg/m\(^2\)/dia}&&&{(  ) Sim (  ) Não}&\\
    \cline{2-3}\cline{6-6}
    \multicolumn{1}{c|}{\multirow{3}{*}{\textbf{39}}}&\multirow{3}{*}{}&{Cisplatina \(20\) mg/m\(^2\)/dia}&{}&&{(  ) Sim (  ) Não}&\\
    \multicolumn{1}{c|}{}&&{Ifosfamida \(1500\) mg/m\(^2\)/dia}&&&{(  ) Sim (  ) Não}&\\
    \multicolumn{1}{c|}{}&&{Vincristina \(1,5\) mg/m\(^2\)/dia}&&&{(  ) Sim (  ) Não}&\\
    \hline
\end{longtable}
\textbf{RNM após radioquimioterapia: considerar segunda cirurgia.}

\textbf{Intervalo de 28 dias}
\end{center}
\clearpage
\subsection{Manutenção: 08 ciclos}

\begin{center}
\begin{longtable}{p{1cm}c|p{5cm}|p{1cm}p{2cm}|c|c}
	\hline
	\multicolumn{7}{c}{Vincristina semanal} \\
	\hline
	\multicolumn{1}{c|}{\multirow{1}{*}{\textbf{Dia}}}&{Data}&{}&{}&&{Administrado}&{Rubrica} \\
    \hline
    \multicolumn{1}{c|}{\multirow{4}{*}{\textbf{63}}}&&{Prednisona \(40\) mg/m\(^2\)/dia por 11 dias}&\multicolumn{1}{c}{}&&{(  ) Sim (  ) Não}&\\
    \multicolumn{1}{c|}{}&&{Prednisona \(20\) mg/m\(^2\)/dia por 3 dias}&\multicolumn{1}{c}{}&&{(  ) Sim (  ) Não}&\\
    \multicolumn{1}{c|}{}&&{Prednisona \(10\) mg/m\(^2\)/dia por 3 dias}&\multicolumn{1}{c}{}&&{(  ) Sim (  ) Não}&\\
    \multicolumn{1}{c|}{\textbf{}}&&{Vincristina \(1,5\) mg/m\(^2\)/dia}&\multicolumn{1}{c}{}&&{(  ) Sim (  ) Não}&\\
    \cline{1-6}
    \multicolumn{1}{c|}{\textbf{70}}&&{Vincristina \(1,5\) mg/m\(^2\)/dia}&\multicolumn{1}{c}{}&&{(  ) Sim (  ) Não}&\\
    \cline{1-3}\cline{6-6}
    \multicolumn{1}{c|}{\textbf{77}}&&{Vincristina \(1,5\) mg/m\(^2\)/dia}&\multicolumn{1}{c}{}&&{(  ) Sim (  ) Não}&\\
    \hline
\end{longtable}
\textbf{Intervalo de 21 dias}
\begin{longtable}{p{1cm}c|p{5cm}|p{1cm}p{2cm}|c|c}
	\hline
	\multicolumn{7}{c}{Vincristina semanal} \\
	\hline
	\multicolumn{1}{c|}{\multirow{1}{*}{\textbf{Dia}}}&{Data}&{}&{}&&{Administrado}&{Rubrica} \\
    \hline
    \multicolumn{1}{c|}{\multirow{4}{*}{\textbf{98}}}&&{Prednisona \(40\) mg/m\(^2\)/dia por 11 dias}&\multicolumn{1}{c}{}&&{(  ) Sim (  ) Não}&\\
    \multicolumn{1}{c|}{}&&{Prednisona \(20\) mg/m\(^2\)/dia por 3 dias}&\multicolumn{1}{c}{}&&{(  ) Sim (  ) Não}&\\
    \multicolumn{1}{c|}{}&&{Prednisona \(10\) mg/m\(^2\)/dia por 3 dias}&\multicolumn{1}{c}{}&&{(  ) Sim (  ) Não}&\\
    \multicolumn{1}{c|}{\textbf{}}&&{Vincristina \(1,5\) mg/m\(^2\)/dia}&\multicolumn{1}{c}{}&&{(  ) Sim (  ) Não}&\\
    \cline{1-6}
    \multicolumn{1}{c|}{\textbf{105}}&&{Vincristina \(1,5\) mg/m\(^2\)/dia}&\multicolumn{1}{c}{}&&{(  ) Sim (  ) Não}&\\
    \cline{1-3}\cline{6-6}
    \multicolumn{1}{c|}{\textbf{112}}&&{Vincristina \(1,5\) mg/m\(^2\)/dia}&\multicolumn{1}{c}{}&&{(  ) Sim (  ) Não}&\\
    \hline
\end{longtable}
\textbf{Intervalo de 21 dias}
\begin{longtable}{p{1cm}c|p{5cm}|p{1cm}p{2cm}|c|c}
	\hline
	\multicolumn{7}{c}{Vincristina semanal} \\
	\hline
	\multicolumn{1}{c|}{\multirow{1}{*}{\textbf{Dia}}}&{Data}&{}&{}&&{Administrado}&{Rubrica} \\
    \hline
    \multicolumn{1}{c|}{\multirow{4}{*}{\textbf{133}}}&&{Prednisona \(40\) mg/m\(^2\)/dia por 11 dias}&\multicolumn{1}{c}{}&&{(  ) Sim (  ) Não}&\\
    \multicolumn{1}{c|}{}&&{Prednisona \(20\) mg/m\(^2\)/dia por 3 dias}&\multicolumn{1}{c}{}&&{(  ) Sim (  ) Não}&\\
    \multicolumn{1}{c|}{}&&{Prednisona \(10\) mg/m\(^2\)/dia por 3 dias}&\multicolumn{1}{c}{}&&{(  ) Sim (  ) Não}&\\
    \multicolumn{1}{c|}{\textbf{}}&&{Vincristina \(1,5\) mg/m\(^2\)/dia}&\multicolumn{1}{c}{}&&{(  ) Sim (  ) Não}&\\
    \cline{1-6}
    \multicolumn{1}{c|}{\textbf{140}}&&{Vincristina \(1,5\) mg/m\(^2\)/dia}&\multicolumn{1}{c}{}&&{(  ) Sim (  ) Não}&\\
    \cline{1-3}\cline{6-6}
    \multicolumn{1}{c|}{\textbf{147}}&&{Vincristina \(1,5\) mg/m\(^2\)/dia}&\multicolumn{1}{c}{}&&{(  ) Sim (  ) Não}&\\
    \hline
\end{longtable}
\textbf{Intervalo de 21 dias}
\\[0.8cm]
\begin{longtable}{p{1cm}c|p{5cm}|p{1cm}p{2cm}|c|c}
	\hline
	\multicolumn{7}{c}{Vincristina semanal} \\
	\hline
	\multicolumn{1}{c|}{\multirow{1}{*}{\textbf{Dia}}}&{Data}&{}&{}&&{Administrado}&{Rubrica} \\
    \hline
    \multicolumn{1}{c|}{\multirow{4}{*}{\textbf{168}}}&&{Prednisona \(40\) mg/m\(^2\)/dia por 11 dias}&\multicolumn{1}{c}{}&&{(  ) Sim (  ) Não}&\\
    \multicolumn{1}{c|}{}&&{Prednisona \(20\) mg/m\(^2\)/dia por 3 dias}&\multicolumn{1}{c}{}&&{(  ) Sim (  ) Não}&\\
    \multicolumn{1}{c|}{}&&{Prednisona \(10\) mg/m\(^2\)/dia por 3 dias}&\multicolumn{1}{c}{}&&{(  ) Sim (  ) Não}&\\
    \multicolumn{1}{c|}{\textbf{}}&&{Vincristina \(1,5\) mg/m\(^2\)/dia}&\multicolumn{1}{c}{}&&{(  ) Sim (  ) Não}&\\
    \cline{1-6}
    \multicolumn{1}{c|}{\textbf{175}}&&{Vincristina \(1,5\) mg/m\(^2\)/dia}&\multicolumn{1}{c}{}&&{(  ) Sim (  ) Não}&\\
    \cline{1-3}\cline{6-6}
    \multicolumn{1}{c|}{\textbf{182}}&&{Vincristina \(1,5\) mg/m\(^2\)/dia}&\multicolumn{1}{c}{}&&{(  ) Sim (  ) Não}&\\
    \hline
\end{longtable}
\textbf{Intervalo de 21 dias}
\begin{longtable}{p{1cm}c|p{5cm}|p{1cm}p{2cm}|c|c}
	\hline
	\multicolumn{7}{c}{Vincristina semanal} \\
	\hline
	\multicolumn{1}{c|}{\multirow{1}{*}{\textbf{Dia}}}&{Data}&{}&{}&&{Administrado}&{Rubrica} \\
    \hline
    \multicolumn{1}{c|}{\multirow{4}{*}{\textbf{203}}}&&{Prednisona \(40\) mg/m\(^2\)/dia por 11 dias}&\multicolumn{1}{c}{}&&{(  ) Sim (  ) Não}&\\
    \multicolumn{1}{c|}{}&&{Prednisona \(20\) mg/m\(^2\)/dia por 3 dias}&\multicolumn{1}{c}{}&&{(  ) Sim (  ) Não}&\\
    \multicolumn{1}{c|}{}&&{Prednisona \(10\) mg/m\(^2\)/dia por 3 dias}&\multicolumn{1}{c}{}&&{(  ) Sim (  ) Não}&\\
    \multicolumn{1}{c|}{\textbf{}}&&{Vincristina \(1,5\) mg/m\(^2\)/dia}&\multicolumn{1}{c}{}&&{(  ) Sim (  ) Não}&\\
    \cline{1-6}
    \multicolumn{1}{c|}{\textbf{210}}&&{Vincristina \(1,5\) mg/m\(^2\)/dia}&\multicolumn{1}{c}{}&&{(  ) Sim (  ) Não}&\\
    \cline{1-3}\cline{6-6}
    \multicolumn{1}{c|}{\textbf{217}}&&{Vincristina \(1,5\) mg/m\(^2\)/dia}&\multicolumn{1}{c}{}&&{(  ) Sim (  ) Não}&\\
    \hline
\end{longtable}
\textbf{Intervalo de 21 dias}
\begin{longtable}{p{1cm}c|p{5cm}|p{1cm}p{2cm}|c|c}
	\hline
	\multicolumn{7}{c}{Vincristina semanal} \\
	\hline
	\multicolumn{1}{c|}{\multirow{1}{*}{\textbf{Dia}}}&{Data}&{}&{}&&{Administrado}&{Rubrica} \\
    \hline
    \multicolumn{1}{c|}{\multirow{4}{*}{\textbf{238}}}&&{Prednisona \(40\) mg/m\(^2\)/dia por 11 dias}&\multicolumn{1}{c}{}&&{(  ) Sim (  ) Não}&\\
    \multicolumn{1}{c|}{}&&{Prednisona \(20\) mg/m\(^2\)/dia por 3 dias}&\multicolumn{1}{c}{}&&{(  ) Sim (  ) Não}&\\
    \multicolumn{1}{c|}{}&&{Prednisona \(10\) mg/m\(^2\)/dia por 3 dias}&\multicolumn{1}{c}{}&&{(  ) Sim (  ) Não}&\\
    \multicolumn{1}{c|}{\textbf{}}&&{Vincristina \(1,5\) mg/m\(^2\)/dia}&\multicolumn{1}{c}{}&&{(  ) Sim (  ) Não}&\\
    \cline{1-6}
    \multicolumn{1}{c|}{\textbf{245}}&&{Vincristina \(1,5\) mg/m\(^2\)/dia}&\multicolumn{1}{c}{}&&{(  ) Sim (  ) Não}&\\
    \cline{1-3}\cline{6-6}
    \multicolumn{1}{c|}{\textbf{252}}&&{Vincristina \(1,5\) mg/m\(^2\)/dia}&\multicolumn{1}{c}{}&&{(  ) Sim (  ) Não}&\\
    \hline
\end{longtable}
\textbf{Intervalo de 21 dias}
\\[1.5cm]
\begin{longtable}{p{1cm}c|p{5cm}|p{1cm}p{2cm}|c|c}
	\hline
	\multicolumn{7}{c}{Vincristina semanal} \\
	\hline
	\multicolumn{1}{c|}{\multirow{1}{*}{\textbf{Dia}}}&{Data}&{}&{}&&{Administrado}&{Rubrica} \\
    \hline
    \multicolumn{1}{c|}{\multirow{4}{*}{\textbf{273}}}&&{Prednisona \(40\) mg/m\(^2\)/dia por 11 dias}&\multicolumn{1}{c}{}&&{(  ) Sim (  ) Não}&\\
    \multicolumn{1}{c|}{}&&{Prednisona \(20\) mg/m\(^2\)/dia por 3 dias}&\multicolumn{1}{c}{}&&{(  ) Sim (  ) Não}&\\
    \multicolumn{1}{c|}{}&&{Prednisona \(10\) mg/m\(^2\)/dia por 3 dias}&\multicolumn{1}{c}{}&&{(  ) Sim (  ) Não}&\\
    \multicolumn{1}{c|}{\textbf{}}&&{Vincristina \(1,5\) mg/m\(^2\)/dia}&\multicolumn{1}{c}{}&&{(  ) Sim (  ) Não}&\\
    \cline{1-6}
    \multicolumn{1}{c|}{\textbf{280}}&&{Vincristina \(1,5\) mg/m\(^2\)/dia}&\multicolumn{1}{c}{}&&{(  ) Sim (  ) Não}&\\
    \cline{1-3}\cline{6-6}
    \multicolumn{1}{c|}{\textbf{287}}&&{Vincristina \(1,5\) mg/m\(^2\)/dia}&\multicolumn{1}{c}{}&&{(  ) Sim (  ) Não}&\\
    \hline
\end{longtable}
\textbf{Intervalo de 21 dias}
\begin{longtable}{p{1cm}c|p{5cm}|p{1cm}p{2cm}|c|c}
	\hline
	\multicolumn{7}{c}{Vincristina semanal} \\
	\hline
	\multicolumn{1}{c|}{\multirow{1}{*}{\textbf{Dia}}}&{Data}&{}&{}&&{Administrado}&{Rubrica} \\
    \hline
    \multicolumn{1}{c|}{\multirow{4}{*}{\textbf{308}}}&&{Prednisona \(40\) mg/m\(^2\)/dia por 11 dias}&\multicolumn{1}{c}{}&&{(  ) Sim (  ) Não}&\\
    \multicolumn{1}{c|}{}&&{Prednisona \(20\) mg/m\(^2\)/dia por 3 dias}&\multicolumn{1}{c}{}&&{(  ) Sim (  ) Não}&\\
    \multicolumn{1}{c|}{}&&{Prednisona \(10\) mg/m\(^2\)/dia por 3 dias}&\multicolumn{1}{c}{}&&{(  ) Sim (  ) Não}&\\
    \multicolumn{1}{c|}{\textbf{}}&&{Vincristina \(1,5\) mg/m\(^2\)/dia}&\multicolumn{1}{c}{}&&{(  ) Sim (  ) Não}&\\
    \cline{1-6}
    \multicolumn{1}{c|}{\textbf{315}}&&{Vincristina \(1,5\) mg/m\(^2\)/dia}&\multicolumn{1}{c}{}&&{(  ) Sim (  ) Não}&\\
    \cline{1-3}\cline{6-6}
    \multicolumn{1}{c|}{\textbf{322}}&&{Vincristina \(1,5\) mg/m\(^2\)/dia}&\multicolumn{1}{c}{}&&{(  ) Sim (  ) Não}&\\
    \hline
\end{longtable}

\textbf{FIM DE PROTOCOLO}

\end{center}
\subsection{Modificações de dose:} 
\textbf{ATENÇÃO:} este protocolo é off-label (não padronizado) e não tem eficácia comprovada quando comparado com tratamento padrão sem QT. Dessa forma, é inadequado iniciar este protocolo em crianças com risco de complicações graves, como naquelas que têm sequelas importantes e muito limitantes.

\textbf{Avaliação:} imagem a cada 3 ciclos (3 meses), se progressão, interromper protocolo. RNM após fim do protocolo (30 semanas): considerar segunda cirurgia.

\textbf{Evitar dexametasona} durante a radioquimioterapia. Se necessário, na fase de RT apenas (entre PEV e PEIV), usar 1 mg/m\textsuperscript{2}/dia cada 8h.

\textbf{Profilaxia textit{P. carinii:}} sulfametoxazol + trimetoprima durante todo o tratamento.

\noindent{\textbf{Ajustes de dose:}}\\
Durante a radioterapia:\\
1. Não interromper por alterações do hemograma
2. Transfundir plaquetas para manter \(>20\) mil/m\(^3\)\\
3. Transfundir hemoglobina para manter igual ou acima de \(10\) mg/dl\\
4. Tratar leucopenia com G-CSF\\
5. Interromper radioterapia apenasse sintomas clínicos ameaçadores\\
6. Não deixar de fazer as doses de vincristina do D12, 19 e 26
\begin{center}
\textbf{Critérios de controle para radioquimioterapia (PEV):}

\begin{longtable}{p{5cm}|p{4.5cm}}
\hline
{\textbf{Critérios}}&{\textbf{Recomendações de tratamento}}\\
\hline
{Leucócitos \(> 2000\)/mm\(^3\) e Plaquetas > 100 mil/ mm\(^3\)}&{Iniciar radioquimioterapia concomitante}\\
\hline
{Leucócitos: \(1500-2000\)/ mm\(^3\) e/ou Plaquetas: \(50-100\) mil/ mm\(^3\)}&{\(100\)\% da dose de radioterapia \(66\)\% da dose de quimioterapia}\\
\hline
{Leucócitos: \(1000-1500\)/ mm\(^3\) e/ou Plaquetas: \(30-50\) mil/ mm\(^3\)}&{\(100\)\% da dose de radioterapia Adiar a quimioterapia 1 semana}\\
\hline
{Leucócitos: \(<1000\)/ mm\(^3\) e/ou Plaquetas: \(<30\) mil/ mm\(^3\)}&{Adiar ambos os tratamentos por 1 semana}\\
\hline
{Convulsões na semana anterior}&{Adiar vincristina por 1 semana Iniciar anti-epiléptico}\\
\hline
{Obstipação}&{Adiar vincristina por 1 semana Lactulose \(0,1\) g/kg cada 12h profilática}\\
\hline
\end{longtable}
\textbf{Antes do segundo bloco (PEIV):}

\begin{longtable}{p{5cm}|p{4.5cm}}
\hline
{\textbf{Critérios}}&{\textbf{Recomendações de tratamento}}\\
\hline
{Leucócitos \(> 2000\)/mm\(^3\) e Plaquetas \(> 100\) mil/ mm\(^3\)}&{Iniciar radioquimioterapia concomitante}\\
\hline
{Leucócitos: \(1500-2000\)/ mm\(^3\) e/ou Plaquetas: \(50-100\) mil/ mm\(^3\)}&{\(100\)\% da dose de cisplatina \(50\)\% da dose de etoposido/ifosfamida}\\
\hline
{Perda de audição de \(>25\) dB a \(6000\) Hz após 1 bloco de cisplatina}&{Trocar a cisplatina por carboplatina \(200\) mg/m\(^2\) D1-3, infundir diluído em 1h}\\
\hline
{Perda de audição de \(>40\) dB a \(4000\) Hz}&{Evitar qualquer composto de platina}\\
\hline
{Clearance de creatinina de \(50-80\) ml/min}&{Trocar a cisplatina por carboplatina \(200\) mg/m\(^2\) D1-3, infundir diluído em 1h}\\
\hline
{Clearance de creatinina menor que \(50\) ml/min}&{Sem compostos de platina \(30\)\% da dose de ifosfamida}\\
\hline
{Hematúria}&{Sem ifosfamida}\\
\hline
\end{longtable}

\end{center}

\textbf{Reação ao etoposido:}
Na reação aguda: interromper a infusão, administrar anti-histamínicos e corticóide, re-iniciar infusão na metade da velocidade.
Antes da próxima dose: anti-histamínicos e corticóide pré-quimioterapia e metade da velocidade de infusão.
\cleardoublepage

\section{EPENDIMOMA NÃO METASTÁTICO -- Adaptado dos ensaios ACNS0121 e ACNS0831 - protocolo \textit{off-label}}
{\let\thefootnote\relax\footnotetext{Versão Janeiro/2017}}

\textbf{Racional:} no estudo recém-terminado do COG (ACNS0121), os pacientes foram estratificados em 4 grupos (\footnote{Merchant et al, 2015}):
\begin{itemize}
\item[Estrato 1] - Paciente com ependimomas clássicos supratentoriais com ressecção microscopicamente completa.
\item[Estrato 2] - Pacientes com ressecção parcial.
\item[Estrato 3] - Pacientes com ressecção total macroscópica ou subtotal (até 5 mm de tumor residual).
\item[Estrato 4] - Pacientes com ependimomas anaplásicos supratentoriais com ressecção total ou infratentoriais (qualquer histologia) com ressecção total.
\end{itemize}

Os pacientes do estrato 1 foram apenas observados, sendo que 5 de 11 apresentaram recidiva. Os pacientes do estrato 3 e 4 receberam RT conformacional no leito tumoral logo após a ressecção, com sobrevida livre de eventos de 75\% para ependimoma clássico e 60\% para ependimoma anaplásico. Os pacientes do estrato 2 receberam QT antes da RT e foram avaliados para possível nova cirurgia. A sobrevida livre de eventos deste último grupo foi de 40\%. Os resultados deste estudo deixaram claro que a primeira escolha de tratamento para ependimoma, clássico ou anaplásico, é a RT no leito tumoral imediatamente após a cirurgia, a partir de 1 ano de idade. O uso de QT com este protocolo, assim, deve ser deixado para casos selecionados nos quais a RT inicial tem poucas chances de ser bem sucedida. O estudo em andamento ACNS0831 inclui QT pós-RT para pacientes do estrato 2.

\textbf{Elegível:}Estadiamento pré-tratamento: mais de 0,5cm de tumor residual (RNM de controle até 21 dias pós-op, preferido 72h após); sem metástases (RNM de neuro-eixo e PL/MO). Tratamento precisa iniciar até 56 dias após cirurgia. Apenas EPENDIMOMA (excluindo espinhal). Pacientes com mais de 3 anos de idade. Somente iniciar o protocolo após a assinatura do TERMO DE CONSENTIMENTO INFORMADO, conforme acordado. NÃO INICIAR ESTE PROTOCOLO EM CRIANÇAS GRAVEMENTE ENFERMAS.

\textbf{Alternativa:} não existe esquema de QT amplamente aceito para tratar crianças com ependimoma. O tratamento padrão é RT no leito tumoral. Os pacientes com ressecção incompleta têm um prognóstico insatisfatório e sobrevida livre de progressão prolongada reduzida.

\vspace{5mm}
\entrywithlabel[.96\hsize]{\textbf{Nome}}\hfill \\

\entrywithlabel[.45\hsize]{\textbf{Peso}}\hfill  \entrywithlabel[.45\hsize]{\textbf{Estatura}}

\subsection{Indução: 2 ciclos (pré-radioterapia)}
\renewcommand{\arraystretch}{1.5}

\begin{center}
\begin{longtable}{p{1cm}c|p{5cm}|p{1.5cm}p{1.5cm}|c|c}
	\hline
	\multicolumn{1}{c|}{\multirow{1}{*}{\textbf{Dia}}}&{Data}&{}&\multicolumn{1}{c|}{Leuco}&\multicolumn{1}{c|}{Plaq}&{Administrado}&{Rubrica} \\
    \hline
    \multicolumn{1}{c|}{\multirow{4}{*}{\textbf{1}}}&\multirow{2}{*}{}&{Carboplatina 3\(75\) mg/m\(^2\)/dia}&\multicolumn{1}{c|}{\(>10^3\)}&\multicolumn{1}{c|}{\(>10^5\)}&{(  ) Sim (  ) Não}&\\
    \cline{4-5}
    \multicolumn{1}{c|}{}&&{Vincristina \(1,5\) mg/m\(^2\)/dia}&\multicolumn{1}{c|}{}&&{(  ) Sim (  ) Não}&\\
    \cline{4-5}
    \multicolumn{1}{c|}{}&\multirow{1}{*}{}&{Ciclofosfamida \(1000\) mg/m\(^2\)/dia}&{}&&{(  ) Sim (  ) Não}&\\
    \multicolumn{1}{c|}{}&&{MESNA \(200\) mg/m\(^2 \times 0,1,5\)h}&&&{(  ) Sim (  ) Não}&\\
    \cline{2-3}\cline{6-6}
    \multicolumn{1}{c|}{\multirow{2}{*}{\textbf{2}}}&\multirow{2}{*}{}&{Ciclofosfamida \(1000\) mg/m\(^2\)/dia}&{}&&{(  ) Sim (  ) Não}&\\
    \multicolumn{1}{c|}{}&&{MESNA \(200\) mg/m\(^2 \times 0,1,5\)h}&&&{(  ) Sim (  ) Não}&\\
    \hline
    \multicolumn{1}{c|}{\textbf{3-12}}&&{G-CSF \(5 \mu g/ml\) }&&&{(  ) Sim (  ) Não}&\\
    \hline
\end{longtable}
\begin{longtable}{p{1cm}c|p{4cm}|p{2cm}p{2cm}|c|c}
	\hline
	\multicolumn{1}{c|}{\multirow{1}{*}{\textbf{Dia}}}&{Data}&{}&{}&&{Administrado}&{Rubrica} \\
    \hline
    \multicolumn{1}{c|}{\textbf{8}}&&{Vincristina \(1,5\) mg/m\(^2\)/dia}&\multicolumn{1}{c}{}&&{(  ) Sim (  ) Não}&\\
    \hline
\end{longtable}
\textbf{Intervalo de 14 dias}

\clearpage

\begin{longtable}{p{1cm}c|p{5cm}|p{1.5cm}p{1.5cm}|c|c}
	\hline
	\multicolumn{1}{c|}{\multirow{1}{*}{\textbf{Dia}}}&{Data}&{}&\multicolumn{1}{c|}{Leuco}&\multicolumn{1}{c|}{Plaq}&{Administrado}&{Rubrica} \\
    \hline
    \multicolumn{1}{c|}{\multirow{3}{*}{\textbf{21}}}&\multirow{2}{*}{}&{Carboplatina 3\(75\) mg/m\(^2\)/dia}&\multicolumn{1}{c|}{\(>10^3\)}&\multicolumn{1}{c|}{\(>10^5\)}&{(  ) Sim (  ) Não}&\\
    \cline{4-5}
    \multicolumn{1}{c|}{}&&{Vincristina \(1,5\) mg/m\(^2\)/dia}&\multicolumn{1}{c|}{}&&{(  ) Sim (  ) Não}&\\
    \cline{4-5}
    \multicolumn{1}{c|}{}&{}&{Etoposido \(100\) mg/m\(^2\)/dia)}&{}&&{(  ) Sim (  ) Não}&\\
    \cline{2-3}\cline{6-6}
    \multicolumn{1}{c|}{\textbf{22}}&&{Etoposido \(100\) mg/m\(^2\)/dia)}&&&{(  ) Sim (  ) Não}&\\
    \cline{2-3}\cline{6-6}
    \multicolumn{1}{c|}{\multirow{1}{*}{\textbf{23}}}&&{Etoposido \(100\) mg/m\(^2\)/dia)}&{}&&{(  ) Sim (  ) Não}&\\
    \hline
\end{longtable}
\begin{longtable}{p{1cm}c|p{4cm}|p{2cm}p{2cm}|c|c}
	\hline
	\multicolumn{1}{c|}{\multirow{1}{*}{\textbf{Dia}}}&{Data}&{}&{}&&{Administrado}&{Rubrica} \\
    \hline
    \multicolumn{1}{c|}{\textbf{28}}&&{Vincristina \(1,5\) mg/m\(^2\)/dia}&\multicolumn{1}{c}{}&&{(  ) Sim (  ) Não}&\\
    \hline
\end{longtable}
\textit{\textbf{Reavaliar com imagem – Re-operação se possível}}

\textit{\textbf{Encaminhar para Radioterapia}}

\end{center}

\subsection{Manutenção: 04 ciclos (VCEC)}
Apenas pacientes com ependimomas ressecados incompletamente, de acordo com a definição do COG.

\begin{center}
\begin{longtable}{p{1cm}c|p{5cm}|p{1.5cm}p{1.5cm}|c|c}
	\hline
	\multicolumn{7}{c}{Ciclo 1} \\
	\hline
	\multicolumn{1}{c|}{\multirow{1}{*}{\textbf{Dia}}}&{Data}&{}&\multicolumn{1}{c|}{Leuco}&\multicolumn{1}{c|}{Plaq}&{Administrado}&{Rubrica} \\
    \hline
    \multicolumn{1}{c|}{\multirow{3}{*}{\textbf{1}}}&&{Vincristina \(1,5\) mg/m\(^2\)/dia}&\multicolumn{1}{c|}{\(>10^3\)}&\multicolumn{1}{c|}{\(>10^5\)}&{(  ) Sim (  ) Não}&\\
    \cline{4-5}
    \multicolumn{1}{c|}{}&&{Etoposido \(100\) mg/m\(^2\)/dia)}&&&{(  ) Sim (  ) Não}&\\
    \cline{4-5}
    \multicolumn{1}{c|}{}&\multirow{1}{*}{}&{Cisplatina \(100\) mg/m\(^2\)/dia)}&&&{(  ) Sim (  ) Não}&\\
    \hline
    \multicolumn{1}{c|}{\multirow{3}{*}{\textbf{2}}}&&{Ciclofosfamida \(1000\) mg/m\(^2\)/dia}&{}&&{(  ) Sim (  ) Não}&\\
    \multicolumn{1}{c|}{}&&{MESNA \(200\) mg/m\(^2 \times 0,1,5\)h}&&&{(  ) Sim (  ) Não}&\\
    \multicolumn{1}{c|}{}&&{Etoposido \(100\) mg/m\(^2\)/dia)}&&&{(  ) Sim (  ) Não}&\\
    \hline
    \multicolumn{1}{c|}{\multirow{3}{*}{\textbf{3}}}&&{Ciclofosfamida \(1000\) mg/m\(^2\)/dia}&{}&&{(  ) Sim (  ) Não}&\\
    \multicolumn{1}{c|}{}&&{MESNA \(200\) mg/m\(^2 \times 0,1,5\)h}&&&{(  ) Sim (  ) Não}&\\
    \multicolumn{1}{c|}{}&\multirow{1}{*}{}&{Etoposido \(100\) mg/m\(^2\)/dia)}&{}&&{(  ) Sim (  ) Não}&\\
    \hline
    \multicolumn{1}{c|}{\textbf{4-13}}&&{G-CSF \(5 \mu g/ml\) }&&&{(  ) Sim (  ) Não}&\\
    \hline
\end{longtable}
\begin{longtable}{p{1cm}c|p{4cm}|p{2cm}p{2cm}|c|c}
	\hline
	\multicolumn{1}{c|}{\multirow{1}{*}{\textbf{Dia}}}&{Data}&{}&{}&&{Administrado}&{Rubrica} \\
    \hline
    \multicolumn{1}{c|}{\textbf{8}}&&{Vincristina \(1,5\) mg/m\(^2\)/dia}&\multicolumn{1}{c}{}&&{(  ) Sim (  ) Não}&\\
    \hline
    \multicolumn{1}{c|}{\textbf{15}}&&{Vincristina \(1,5\) mg/m\(^2\)/dia}&\multicolumn{1}{c}{}&&{(  ) Sim (  ) Não}&\\
    \hline
\end{longtable}

\begin{longtable}{p{1cm}c|p{5cm}|p{1.5cm}p{1.5cm}|c|c}
	\hline
	\multicolumn{7}{c}{Ciclo 2} \\
	\hline
	\multicolumn{1}{c|}{\multirow{1}{*}{\textbf{Dia}}}&{Data}&{}&\multicolumn{1}{c|}{Leuco}&\multicolumn{1}{c|}{Plaq}&{Administrado}&{Rubrica} \\
    \hline
    \multicolumn{1}{c|}{\multirow{3}{*}{\textbf{22}}}&&{Vincristina \(1,5\) mg/m\(^2\)/dia}&\multicolumn{1}{c|}{\(>10^3\)}&\multicolumn{1}{c|}{\(>10^5\)}&{(  ) Sim (  ) Não}&\\
    \cline{4-5}
    \multicolumn{1}{c|}{}&&{Etoposido \(100\) mg/m\(^2\)/dia)}&&&{(  ) Sim (  ) Não}&\\
    \cline{4-5}
    \multicolumn{1}{c|}{}&\multirow{1}{*}{}&{Cisplatina \(100\) mg/m\(^2\)/dia)}&&&{(  ) Sim (  ) Não}&\\
    \hline
    \multicolumn{1}{c|}{\multirow{3}{*}{\textbf{23}}}&&{Ciclofosfamida \(1000\) mg/m\(^2\)/dia}&{}&&{(  ) Sim (  ) Não}&\\
    \multicolumn{1}{c|}{}&&{MESNA \(200\) mg/m\(^2 \times 0,1,5\)h}&&&{(  ) Sim (  ) Não}&\\
    \multicolumn{1}{c|}{}&&{Etoposido \(100\) mg/m\(^2\)/dia)}&&&{(  ) Sim (  ) Não}&\\
    \hline
    \multicolumn{1}{c|}{\multirow{3}{*}{\textbf{24}}}&&{Ciclofosfamida \(1000\) mg/m\(^2\)/dia}&{}&&{(  ) Sim (  ) Não}&\\
    \multicolumn{1}{c|}{}&&{MESNA \(200\) mg/m\(^2 \times 0,1,5\)h}&&&{(  ) Sim (  ) Não}&\\
    \multicolumn{1}{c|}{}&\multirow{1}{*}{}&{Etoposido \(100\) mg/m\(^2\)/dia)}&{}&&{(  ) Sim (  ) Não}&\\
    \hline
    \multicolumn{1}{c|}{\textbf{25-34}}&&{G-CSF \(5 \mu g/ml\) }&&&{(  ) Sim (  ) Não}&\\
    \hline
\end{longtable}
\begin{longtable}{p{1cm}c|p{4cm}|p{2cm}p{2cm}|c|c}
	\hline
	\multicolumn{1}{c|}{\multirow{1}{*}{\textbf{Dia}}}&{Data}&{}&{}&&{Administrado}&{Rubrica} \\
    \hline
    \multicolumn{1}{c|}{\textbf{29}}&&{Vincristina \(1,5\) mg/m\(^2\)/dia}&\multicolumn{1}{c}{}&&{(  ) Sim (  ) Não}&\\
    \hline
    \multicolumn{1}{c|}{\textbf{36}}&&{Vincristina \(1,5\) mg/m\(^2\)/dia}&\multicolumn{1}{c}{}&&{(  ) Sim (  ) Não}&\\
    \hline
\end{longtable}
\begin{longtable}{p{1cm}c|p{5cm}|p{1.5cm}p{1.5cm}|c|c}
	\hline
	\multicolumn{7}{c}{Ciclo 3} \\
	\hline
	\multicolumn{1}{c|}{\multirow{1}{*}{\textbf{Dia}}}&{Data}&{}&\multicolumn{1}{c|}{Leuco}&\multicolumn{1}{c|}{Plaq}&{Administrado}&{Rubrica} \\
    \hline
    \multicolumn{1}{c|}{\multirow{3}{*}{\textbf{43}}}&&{Vincristina \(1,5\) mg/m\(^2\)/dia}&\multicolumn{1}{c|}{\(>10^3\)}&\multicolumn{1}{c|}{\(>10^5\)}&{(  ) Sim (  ) Não}&\\
    \cline{4-5}
    \multicolumn{1}{c|}{}&&{Etoposido \(100\) mg/m\(^2\)/dia)}&&&{(  ) Sim (  ) Não}&\\
    \cline{4-5}
    \multicolumn{1}{c|}{}&\multirow{1}{*}{}&{Cisplatina \(100\) mg/m\(^2\)/dia)}&&&{(  ) Sim (  ) Não}&\\
    \hline
    \multicolumn{1}{c|}{\multirow{3}{*}{\textbf{44}}}&&{Ciclofosfamida \(1000\) mg/m\(^2\)/dia}&{}&&{(  ) Sim (  ) Não}&\\
    \multicolumn{1}{c|}{}&&{MESNA \(200\) mg/m\(^2 \times 0,1,5\)h}&&&{(  ) Sim (  ) Não}&\\
    \multicolumn{1}{c|}{}&&{Etoposido \(100\) mg/m\(^2\)/dia)}&&&{(  ) Sim (  ) Não}&\\
    \hline
    \multicolumn{1}{c|}{\multirow{3}{*}{\textbf{45}}}&&{Ciclofosfamida \(1000\) mg/m\(^2\)/dia}&{}&&{(  ) Sim (  ) Não}&\\
    \multicolumn{1}{c|}{}&&{MESNA \(200\) mg/m\(^2 \times 0,1,5\)h}&&&{(  ) Sim (  ) Não}&\\
    \multicolumn{1}{c|}{}&\multirow{1}{*}{}&{Etoposido \(100\) mg/m\(^2\)/dia)}&{}&&{(  ) Sim (  ) Não}&\\
    \hline
    \multicolumn{1}{c|}{\textbf{46-55}}&&{G-CSF \(5 \mu g/ml\) }&&&{(  ) Sim (  ) Não}&\\
    \hline
\end{longtable}
\clearpage
\begin{longtable}{p{1cm}c|p{4cm}|p{2cm}p{2cm}|c|c}
	\hline
	\multicolumn{1}{c|}{\multirow{1}{*}{\textbf{Dia}}}&{Data}&{}&{}&&{Administrado}&{Rubrica} \\
    \hline
    \multicolumn{1}{c|}{\textbf{50}}&&{Vincristina \(1,5\) mg/m\(^2\)/dia}&\multicolumn{1}{c}{}&&{(  ) Sim (  ) Não}&\\
    \hline
    \multicolumn{1}{c|}{\textbf{57}}&&{Vincristina \(1,5\) mg/m\(^2\)/dia}&\multicolumn{1}{c}{}&&{(  ) Sim (  ) Não}&\\
    \hline
\end{longtable}

\begin{longtable}{p{1cm}c|p{5cm}|p{1.5cm}p{1.5cm}|c|c}
	\hline
	\multicolumn{7}{c}{Ciclo 4} \\
	\hline
	\multicolumn{1}{c|}{\multirow{1}{*}{\textbf{Dia}}}&{Data}&{}&\multicolumn{1}{c|}{Leuco}&\multicolumn{1}{c|}{Plaq}&{Administrado}&{Rubrica} \\
    \hline
    \multicolumn{1}{c|}{\multirow{2}{*}{\textbf{64}}}&&{Etoposido \(100\) mg/m\(^2\)/dia)}&\multicolumn{1}{c|}{\(>10^3\)}&\multicolumn{1}{c|}{\(>10^5\)}&{(  ) Sim (  ) Não}&\\
    \cline{4-5}
    \multicolumn{1}{c|}{}&\multirow{1}{*}{}&{Cisplatina \(100\) mg/m\(^2\)/dia)}&&&{(  ) Sim (  ) Não}&\\
    \hline
    \multicolumn{1}{c|}{\multirow{3}{*}{\textbf{65}}}&&{Ciclofosfamida \(1000\) mg/m\(^2\)/dia}&{}&&{(  ) Sim (  ) Não}&\\
    \multicolumn{1}{c|}{}&&{MESNA \(200\) mg/m\(^2 \times 0,1,5\)h}&&&{(  ) Sim (  ) Não}&\\
    \multicolumn{1}{c|}{}&&{Etoposido \(100\) mg/m\(^2\)/dia)}&&&{(  ) Sim (  ) Não}&\\
    \hline
    \multicolumn{1}{c|}{\multirow{3}{*}{\textbf{66}}}&&{Ciclofosfamida \(1000\) mg/m\(^2\)/dia}&{}&&{(  ) Sim (  ) Não}&\\
    \multicolumn{1}{c|}{}&&{MESNA \(200\) mg/m\(^2 \times 0,1,5\)h}&&&{(  ) Sim (  ) Não}&\\
    \multicolumn{1}{c|}{}&\multirow{1}{*}{}&{Etoposido \(100\) mg/m\(^2\)/dia)}&{}&&{(  ) Sim (  ) Não}&\\
    \hline
    \multicolumn{1}{c|}{\textbf{67-76}}&&{G-CSF \(5 \mu g/ml\) }&&&{(  ) Sim (  ) Não}&\\
    \hline
\end{longtable}
\textit{\textbf{Final de Protocolo}}
\end{center}
\subsection{Modificações de dose:} 

Adiar se L < 1000/mm\(^3\) ou P < 100000/mm\(^3\). Se atraso maior que 7 dias, reduzir dose de ciclofosfamida em 20%.
Toxicidade grau 3-4 pela VCR, suspender dose seguinte. Reiniciar com dose normal. Recorrência: reduzir dose.
Bilirrunina total de 1,5-1,9 mg/dl, reduzir VCR para 1,0 mg/m2; se bilirrubina > 1,9 mg/dl, suspender uma dose de VCR.

Se o clearance de creatinina <50\% basal ou <60, suspender CDDP. No ciclo seguinte, se exames normalizados, fazer 50\% da dose de CDDP. Aumente novamente para 100\% somente no terceiro ciclo, se exames mantiverem-se normais.
Se ocorrer redução de 20dB ou mais em freqüências auditivas baixas (500-2000Hz), reduzir carboplatina em 50\%. Se ocorrer redução de 30dB na faixa de 4000-8000 Hz), reduzir carboplatina em 50\%. Ototoxicidade grau IV: interromper carboplatina até nível de lesão retornar ao grau II.

\textbf{Pacientes com superfície corpórea menor ou igual a 0,45m\textsuperscript{2}}:

Carboplatina: 12,5 mg/kg – Vincristina: 0,05 mg/kg – Ciclofosfamida: 33 mg/kg – MESNA: 7 mg/kg – Etoposido: 3,4 mg/kg

\textbf{ATENÇÃO:} este protocolo é \textit{off-label} (não padronizado) e não tem eficácia comprovada quando comparado com tratamento padrão sem QT. Dessa forma, é inadequado iniciar este protocolo em crianças com risco de complicações graves, como naquelas que têm sequelas importantes e muito limitantes.\\

\cleardoublepage
\chapter{Quimioterapia de resgate (doença recorrente/progressiva)}
\cleardoublepage
\section{IFOSFAMIDA/ETOPOSIDO - protocolo \textit{off-label}}
{\let\thefootnote\relax\footnotetext{Versão Janeiro/2017}}
\textbf{Racional:} o Pediatric Oncology Group (POG) publicou os resultados de um ensaio fase II que incluiu 294 pacientes com tumores sólidos previamente tratados (mais de 70\% metastáticos) que receberam uma mediana de 4 ciclos, dos quais 30\% obtiveram resposta parcial ou completa. Este ensaio não incluiu pacientes com tumores cerebrais \footnote{Kung \textit{et al}, 1993}. No entanto, esta combinação tem sido usada em vários protocolos no tratamento de pacientes pediátricos com tumores cerebrais, em vários cenários diferentes.

\textbf{Elegível:} Inclui meduloblastoma, outros tumores embrionários, ependimoma clássico ou anaplásico, gliomas difusos de linha média H3K27M+ (DIPG), glioblastoma multiforme, astrocitoma anaplásico, oligodendroglioma anaplásico, tumores malignos raros do SNC. Pacientes com doença recorrente e/ou progressiva após 1 ou mais esquemas de tratamento. Somente iniciar o protocolo após a assinatura do TERMO DE CONSENTIMENTO INFORMADO, conforme acordado. NÃO INICIAR ESTE PROTOCOLO EM CRIANÇAS GRAVEMENTE ENFERMAS.

\textbf{Alternativa:} não existe esquema de QT amplamente aceito para tratar crianças com tumores malignos do SNC recorrentes e/ou progressivos. Verificar a possibilidade de incluir o paciente em algum estudo investigacional. Outra alternativa é repetir RT craniana.

\entrywithlabel[.96\hsize]{\textbf{Nome}}\hfill \\

\entrywithlabel[.45\hsize]{\textbf{Peso}}\hfill  \entrywithlabel[.45\hsize]{\textbf{Estatura}}\\

\subsection{Resgate: 03 ciclos - repetir enquanto não houver progressão}

\renewcommand{\arraystretch}{1.5}

\begin{center}
\begin{longtable}{p{1cm}p{4cm}|p{1cm}|p{3cm}|p{3cm}}
	\hline
	\multicolumn{5}{c}{\textbf{CICLO 1}}\\
\hline
    \multicolumn{1}{c|}{\multirow{1}{*}{\textbf{Dia}}}&{Dose}&{Data}&{Administrado}&{Rubrica} \\
    \hline
    \multicolumn{1}{c|}{\multirow{1}{*}{\textbf{D1}}}&{Ifosfamida \(2000\) mg/m\(^2\)}&&{(  ) Sim (  ) Não}&\\
    \multicolumn{1}{c|}{\multirow{1}{*}{\textbf{}}}&{Etoposido \(100\) mg/m\(^2\)}&&{(  ) Sim (  ) Não}&\\
    \multicolumn{1}{c|}{\multirow{1}{*}{\textbf{}}}&{MESNA \(1500\) mg/m\(^2\)}&&{(  ) Sim (  ) Não}&\\
    \multicolumn{1}{c|}{\multirow{1}{*}{\textbf{D2}}}&{Ifosfamida \(2000\) mg/m\(^2\)}&&{(  ) Sim (  ) Não}&\\
    \multicolumn{1}{c|}{\multirow{1}{*}{\textbf{}}}&{Etoposido \(100\) mg/m\(^2\)}&&{(  ) Sim (  ) Não}&\\
    \multicolumn{1}{c|}{\multirow{1}{*}{\textbf{}}}&{MESNA \(1500\) mg/m\(^2\)}&&{(  ) Sim (  ) Não}&\\
    \multicolumn{1}{c|}{\multirow{1}{*}{\textbf{D3}}}&{Ifosfamida \(2000\) mg/m\(^2\)}&&{(  ) Sim (  ) Não}&\\
    \multicolumn{1}{c|}{\multirow{1}{*}{\textbf{}}}&{Etoposido \(100\) mg/m\(^2\)}&&{(  ) Sim (  ) Não}&\\
    \multicolumn{1}{c|}{\multirow{1}{*}{\textbf{}}}&{MESNA \(1500\) mg/m\(^2\)}&&{(  ) Sim (  ) Não}&\\

    \hline
    \multicolumn{1}{c|}{\multirow{2}{*}{\textbf{Exames}}}&\multicolumn{2}{l|}{Neut (\(>1,0\times10^3\)):}&{Plaq (\(>7,5\times10^4\)):}&{TGO:}\\
    \cline{2-5}
    \multicolumn{1}{c|}{\multirow{2}{*}{{}}}&\multicolumn{2}{l|}{Creat(\(<1,5\) vezes)}&{BT(\(<1,5\) vezes):}&{TGP:}
    \\
    \hline
\end{longtable}
\textbf{Intervalo de 14-21 dias.}
\end{center}
\clearpage
\begin{center}
\begin{longtable}{p{1cm}p{4cm}|p{1cm}|p{3cm}|p{3cm}}
	\hline
	\multicolumn{5}{c}{\textbf{CICLO 2}}\\
\hline
    \multicolumn{1}{c|}{\multirow{1}{*}{\textbf{Dia}}}&{Dose}&{Data}&{Administrado}&{Rubrica} \\
    \hline
    \multicolumn{1}{c|}{\multirow{1}{*}{\textbf{D1}}}&{Ifosfamida \(2000\) mg/m\(^2\)}&&{(  ) Sim (  ) Não}&\\
    \multicolumn{1}{c|}{\multirow{1}{*}{\textbf{}}}&{Etoposido \(100\) mg/m\(^2\)}&&{(  ) Sim (  ) Não}&\\
    \multicolumn{1}{c|}{\multirow{1}{*}{\textbf{}}}&{MESNA \(1500\) mg/m\(^2\)}&&{(  ) Sim (  ) Não}&\\
    \multicolumn{1}{c|}{\multirow{1}{*}{\textbf{D2}}}&{Ifosfamida \(2000\) mg/m\(^2\)}&&{(  ) Sim (  ) Não}&\\
    \multicolumn{1}{c|}{\multirow{1}{*}{\textbf{}}}&{Etoposido \(100\) mg/m\(^2\)}&&{(  ) Sim (  ) Não}&\\
    \multicolumn{1}{c|}{\multirow{1}{*}{\textbf{}}}&{MESNA \(1500\) mg/m\(^2\)}&&{(  ) Sim (  ) Não}&\\
    \multicolumn{1}{c|}{\multirow{1}{*}{\textbf{D3}}}&{Ifosfamida \(2000\) mg/m\(^2\)}&&{(  ) Sim (  ) Não}&\\
    \multicolumn{1}{c|}{\multirow{1}{*}{\textbf{}}}&{Etoposido \(100\) mg/m\(^2\)}&&{(  ) Sim (  ) Não}&\\
    \multicolumn{1}{c|}{\multirow{1}{*}{\textbf{}}}&{MESNA \(1500\) mg/m\(^2\)}&&{(  ) Sim (  ) Não}&\\

    \hline
    \multicolumn{1}{c|}{\multirow{2}{*}{\textbf{Exames}}}&\multicolumn{2}{l|}{Neut (\(>1,0\times10^3\)):}&{Plaq (\(>7,5\times10^4\)):}&{TGO:}\\
    \cline{2-5}
    \multicolumn{1}{c|}{\multirow{2}{*}{{}}}&\multicolumn{2}{l|}{Creat(\(<1,5\) vezes)}&{BT(\(<1,5\) vezes):}&{TGP:}
    \\
    \hline
\end{longtable}
\textbf{Intervalo de 14-21 dias.}
\end{center}
 
\begin{center}
\begin{longtable}{p{1cm}p{4cm}|p{1cm}|p{3cm}|p{3cm}}
	\hline
	\multicolumn{5}{c}{\textbf{CICLO 3}}\\
\hline
    \multicolumn{1}{c|}{\multirow{1}{*}{\textbf{Dia}}}&{Dose}&{Data}&{Administrado}&{Rubrica} \\
    \hline
    \multicolumn{1}{c|}{\multirow{1}{*}{\textbf{D1}}}&{Ifosfamida \(2000\) mg/m\(^2\)}&&{(  ) Sim (  ) Não}&\\
    \multicolumn{1}{c|}{\multirow{1}{*}{\textbf{}}}&{Etoposido \(100\) mg/m\(^2\)}&&{(  ) Sim (  ) Não}&\\
    \multicolumn{1}{c|}{\multirow{1}{*}{\textbf{}}}&{MESNA \(1500\) mg/m\(^2\)}&&{(  ) Sim (  ) Não}&\\
    \multicolumn{1}{c|}{\multirow{1}{*}{\textbf{D2}}}&{Ifosfamida \(2000\) mg/m\(^2\)}&&{(  ) Sim (  ) Não}&\\
    \multicolumn{1}{c|}{\multirow{1}{*}{\textbf{}}}&{Etoposido \(100\) mg/m\(^2\)}&&{(  ) Sim (  ) Não}&\\
    \multicolumn{1}{c|}{\multirow{1}{*}{\textbf{}}}&{MESNA \(1500\) mg/m\(^2\)}&&{(  ) Sim (  ) Não}&\\
    \multicolumn{1}{c|}{\multirow{1}{*}{\textbf{D3}}}&{Ifosfamida \(2000\) mg/m\(^2\)}&&{(  ) Sim (  ) Não}&\\
    \multicolumn{1}{c|}{\multirow{1}{*}{\textbf{}}}&{Etoposido \(100\) mg/m\(^2\)}&&{(  ) Sim (  ) Não}&\\
    \multicolumn{1}{c|}{\multirow{1}{*}{\textbf{}}}&{MESNA \(1500\) mg/m\(^2\)}&&{(  ) Sim (  ) Não}&\\

    \hline
    \multicolumn{1}{c|}{\multirow{2}{*}{\textbf{Exames}}}&\multicolumn{2}{l|}{Neut (\(>1,0\times10^3\)):}&{Plaq (\(>7,5\times10^4\)):}&{TGO:}\\
    \cline{2-5}
    \multicolumn{1}{c|}{\multirow{2}{*}{{}}}&\multicolumn{2}{l|}{Creat(\(<1,5\) vezes)}&{BT(\(<1,5\) vezes):}&{TGP:}
    \\
    \hline
\end{longtable}

\textbf{REAVALIAR A CADA 3 CICLOS}
\end{center}

\subsection{Modificações de dose:} 
Se atraso maior que 7 dias por toxicidade, reduzir os ciclos subsequentes em 25% (ambas as drogas).

\textbf{Avaliação:} imagem a cada 3 ciclos, se progressão, interromper protocolo.

\textbf{ATENÇÃO:} este protocolo é \textit{off-label} (não padronizado) e não tem eficácia comprovada quando comparado com tratamento padrão sem QT. Dessa forma, é inadequado iniciar este protocolo em crianças com risco de complicações graves, como naquelas que têm sequelas importantes e muito limitantes.\\
\cleardoublepage

\section{IFOSFAMIDA, CARBOPLATINA E ETOPOSIDO (ICE) - protocolo \textit{off-label}}
{\let\thefootnote\relax\footnotetext{Versão Janeiro/2017}}
\textbf{Racional:} o Children's Cancer Group (CCG) publicou os resultados de alguns ensaios fase II que incluíram, conjuntamente, 229 pacientes com tumores previamente tratados (43 com tumores cerebrais) que receberam uma mediana de 1 ciclo. Dos pacientes com sarcomas, cerca de 50\% obtiveram resposta parcial ou completa \footnote{Davenport \textit{et al}, 2000; Winkle \textit{et al}, 2005}.

\textbf{Elegível:} Inclui meduloblastoma, outros tumores embrionários, ependimoma clássico ou anaplásico, gliomas difusos de linha média H3K27M+ (DIPG), glioblastoma multiforme, astrocitoma anaplásico, oligodendroglioma anaplásico, tumores malignos raros do SNC. Pacientes com doença recorrente e/ou progressiva após 1 ou mais esquemas de tratamento. Somente iniciar o protocolo após a assinatura do TERMO DE CONSENTIMENTO INFORMADO, conforme acordado. NÃO INICIAR ESTE PROTOCOLO EM CRIANÇAS GRAVEMENTE ENFERMAS.

\textbf{Alternativa:} não existe esquema de QT amplamente aceito para tratar crianças com tumores malignos do SNC recorrentes e/ou progressivos. Verificar a possibilidade de incluir o paciente em algum estudo investigacional. Outra alternativa é repetir RT craniana.

\entrywithlabel[.96\hsize]{\textbf{Nome}}\hfill \\

\entrywithlabel[.45\hsize]{\textbf{Peso}}\hfill  \entrywithlabel[.45\hsize]{\textbf{Estatura}}

\subsection{Resgate: 02 ciclos - repetir enquanto não houver progressão}

\renewcommand{\arraystretch}{1.5}

\begin{center}
\begin{longtable}{p{1cm}p{4cm}|p{1cm}|p{3cm}|p{3cm}}
	\hline
	\multicolumn{5}{c}{\textbf{CICLO 1}}\\
\hline
    \multicolumn{1}{c|}{\multirow{1}{*}{\textbf{Dia}}}&{Dose}&{Data}&{Administrado}&{Rubrica} \\
    \hline
    \multicolumn{1}{c|}{\multirow{1}{*}{\textbf{D1}}}&{Ifosfamida \(1800\) mg/m\(^2\)}&&{(  ) Sim (  ) Não}&\\
    \multicolumn{1}{c|}{\multirow{1}{*}{\textbf{}}}&{Carboplatina \(400\) mg/m\(^2\)}&&{(  ) Sim (  ) Não}&\\
    \multicolumn{1}{c|}{\multirow{1}{*}{\textbf{}}}&{Etoposido \(100\) mg/m\(^2\)}&&{(  ) Sim (  ) Não}&\\
    \multicolumn{1}{c|}{\multirow{1}{*}{\textbf{}}}&{MESNA \(1500\) mg/m\(^2\)}&&{(  ) Sim (  ) Não}&\\
    \multicolumn{1}{c|}{\multirow{1}{*}{\textbf{D2}}}&{Ifosfamida \(1800\) mg/m\(^2\)}&&{(  ) Sim (  ) Não}&\\
    \multicolumn{1}{c|}{\multirow{1}{*}{\textbf{}}}&{Carboplatina \(400\) mg/m\(^2\)}&&{(  ) Sim (  ) Não}&\\
    \multicolumn{1}{c|}{\multirow{1}{*}{\textbf{}}}&{Etoposido \(100\) mg/m\(^2\)}&&{(  ) Sim (  ) Não}&\\
    \multicolumn{1}{c|}{\multirow{1}{*}{\textbf{}}}&{MESNA \(1500\) mg/m\(^2\)}&&{(  ) Sim (  ) Não}&\\
    \multicolumn{1}{c|}{\multirow{1}{*}{\textbf{D3}}}&{Ifosfamida \(1800\) mg/m\(^2\)}&&{(  ) Sim (  ) Não}&\\
    \multicolumn{1}{c|}{\multirow{1}{*}{\textbf{}}}&{Etoposido \(100\) mg/m\(^2\)}&&{(  ) Sim (  ) Não}&\\
    \multicolumn{1}{c|}{\multirow{1}{*}{\textbf{}}}&{MESNA \(1500\) mg/m\(^2\)}&&{(  ) Sim (  ) Não}&\\
    \multicolumn{1}{c|}{\multirow{1}{*}{\textbf{D4}}}&{Ifosfamida \(1800\) mg/m\(^2\)}&&{(  ) Sim (  ) Não}&\\
    \multicolumn{1}{c|}{\multirow{1}{*}{\textbf{}}}&{Etoposido \(100\) mg/m\(^2\)}&&{(  ) Sim (  ) Não}&\\
    \multicolumn{1}{c|}{\multirow{1}{*}{\textbf{}}}&{MESNA \(1500\) mg/m\(^2\)}&&{(  ) Sim (  ) Não}&\\
    \hline
    \multicolumn{1}{c|}{\multirow{1}{*}{\textbf{D5}}}&{Ifosfamida \(1800\) mg/m\(^2\)}&&{(  ) Sim (  ) Não}&\\
    \multicolumn{1}{c|}{\multirow{1}{*}{\textbf{}}}&{Etoposido \(100\) mg/m\(^2\)}&&{(  ) Sim (  ) Não}&\\
    \multicolumn{1}{c|}{\multirow{1}{*}{\textbf{}}}&{MESNA \(1500\) mg/m\(^2\)}&&{(  ) Sim (  ) Não}&\\
    \multicolumn{1}{c|}{\multirow{1}{*}{\textbf{D6 até recuperação}}}&{Filgrastima \(0,01\) mg/m\(^2\)}&&{(  ) Sim (  ) Não}&\\
    
    \hline
    \multicolumn{1}{c|}{\multirow{1}{*}{\textbf{Exames}}}&\multicolumn{2}{l|}{Neut (\(>1,0\times10^3\)):}&{Plaq (\(>7,5\times10^4\)):}&{}
    \\hline
\end{longtable}
\textbf{Intervalo de 21 dias.}
\end{center}

\begin{center}
\begin{longtable}{p{1cm}p{4cm}|p{1cm}|p{3cm}|p{3cm}}
	\hline
	\multicolumn{5}{c}{\textbf{CICLO 2}}\\
\hline
    \multicolumn{1}{c|}{\multirow{1}{*}{\textbf{Dia}}}&{Dose}&{Data}&{Administrado}&{Rubrica} \\
    \hline
    \multicolumn{1}{c|}{\multirow{1}{*}{\textbf{D1}}}&{Ifosfamida \(1800\) mg/m\(^2\)}&&{(  ) Sim (  ) Não}&\\
    \multicolumn{1}{c|}{\multirow{1}{*}{\textbf{}}}&{Carboplatina \(400\) mg/m\(^2\)}&&{(  ) Sim (  ) Não}&\\
    \multicolumn{1}{c|}{\multirow{1}{*}{\textbf{}}}&{Etoposido \(100\) mg/m\(^2\)}&&{(  ) Sim (  ) Não}&\\
    \multicolumn{1}{c|}{\multirow{1}{*}{\textbf{}}}&{MESNA \(1500\) mg/m\(^2\)}&&{(  ) Sim (  ) Não}&\\
    \multicolumn{1}{c|}{\multirow{1}{*}{\textbf{D2}}}&{Ifosfamida \(1800\) mg/m\(^2\)}&&{(  ) Sim (  ) Não}&\\
    \multicolumn{1}{c|}{\multirow{1}{*}{\textbf{}}}&{Carboplatina \(400\) mg/m\(^2\)}&&{(  ) Sim (  ) Não}&\\
    \multicolumn{1}{c|}{\multirow{1}{*}{\textbf{}}}&{Etoposido \(100\) mg/m\(^2\)}&&{(  ) Sim (  ) Não}&\\
    \multicolumn{1}{c|}{\multirow{1}{*}{\textbf{}}}&{MESNA \(1500\) mg/m\(^2\)}&&{(  ) Sim (  ) Não}&\\
    \multicolumn{1}{c|}{\multirow{1}{*}{\textbf{D3}}}&{Ifosfamida \(1800\) mg/m\(^2\)}&&{(  ) Sim (  ) Não}&\\
    \multicolumn{1}{c|}{\multirow{1}{*}{\textbf{}}}&{Etoposido \(100\) mg/m\(^2\)}&&{(  ) Sim (  ) Não}&\\
    \multicolumn{1}{c|}{\multirow{1}{*}{\textbf{}}}&{MESNA \(1500\) mg/m\(^2\)}&&{(  ) Sim (  ) Não}&\\
    \multicolumn{1}{c|}{\multirow{1}{*}{\textbf{D4}}}&{Ifosfamida \(1800\) mg/m\(^2\)}&&{(  ) Sim (  ) Não}&\\
    \multicolumn{1}{c|}{\multirow{1}{*}{\textbf{}}}&{Etoposido \(100\) mg/m\(^2\)}&&{(  ) Sim (  ) Não}&\\
    \multicolumn{1}{c|}{\multirow{1}{*}{\textbf{}}}&{MESNA \(1500\) mg/m\(^2\)}&&{(  ) Sim (  ) Não}&\\
    \multicolumn{1}{c|}{\multirow{1}{*}{\textbf{D5}}}&{Ifosfamida \(1800\) mg/m\(^2\)}&&{(  ) Sim (  ) Não}&\\
    \multicolumn{1}{c|}{\multirow{1}{*}{\textbf{}}}&{Etoposido \(100\) mg/m\(^2\)}&&{(  ) Sim (  ) Não}&\\
    \multicolumn{1}{c|}{\multirow{1}{*}{\textbf{}}}&{MESNA \(1500\) mg/m\(^2\)}&&{(  ) Sim (  ) Não}&\\
    \multicolumn{1}{c|}{\multirow{1}{*}{\textbf{D6 até recuperação}}}&{Filgrastima \(0,01\) mg/m\(^2\)}&&{(  ) Sim (  ) Não}&\\

    \hline
    \multicolumn{1}{c|}{\multirow{1}{*}{\textbf{Exames}}}&\multicolumn{2}{l|}{Neut (\(>1,0\times10^3\)):}&{Plaq (\(>7,5\times10^4\)):}&{}
    \\hline
\end{longtable}
\textbf{REAVALIAR A CADA 2 CICLOS}
\end{center}


\subsection{Modificações de dose:} 
Se recuperação hematológica nâo ocorrer até o D21, reduzir os ciclos subsequentes em 25% (todas as drogas). Caso não ocorra recuperação após o D35, o protocolo deve ser interrompido.

\textbf{Avaliação:} imagem a cada 4 ciclos, se progressão, interromper protocolo.

\textbf{ATENÇÃO:} este protocolo é \textit{off-label} (não padronizado) e não tem eficácia comprovada quando comparado com tratamento padrão sem QT. Dessa forma, é inadequado iniciar este protocolo em crianças com risco de complicações graves, como naquelas que têm sequelas importantes e muito limitantes.\\

\cleardoublepage

\chapter{Protocolos depreciados (substituídos por atualizações)}
\cleardoublepage
\section{GLIOMA DE ALTO GRAU E DIPG -- Adaptado do ensaio ACNS0126 - protocolo \textit{off-label}}
{\let\thefootnote\relax\footnotetext{Versão Janeiro/2015}}
\textbf{Racional:} no estudo do COG, a temozolomida (TMZ) não mostrou aumento de sobrevida em relação ao controle histórico (CCG-945), porém é proposta como alternativa principal por suas vantagens \footnote{Cohen \textit{et al}, 2011}. Deve-se levar em conta a disponibilidade e custo do esquema.

\textbf{Elegível:} extensão da ressecção cirúrgica (RNM de controle até 21 dias pós-op, preferido 72h após); não é necessário pesquisar metástases de rotina (RNM de neuro-eixo e PL/MO). Tratamento precisa iniciar até 42 dias após cirurgia. Inclui glioblastoma multiforme, astrocitoma anaplásico, oligodendroglioma anaplásico, gliossarcoma. Pacientes com mais de 3 anos de idade. Somente iniciar o protocolo após a assinatura do TERMO DE CONSENTIMENTO INFORMADO, conforme acordado. NÃO INICIAR ESTE PROTOCOLO EM CRIANÇAS GRAVEMENTE ENFERMAS.

\textbf{Alternativa:} não existe esquema de QT amplamente aceito para tratar crianças com gliomas de alto grau, incluindo gliomas pontinos intrínsecos difusos (DIPG). O tratamento padrão é RT craniana. Os pacientes com mais de 3 anos têm um prognóstico insatisfatório e sobrevida mediana de pouco mais de 12 meses.

\entrywithlabel[.96\hsize]{\textbf{Nome}}\hfill \\

\entrywithlabel[.45\hsize]{\textbf{Peso}}\hfill  \entrywithlabel[.45\hsize]{\textbf{Estatura}}\\

\subsection{Indução: 6 semanas (radioquimioterapia)}

\renewcommand{\arraystretch}{1.5}

\begin{center}
\begin{longtable}{p{1cm}c|c|c|c|c|c}
	\hline
\multicolumn{1}{c|}{\multirow{1}{*}{\textbf{Semana}}}&{Dose}&{Data}&{Neut}&{Plaq}&{Administrado}&{Rubrica} \\
    \hline
    \multicolumn{1}{c|}{\multirow{1}{*}{\textbf{1}}}&{Temozolomida 90mg/m\textsuperscript{2}/dia Seg-Sex}&{}&&&{(  ) Sim (  ) Não}&\\
    \cline{3-5}
    \multicolumn{1}{c|}{\multirow{1}{*}{{\textbf{2}}}}&{Temozolomida 90mg/m\textsuperscript{2}/dia Seg-Sex}&{}&&&{(  ) Sim (  ) Não}&\\
    \cline{3-5}
    \multicolumn{1}{c|}{\multirow{1}{*}{{\textbf{3}}}}&{Temozolomida 90mg/m\textsuperscript{2}/dia Seg-Sex}&{}&&&{(  ) Sim (  ) Não}&\\
    \cline{3-5}
    \multicolumn{1}{c|}{\multirow{1}{*}{{\textbf{4}}}}&{Temozolomida 90mg/m\textsuperscript{2}/dia Seg-Sex}&{}&&&{(  ) Sim (  ) Não}&\\
    \cline{3-5}
    \multicolumn{1}{c|}{\multirow{1}{*}{{\textbf{5}}}}&{Temozolomida 90mg/m\textsuperscript{2}/dia Seg-Sex}&{}&&&{(  ) Sim (  ) Não}&\\
    \cline{3-5}
    \multicolumn{1}{c|}{\multirow{1}{*}{{\textbf{6}}}}&{Temozolomida 90mg/m\textsuperscript{2}/dia Seg-Sex}&{}&&&{(  ) Sim (  ) Não}&\\
    \hline
\end{longtable}
\textbf{Intervalo de 28 dias.}
\end{center}
\clearpage
\subsection{Manutenção: 10 ciclos}

\begin{center}
\begin{longtable}{p{1cm}p{4cm}|p{1cm}|p{5cm}|p{3cm}}
	\hline
	\multicolumn{5}{c}{\textbf{CICLO 1}}\\
\hline
    \multicolumn{1}{c|}{\multirow{1}{*}{\textbf{Dia}}}&{Dose}&{Data}&{Administrado}&{Rubrica} \\
    \hline
    \multicolumn{1}{c|}{\multirow{1}{*}{\textbf{D1}}}&{Temozolomida \(200\) mg/m\(^2\)}&&{(  ) Sim (  ) Não}&\\
    \multicolumn{1}{c|}{\multirow{1}{*}{\textbf{D2}}}&{Temozolomida \(200\) mg/m\(^2\)}&&{(  ) Sim (  ) Não}&\\
    \multicolumn{1}{c|}{\multirow{1}{*}{\textbf{D3}}}&{Temozolomida \(200\) mg/m\(^2\)}&&{(  ) Sim (  ) Não}&\\
    \multicolumn{1}{c|}{\multirow{1}{*}{\textbf{D4}}}&{Temozolomida \(200\) mg/m\(^2\)}&&{(  ) Sim (  ) Não}&\\
    \multicolumn{1}{c|}{\multirow{1}{*}{\textbf{D5}}}&{Temozolomida \(200\) mg/m\(^2\)}&&{(  ) Sim (  ) Não}&\\
    \hline
    \multicolumn{1}{c|}{\multirow{2}{*}{\textbf{Exames}}}&\multicolumn{2}{l|}{Neut (\(>7,5\times10^2\)):}&{Plaq (\(>7,5\times10^4\)):}&{TGO:}\\
    \cline{2-5}
    \multicolumn{1}{c|}{\multirow{2}{*}{{}}}&\multicolumn{2}{l|}{Creat(\(<1,5\) vezes)}&{BT(\(<1,5\) vezes):}&{TGP:}
    \\
    \hline
\end{longtable}
\textbf{Intervalo de 14 dias.}
\begin{longtable}{p{5cm}|p{5cm}|p{5cm}}
    \hline
    \textbf{Exames (data):}&{Neut (\(>7,5\times10^2\)):}&{Plaq (\(>7,5\times10^4\)):}
    \\
    \hline
\end{longtable}
\textbf{Intervalo de 14 dias.}
\\[1.5cm]
\end{center}

\begin{center}
\begin{longtable}{p{1cm}p{4cm}|p{1cm}|p{5cm}|p{3cm}}
	\hline
	\multicolumn{5}{c}{\textbf{CICLO 2}}\\
\hline
    \multicolumn{1}{c|}{\multirow{1}{*}{\textbf{Dia}}}&{Dose}&{Data}&{Administrado}&{Rubrica} \\
    \hline
    \multicolumn{1}{c|}{\multirow{1}{*}{\textbf{D1}}}&{Temozolomida \(200\) mg/m\(^2\)}&&{(  ) Sim (  ) Não}&\\
    \multicolumn{1}{c|}{\multirow{1}{*}{\textbf{D2}}}&{Temozolomida \(200\) mg/m\(^2\)}&&{(  ) Sim (  ) Não}&\\
    \multicolumn{1}{c|}{\multirow{1}{*}{\textbf{D3}}}&{Temozolomida \(200\) mg/m\(^2\)}&&{(  ) Sim (  ) Não}&\\
    \multicolumn{1}{c|}{\multirow{1}{*}{\textbf{D4}}}&{Temozolomida \(200\) mg/m\(^2\)}&&{(  ) Sim (  ) Não}&\\
    \multicolumn{1}{c|}{\multirow{1}{*}{\textbf{D5}}}&{Temozolomida \(200\) mg/m\(^2\)}&&{(  ) Sim (  ) Não}&\\
    \hline
    \multicolumn{1}{c|}{\multirow{2}{*}{\textbf{Exames}}}&\multicolumn{2}{l|}{Neut (\(>7,5\times10^2\)):}&{Plaq (\(>7,5\times10^4\)):}&{TGO:}\\
    \cline{2-5}
    \multicolumn{1}{c|}{\multirow{2}{*}{{}}}&\multicolumn{2}{l|}{Creat(\(<1,5\) vezes)}&{BT(\(<1,5\) vezes):}&{TGP:}
    \\
    \hline
\end{longtable}
\textbf{Intervalo de 14 dias.}
\begin{longtable}{p{5cm}|p{5cm}|p{5cm}}
    \hline
    \textbf{Exames (data):}&{Neut (\(>7,5\times10^2\)):}&{Plaq (\(>7,5\times10^4\)):}
    \\
    \hline
\end{longtable}
\textbf{Intervalo de 14 dias.}
\end{center}
\clearpage
\begin{center}
\begin{longtable}{p{1cm}p{4cm}|p{1cm}|p{5cm}|p{3cm}}
	\hline
	\multicolumn{5}{c}{\textbf{CICLO 3}}\\
\hline
    \multicolumn{1}{c|}{\multirow{1}{*}{\textbf{Dia}}}&{Dose}&{Data}&{Administrado}&{Rubrica} \\
    \hline
    \multicolumn{1}{c|}{\multirow{1}{*}{\textbf{D1}}}&{Temozolomida \(200\) mg/m\(^2\)}&&{(  ) Sim (  ) Não}&\\
    \multicolumn{1}{c|}{\multirow{1}{*}{\textbf{D2}}}&{Temozolomida \(200\) mg/m\(^2\)}&&{(  ) Sim (  ) Não}&\\
    \multicolumn{1}{c|}{\multirow{1}{*}{\textbf{D3}}}&{Temozolomida \(200\) mg/m\(^2\)}&&{(  ) Sim (  ) Não}&\\
    \multicolumn{1}{c|}{\multirow{1}{*}{\textbf{D4}}}&{Temozolomida \(200\) mg/m\(^2\)}&&{(  ) Sim (  ) Não}&\\
    \multicolumn{1}{c|}{\multirow{1}{*}{\textbf{D5}}}&{Temozolomida \(200\) mg/m\(^2\)}&&{(  ) Sim (  ) Não}&\\
    \hline
    \multicolumn{1}{c|}{\multirow{2}{*}{\textbf{Exames}}}&\multicolumn{2}{l|}{Neut (\(>7,5\times10^2\)):}&{Plaq (\(>7,5\times10^4\)):}&{TGO:}\\
    \cline{2-5}
    \multicolumn{1}{c|}{\multirow{2}{*}{{}}}&\multicolumn{2}{l|}{Creat(\(<1,5\) vezes)}&{BT(\(<1,5\) vezes):}&{TGP:}
    \\
    \hline
\end{longtable}
\textbf{Intervalo de 14 dias.}
\begin{longtable}{p{5cm}|p{5cm}|p{5cm}}
    \hline
    \textbf{Exames (data):}&{Neut (\(>7,5\times10^2\)):}&{Plaq (\(>7,5\times10^4\)):}
    \\
    \hline
\end{longtable}
\textbf{Intervalo de 14 dias.}
\\[1.5cm]
\end{center}

\begin{center}
\begin{longtable}{p{1cm}p{4cm}|p{1cm}|p{5cm}|p{3cm}}
	\hline
	\multicolumn{5}{c}{\textbf{CICLO 4}}\\
\hline
    \multicolumn{1}{c|}{\multirow{1}{*}{\textbf{Dia}}}&{Dose}&{Data}&{Administrado}&{Rubrica} \\
    \hline
    \multicolumn{1}{c|}{\multirow{1}{*}{\textbf{D1}}}&{Temozolomida \(200\) mg/m\(^2\)}&&{(  ) Sim (  ) Não}&\\
    \multicolumn{1}{c|}{\multirow{1}{*}{\textbf{D2}}}&{Temozolomida \(200\) mg/m\(^2\)}&&{(  ) Sim (  ) Não}&\\
    \multicolumn{1}{c|}{\multirow{1}{*}{\textbf{D3}}}&{Temozolomida \(200\) mg/m\(^2\)}&&{(  ) Sim (  ) Não}&\\
    \multicolumn{1}{c|}{\multirow{1}{*}{\textbf{D4}}}&{Temozolomida \(200\) mg/m\(^2\)}&&{(  ) Sim (  ) Não}&\\
    \multicolumn{1}{c|}{\multirow{1}{*}{\textbf{D5}}}&{Temozolomida \(200\) mg/m\(^2\)}&&{(  ) Sim (  ) Não}&\\
    \hline
    \multicolumn{1}{c|}{\multirow{2}{*}{\textbf{Exames}}}&\multicolumn{2}{l|}{Neut (\(>7,5\times10^2\)):}&{Plaq (\(>7,5\times10^4\)):}&{TGO:}\\
    \cline{2-5}
    \multicolumn{1}{c|}{\multirow{2}{*}{{}}}&\multicolumn{2}{l|}{Creat(\(<1,5\) vezes)}&{BT(\(<1,5\) vezes):}&{TGP:}
    \\
    \hline
\end{longtable}
\textbf{Intervalo de 14 dias.}
\begin{longtable}{p{5cm}|p{5cm}|p{5cm}}
    \hline
    \textbf{Exames (data):}&{Neut (\(>7,5\times10^2\)):}&{Plaq (\(>7,5\times10^4\)):}
    \\
    \hline
\end{longtable}
\textbf{Intervalo de 14 dias.}
\end{center}
\clearpage
\begin{center}
\begin{longtable}{p{1cm}p{4cm}|p{1cm}|p{5cm}|p{3cm}}
	\hline
	\multicolumn{5}{c}{\textbf{CICLO 5}}\\
\hline
    \multicolumn{1}{c|}{\multirow{1}{*}{\textbf{Dia}}}&{Dose}&{Data}&{Administrado}&{Rubrica} \\
    \hline
    \multicolumn{1}{c|}{\multirow{1}{*}{\textbf{D1}}}&{Temozolomida \(200\) mg/m\(^2\)}&&{(  ) Sim (  ) Não}&\\
    \multicolumn{1}{c|}{\multirow{1}{*}{\textbf{D2}}}&{Temozolomida \(200\) mg/m\(^2\)}&&{(  ) Sim (  ) Não}&\\
    \multicolumn{1}{c|}{\multirow{1}{*}{\textbf{D3}}}&{Temozolomida \(200\) mg/m\(^2\)}&&{(  ) Sim (  ) Não}&\\
    \multicolumn{1}{c|}{\multirow{1}{*}{\textbf{D4}}}&{Temozolomida \(200\) mg/m\(^2\)}&&{(  ) Sim (  ) Não}&\\
    \multicolumn{1}{c|}{\multirow{1}{*}{\textbf{D5}}}&{Temozolomida \(200\) mg/m\(^2\)}&&{(  ) Sim (  ) Não}&\\
    \hline
    \multicolumn{1}{c|}{\multirow{2}{*}{\textbf{Exames}}}&\multicolumn{2}{l|}{Neut (\(>7,5\times10^2\)):}&{Plaq (\(>7,5\times10^4\)):}&{TGO:}\\
    \cline{2-5}
    \multicolumn{1}{c|}{\multirow{2}{*}{{}}}&\multicolumn{2}{l|}{Creat(\(<1,5\) vezes)}&{BT(\(<1,5\) vezes):}&{TGP:}
    \\
    \hline
\end{longtable}
\textbf{Intervalo de 14 dias.}
\begin{longtable}{p{5cm}|p{5cm}|p{5cm}}
    \hline
    \textbf{Exames (data):}&{Neut (\(>7,5\times10^2\)):}&{Plaq (\(>7,5\times10^4\)):}
    \\
    \hline
\end{longtable}
\textbf{Intervalo de 14 dias.}
\\[1.5cm]
\end{center}

\begin{center}
\begin{longtable}{p{1cm}p{4cm}|p{1cm}|p{5cm}|p{3cm}}
	\hline
	\multicolumn{5}{c}{\textbf{CICLO 6}}\\
\hline
    \multicolumn{1}{c|}{\multirow{1}{*}{\textbf{Dia}}}&{Dose}&{Data}&{Administrado}&{Rubrica} \\
    \hline
    \multicolumn{1}{c|}{\multirow{1}{*}{\textbf{D1}}}&{Temozolomida \(200\) mg/m\(^2\)}&&{(  ) Sim (  ) Não}&\\
    \multicolumn{1}{c|}{\multirow{1}{*}{\textbf{D2}}}&{Temozolomida \(200\) mg/m\(^2\)}&&{(  ) Sim (  ) Não}&\\
    \multicolumn{1}{c|}{\multirow{1}{*}{\textbf{D3}}}&{Temozolomida \(200\) mg/m\(^2\)}&&{(  ) Sim (  ) Não}&\\
    \multicolumn{1}{c|}{\multirow{1}{*}{\textbf{D4}}}&{Temozolomida \(200\) mg/m\(^2\)}&&{(  ) Sim (  ) Não}&\\
    \multicolumn{1}{c|}{\multirow{1}{*}{\textbf{D5}}}&{Temozolomida \(200\) mg/m\(^2\)}&&{(  ) Sim (  ) Não}&\\
    \hline
    \multicolumn{1}{c|}{\multirow{2}{*}{\textbf{Exames}}}&\multicolumn{2}{l|}{Neut (\(>7,5\times10^2\)):}&{Plaq (\(>7,5\times10^4\)):}&{TGO:}\\
    \cline{2-5}
    \multicolumn{1}{c|}{\multirow{2}{*}{{}}}&\multicolumn{2}{l|}{Creat(\(<1,5\) vezes)}&{BT(\(<1,5\) vezes):}&{TGP:}
    \\
    \hline
\end{longtable}
\textbf{Intervalo de 14 dias.}
\begin{longtable}{p{5cm}|p{5cm}|p{5cm}}
    \hline
    \textbf{Exames (data):}&{Neut (\(>7,5\times10^2\)):}&{Plaq (\(>7,5\times10^4\)):}
    \\
    \hline
\end{longtable}
\textbf{Intervalo de 14 dias.}
\end{center}
\clearpage
\begin{center}
\begin{longtable}{p{1cm}p{4cm}|p{1cm}|p{5cm}|p{3cm}}
	\hline
	\multicolumn{5}{c}{\textbf{CICLO 7}}\\
\hline
    \multicolumn{1}{c|}{\multirow{1}{*}{\textbf{Dia}}}&{Dose}&{Data}&{Administrado}&{Rubrica} \\
    \hline
    \multicolumn{1}{c|}{\multirow{1}{*}{\textbf{D1}}}&{Temozolomida \(200\) mg/m\(^2\)}&&{(  ) Sim (  ) Não}&\\
    \multicolumn{1}{c|}{\multirow{1}{*}{\textbf{D2}}}&{Temozolomida \(200\) mg/m\(^2\)}&&{(  ) Sim (  ) Não}&\\
    \multicolumn{1}{c|}{\multirow{1}{*}{\textbf{D3}}}&{Temozolomida \(200\) mg/m\(^2\)}&&{(  ) Sim (  ) Não}&\\
    \multicolumn{1}{c|}{\multirow{1}{*}{\textbf{D4}}}&{Temozolomida \(200\) mg/m\(^2\)}&&{(  ) Sim (  ) Não}&\\
    \multicolumn{1}{c|}{\multirow{1}{*}{\textbf{D5}}}&{Temozolomida \(200\) mg/m\(^2\)}&&{(  ) Sim (  ) Não}&\\
    \hline
    \multicolumn{1}{c|}{\multirow{2}{*}{\textbf{Exames}}}&\multicolumn{2}{l|}{Neut (\(>7,5\times10^2\)):}&{Plaq (\(>7,5\times10^4\)):}&{TGO:}\\
    \cline{2-5}
    \multicolumn{1}{c|}{\multirow{2}{*}{{}}}&\multicolumn{2}{l|}{Creat(\(<1,5\) vezes)}&{BT(\(<1,5\) vezes):}&{TGP:}
    \\
    \hline
\end{longtable}
\textbf{Intervalo de 14 dias.}
\begin{longtable}{p{5cm}|p{5cm}|p{5cm}}
    \hline
    \textbf{Exames (data):}&{Neut (\(>7,5\times10^2\)):}&{Plaq (\(>7,5\times10^4\)):}
    \\
    \hline
\end{longtable}
\textbf{Intervalo de 14 dias.}
\\[1.5cm]
\end{center}

\begin{center}
\begin{longtable}{p{1cm}p{4cm}|p{1cm}|p{5cm}|p{3cm}}
	\hline
	\multicolumn{5}{c}{\textbf{CICLO 8}}\\
\hline
    \multicolumn{1}{c|}{\multirow{1}{*}{\textbf{Dia}}}&{Dose}&{Data}&{Administrado}&{Rubrica} \\
    \hline
    \multicolumn{1}{c|}{\multirow{1}{*}{\textbf{D1}}}&{Temozolomida \(200\) mg/m\(^2\)}&&{(  ) Sim (  ) Não}&\\
    \multicolumn{1}{c|}{\multirow{1}{*}{\textbf{D2}}}&{Temozolomida \(200\) mg/m\(^2\)}&&{(  ) Sim (  ) Não}&\\
    \multicolumn{1}{c|}{\multirow{1}{*}{\textbf{D3}}}&{Temozolomida \(200\) mg/m\(^2\)}&&{(  ) Sim (  ) Não}&\\
    \multicolumn{1}{c|}{\multirow{1}{*}{\textbf{D4}}}&{Temozolomida \(200\) mg/m\(^2\)}&&{(  ) Sim (  ) Não}&\\
    \multicolumn{1}{c|}{\multirow{1}{*}{\textbf{D5}}}&{Temozolomida \(200\) mg/m\(^2\)}&&{(  ) Sim (  ) Não}&\\
    \hline
    \multicolumn{1}{c|}{\multirow{2}{*}{\textbf{Exames}}}&\multicolumn{2}{l|}{Neut (\(>7,5\times10^2\)):}&{Plaq (\(>7,5\times10^4\)):}&{TGO:}\\
    \cline{2-5}
    \multicolumn{1}{c|}{\multirow{2}{*}{{}}}&\multicolumn{2}{l|}{Creat(\(<1,5\) vezes)}&{BT(\(<1,5\) vezes):}&{TGP:}
    \\
    \hline
\end{longtable}
\textbf{Intervalo de 14 dias.}
\begin{longtable}{p{5cm}|p{5cm}|p{5cm}}
    \hline
    \textbf{Exames (data):}&{Neut (\(>7,5\times10^2\)):}&{Plaq (\(>7,5\times10^4\)):}
    \\
    \hline
\end{longtable}
\textbf{Intervalo de 14 dias.}
\end{center}
\clearpage
\begin{center}
\begin{longtable}{p{1cm}p{4cm}|p{1cm}|p{5cm}|p{3cm}}
	\hline
	\multicolumn{5}{c}{\textbf{CICLO 9}}\\
\hline
    \multicolumn{1}{c|}{\multirow{1}{*}{\textbf{Dia}}}&{Dose}&{Data}&{Administrado}&{Rubrica} \\
    \hline
    \multicolumn{1}{c|}{\multirow{1}{*}{\textbf{D1}}}&{Temozolomida \(200\) mg/m\(^2\)}&&{(  ) Sim (  ) Não}&\\
    \multicolumn{1}{c|}{\multirow{1}{*}{\textbf{D2}}}&{Temozolomida \(200\) mg/m\(^2\)}&&{(  ) Sim (  ) Não}&\\
    \multicolumn{1}{c|}{\multirow{1}{*}{\textbf{D3}}}&{Temozolomida \(200\) mg/m\(^2\)}&&{(  ) Sim (  ) Não}&\\
    \multicolumn{1}{c|}{\multirow{1}{*}{\textbf{D4}}}&{Temozolomida \(200\) mg/m\(^2\)}&&{(  ) Sim (  ) Não}&\\
    \multicolumn{1}{c|}{\multirow{1}{*}{\textbf{D5}}}&{Temozolomida \(200\) mg/m\(^2\)}&&{(  ) Sim (  ) Não}&\\
    \hline
    \multicolumn{1}{c|}{\multirow{2}{*}{\textbf{Exames}}}&\multicolumn{2}{l|}{Neut (\(>7,5\times10^2\)):}&{Plaq (\(>7,5\times10^4\)):}&{TGO:}\\
    \cline{2-5}
    \multicolumn{1}{c|}{\multirow{2}{*}{{}}}&\multicolumn{2}{l|}{Creat(\(<1,5\) vezes)}&{BT(\(<1,5\) vezes):}&{TGP:}
    \\
    \hline
\end{longtable}
\textbf{Intervalo de 14 dias.}
\begin{longtable}{p{5cm}|p{5cm}|p{5cm}}
    \hline
    \textbf{Exames (data):}&{Neut (\(>7,5\times10^2\)):}&{Plaq (\(>7,5\times10^4\)):}
    \\
    \hline
\end{longtable}
\textbf{Intervalo de 14 dias.}
\end{center}

\begin{center}
\begin{longtable}{p{1cm}p{4cm}|p{1cm}|p{5cm}|p{3cm}}
	\hline
	\multicolumn{5}{c}{\textbf{CICLO 10}}\\
\hline
    \multicolumn{1}{c|}{\multirow{1}{*}{\textbf{Dia}}}&{Dose}&{Data}&{Administrado}&{Rubrica} \\
    \hline
    \multicolumn{1}{c|}{\multirow{1}{*}{\textbf{D1}}}&{Temozolomida \(200\) mg/m\(^2\)}&&{(  ) Sim (  ) Não}&\\
    \multicolumn{1}{c|}{\multirow{1}{*}{\textbf{D2}}}&{Temozolomida \(200\) mg/m\(^2\)}&&{(  ) Sim (  ) Não}&\\
    \multicolumn{1}{c|}{\multirow{1}{*}{\textbf{D3}}}&{Temozolomida \(200\) mg/m\(^2\)}&&{(  ) Sim (  ) Não}&\\
    \multicolumn{1}{c|}{\multirow{1}{*}{\textbf{D4}}}&{Temozolomida \(200\) mg/m\(^2\)}&&{(  ) Sim (  ) Não}&\\
    \multicolumn{1}{c|}{\multirow{1}{*}{\textbf{D5}}}&{Temozolomida \(200\) mg/m\(^2\)}&&{(  ) Sim (  ) Não}&\\
    \hline
    \multicolumn{1}{c|}{\multirow{2}{*}{\textbf{Exames}}}&\multicolumn{2}{l|}{Neut (\(>7,5\times10^2\)):}&{Plaq (\(>7,5\times10^4\)):}&{TGO:}\\
    \cline{2-5}
    \multicolumn{1}{c|}{\multirow{2}{*}{{}}}&\multicolumn{2}{l|}{Creat(\(<1,5\) vezes)}&{BT(\(<1,5\) vezes):}&{TGP:}
    \\
    \hline
\end{longtable}
\textbf{Intervalo de 14 dias.}
\begin{longtable}{p{5cm}|p{5cm}|p{5cm}}
    \hline
    \textbf{Exames (data):}&{Neut (\(>7,5\times10^2\)):}&{Plaq (\(>7,5\times10^4\)):}
    \\
    \hline
\end{longtable}
 
\textbf{FIM DE PROTOCOLO}

\end{center}
\subsection{Modificações de dose:} 
Se atraso maior que 7 dias por toxicidade, reduzir os ciclos subsequentes para 150 mg/m\textsuperscript{2}/dia

\textbf{Avaliação:} imagem a cada 3 ciclos (3 meses), se progressão, interromper protocolo.

\textbf{APRESENTAÇÕES DE TEMOZOLOMIDA NO HIAS:} cápsulas de 100mg

\textbf{ADVERTÊNCIA:} SMZ+TMP não deve ser administrada juntamente com a temozolomida!

\textbf{ATENÇÃO:} este protocolo é \textit{off-label} (não padronizado) e não tem eficácia comprovada quando comparado com tratamento padrão sem QT. Dessa forma, é inadequado iniciar este protocolo em crianças com risco de complicações graves, como naquelas que têm sequelas importantes e muito limitantes.\\

\end{document}
