
% RECOMMENDED %%%%%%%%%%%%%%%%%%%%%%%%%%%%%%%%%%%%%%%%%%%%%%%%%%%
\documentclass[graybox]{svmult}

% choose options for [] as required from the list
% in the Reference Guide

\usepackage{mathptmx}       % selects Times Roman as basic font
\usepackage{helvet}         % selects Helvetica as sans-serif font
\usepackage{courier}        % selects Courier as typewriter font
\usepackage{type1cm}        % activate if the above 3 fonts are
                            % not available on your system
%
\usepackage{makeidx}         % allows index generation
\usepackage{graphicx}        % standard LaTeX graphics tool
                             % when including figure files
\usepackage{multicol}        % used for the two-column index
\usepackage[bottom]{footmisc}% places footnotes at page bottom

% see the list of further useful packages
% in the Reference Guide

\makeindex             % used for the subject index
                       % please use the style svind.ist with
                       % your makeindex program

%%%%%%%%%%%%%%%%%%%%%%%%%%%%%%%%%%%%%%%%%%%%%%%%%%%%%%%%%%%%%%%%%%%%%%%%%%%%%%%%%%%%%%%%%
\usepackage{multirow}

\begin{document}

\title*{Informações sobre as drogas utilizadas}
% Use \titlerunning{Short Title} for an abbreviated version of
% your contribution title if the original one is too long
\author{Francisco Hélder Cavalcante Félix e Juvenia Bezerra Fontenele}
% Use \authorrunning{Short Title} for an abbreviated version of
% your contribution title if the original one is too long
\institute{Francisco Hélder Cavalcante Félix \at Centro Pediátrico do Câncer, Hospital Infantil Albert Sabin, R. Alberto Montezuma, 350, 60410-780, Fortaleza - CE \email{fhcflx@outlook.com}
\and Juvenia Bezerra Fontenele \at Faculdade de Farmácia, Odontologia e Enfermagem, Universidade Federal do Ceará, R. Alexandre Baraúna, 949, 60430-160, Fortaleza - CE %\email{juvenia.fontenele@gmail.com}
}
%
% Use the package "url.sty" to avoid
% problems with special characters
% used in your e-mail or web address
%
\maketitle

\abstract{}

\section{Carboplatina}

Aprovada pela ANVISA para uso adulto. Sua utilização em crianças e adolescentes é não padronizada (\textit{off-label}). Indicações: carcinoma de ovário, carcinoma de pequenas células de pulmão, carcinomas espino-celulares de cabeça e pescoço e carcinomas de cérvice uterina. Outros usos, como no tratamento de tumores cerebrais, são não padronizados (\textit{off-label}). Contra-indicações: hipersensibilidade à droga ou outros constituintes da fórmula ou à cisplatina; insuficiência renal grave, mielodepressão grave e/ou na presença de sangramento volumoso. Não deve ser usada durante a gravidez e lactação (dados da bula aprovada epla ANVISA).

Seu mecanismo de ação é semelhante ao da cisplatina. Liga-se ao ADN em replicação causando quebras de fita simples e ligações cruzadas interfilamentares. \(\alpha t_\frac{1}{2}\) \footnote{Meia-vida de distribuição} e \(\beta t_\frac{1}{2}\) \footnote{Meia-vida de eliminação} são, respectivamente, 1,1 a 2 horas e 2,6 a 5,9 horas. Não liga-se a proteínas. Via de eliminação principal: renal. Precisa de ajuste de dose na insuficiência renal.

Toxicidade:
\renewcommand{\labelenumi}{\Alph{enumi}}
\begin{enumerate}
	\item Imediata: primeiras 24-48h. Comum (21-100\%): náuseas e vômitos. Ocasional (5-20\%): hipersensibilidade, constipação, diarreía. Raro (<5\%): rash, mucosite.
	\item Subaguda: 2-3 semanas. Comum: mielossupressão, distúrbios eletrolíticos. Ocasional: alteração de função hepática e/ou renal, dor abdominal, astenia. Raro: hiperbilirrubinemia.
	\item Retardada: qualquer momento após tratamento. Ocasional: ototoxicidade. Raro: alopécia, neuropatia periférica, amaurose noturna e/ou a cores; leucemia secundária.
\end{enumerate}

Formulação e estabilidade: 
		
Frascos de solução aquosa pré-diluída ou pó liofilizado para diluição, variando de 150 a 600 mg por frasco. Diluição padrão 10 mg/ml (checar disponibilidade). Checar apresentação sobre validade. Reconstituir em água para injeção, SG5\% ou SF0,9\% a 10mg/ml. Pode ser rediluída em SG5\% ou SF0,9\% até 0,5 mg/ml, permanencendo estável por até 8h a 25\(^\circ\) C. Alumínio reage com a carboplatina e pode causar sua precipitação. Evitar dispositivos de injeção com partes de alumínio. Proteger da luz.

\section{Cisplatina}

Aprovada pela ANVISA para uso adulto e pediátrico. Indicações: tumores metastáticos de testículo, tumores metastáticos de ovário, câncer avançado de bexiga, carcinomas espino-celulares de cabeça e pescoço. Outros usos, como no tratamento de tumores cerebrais, são não padronizados (\textit{off-label}). Contra-indicações: hipersensibilidade à droga ou a componentes da fórmula, mielodepressão, insuficiência renal grave, distúrbios de audição, infecções generalizadas. Não deve ser usada na gravidez ou lactação (dados da bula aprovada pela ANVISA). 

	Complexo inorgânico hidrossolúvel contendo um átomo central de platina, 2 átomos de cloro e 2 moléculas de amônia. Em solução aquosa, os átomos de cloro são lentamente deslocados para a solução, gerando um complexo hidratado positivamente carregado. Este complexo ativado reage com sítios nucleofílicos do ADN, ARN e proteínas, resultando na formação de ligações covalentes bifuncionais. As ligações cruzadas intrafilamentares entre citosina e guanina são as principais responsáveis pela inibição da síntese de ADN. Tem citotoxicidade sinérgica com radiação e outros agentes quimioterápicos. Sua distribuição é rápida (25-80 min) e sua meia-vida de eliminação é de 60-70 h. A platina (mas não a cisplatina em si) liga-se a proteínas plasmáticas (90\% três horas após injeção). A meia-vida de eliminação da platina ligada á albumina é de 5 dias ou mais. Pode-se encontrar platina nos tecidos até 180 dias após a última admisntração. A excreção é renal e a penetração no SNC é mínima.

Toxicidade:
\renewcommand{\labelenumi}{\Alph{enumi}}
\begin{enumerate}
	\item Imediata: primeiras 24-48h. Comum (21-100\%): náuseas e vômitos. Ocasional (5-20\%): sabor metálico. Raro (<5\%): anafilaxia, flebite, ulceração por extravasamento (se [] > 0,5 mg/ml).
	\item Subaguda: 2-3 semanas. Comum: mielossupressão, hipomagnesemia, perda auditiva de frequências altas, nefrotoxicidade. Ocasional: distúrbios eletrolíticos, neuropatia periférica. Raro: vestibulopatia, zumbido, rash, convulsões, disfunção hepática.
	\item Retardada: qualquer momento após tratamento. Ocasional: perda auditiva na faixa de frequências médias. Raro: arreflexia, perda da propriocepção e sensação vibratória, amaurose central, turvação visual, amaurose de cores, falência renal crônica, malignidade secundária.
	\item Outros: teratogênica em animais, excretada no leite materno.
\end{enumerate}

Formulação e estabilidade:

	Disponível em solução pré-diluída a 1 mg/ml de cisplatina e 9 mg/ml de cloreto de sódio. Armazenar a 15-25\(^\circ\) C. NÃO REFRIGERAR! Proteger da luz. Pode ser rediluída em SGF ou SF0,9\%, desde que a solução contenha > 0,2\% de NaCl. Soluções contendo dextrose, salina e/ou manitol são estáveis por 24-72h, porém não podem ser refrigeradas. Imcompatível com bicarbonato e soluções alcalinas. Alumínio reage com a cisplatina e pode causar sua precipitação. Evitar dispositivos de injeção com partes de alumínio. 

\bibliographystyle{unsrt}
\bibliography{cpc-neuro2014/bib}

\end{document}